\chapter{Magnetic bead imaging}\label{sec:magneticBeadImaging}
\section{SEM images of magnetic beads}
All magnetic bead types have been imaged using scanning electron microscopy (SEM). The SEM images are used to visualize the particles' shape and to estimate their size distribution. The following pictures show the SEM images of the various particle types. Similar images can be found in the literature~\cite{Herrasti2016}.
\begin{figure}[htb]
	\centering
    \begin{subfigure}[b]{\textwidth}
	    	\centering
		\includegraphics[width = 0.45\textwidth]{img/chapters/appendix/screenMag_01_i002.jpg}
		\includegraphics[width = 0.45\textwidth]{img/chapters/appendix/screenMag_01_i010.jpg}
        \caption{}
	\end{subfigure}
    	\begin{subfigure}[b]{\textwidth}
    		\centering
		\includegraphics[width = 0.45\textwidth]{img/chapters/appendix/siMag_01_i001.png}
		\includegraphics[width = 0.45\textwidth]{img/chapters/appendix/siMag_01_i002.png}
		\caption{}
	\end{subfigure}
	\caption[SEM images of magnetic Chemicell beads]{SEM images of magnetic Chemicell beads. (a) Chemicell ScreenMAG beads. (b) Chemicell SiMAG beads.}
\label{fig:}
\end{figure}

\begin{figure}[htb]
	\centering
	\begin{subfigure}[b]{\textwidth}
		\centering
		\includegraphics[width = 0.45\textwidth]{img/chapters/appendix/myOne_01_i008.png}
		\includegraphics[width = 0.45\textwidth]{img/chapters/appendix/myOne_01_i035.png}
		\caption{}
	\end{subfigure}
	\begin{subfigure}[b]{\textwidth}
		\centering
		\includegraphics[width = 0.45\textwidth]{img/chapters/appendix/m280_01_i017.png}
		\includegraphics[width = 0.45\textwidth]{img/chapters/appendix/m280_01_i023.png}
		\caption{}
	\end{subfigure}
	\begin{subfigure}[b]{\textwidth}
		\centering
		\includegraphics[width = 0.45\textwidth]{img/chapters/appendix/m270_04_i029.jpg}
		\includegraphics[width = 0.45\textwidth]{img/chapters/appendix/m270_04_i033.jpg}
		\caption{}
	\end{subfigure}
	\caption[SEM images of magnetic Dynabead microspheres]{SEM images of magnetic Dynabead microspheres. (a) Dynabeads MyOne. (b) Dynabeads M280. (c) Dynabeads M270.}
\end{figure}

\newpage

\chapter{Magnetic bead separation}\label{sec:magneticBeadSeparationAppendix}
\section{Magnetic bead separation in continuous flow}
Experimental results of the continuous separation using the Double Magnet configuration. The flow rate ratio was kept as close as possible to $1:1:1$ for all flow rates.
\begin{figure}[htb!]
	\centering
	\includegraphics[width=0.59\textwidth]{img/chapters/appendix/3d_bar_plot_legends.png}
	\label{fig:continuousSeparationEfficiency}
	\caption[Comparison of the magnetic bead separation efficiency in continuous flow with and without the Double Magnet configuration]{Comparison of the magnetic bead separation efficiency in continuous flow. The bar plot shows the relative separation of the iBidi $\mu$-Slide III 3in3 with and without the double magnet configuration in place. The flow rate is the sum of all inlet flow rates.}
\end{figure}

\newpage

\chapter{Superparamagnetic microspheres in an external magnetic field}\label{sec:superparamagneticMicrospheresInAnExternalMagneticField}

\section{Interaction between single superparamagnetic microspheres}
Magnetic particles induce a magnetic dipole when exposed to an external magnetic field. The induced magnetic dipoles of the particles point all in the same direction. Thus, the magnetic particles experience a repellent force ($F_{p\ast}$) between each other. This phenomena can be seen in Figure~\ref{fig:interactionBetweenSingleSuperparamagneticMicrospheres} where the single magnetic particles have all approximately the same distance to each other. The inset in Figure~\ref{fig:interactionBetweenSingleSuperparamagneticMicrospheres} schematically shows the repelling force between two particles. 
\begin{figure}[htb]
	\centering
	\includegraphics[width=0.7\textwidth]{img/chapters/appendix/magneticParticleInMagneticField_50x.pdf}
	\label{fig:interactionBetweenSingleSuperparamagneticMicrospheres}
	\caption[Repellent force between individual dipoles]{Magnetic particles (Dynabeads M270) induce a magnetic dipole once they are exposed to an external magnetic field ($H_{ext}$). The induced dipoles of the particles point all in the same direction (out of plane). This causes a repelling magnetic force between all magnetic particles, which prevents them from touching each other. The inset schematically shows the induced magnetic dipole ($M$) and the direction of the repelling force, $F_{p\ast}$, the two particles experience when getting closer.}
\end{figure}

\section{Interaction between superparamagnetic agglomerations}
The magnetic particles formed mostly chains but also more complicated 3D structures when being exposed to an external magnetic field, as seen in Figure~\ref{fig:interactionBetweenSuperparamagneticAgglomerates}. The fact that the chains and larger agglomerates go out of focus shows that the particles align along the field lines of the external magnetic field $H_{ext}$.
\begin{figure}[htb]
	\centering
	\begin{subfigure}[b]{0.48\textwidth}
		\includegraphics[width = \textwidth]{img/chapters/appendix/magneticParticleInMagneticFieldAgglomerates_10x.pdf}
		\caption{$10\times$ magnification}
		\label{fig:interactionBetweenSuperparamagneticAgglomerates10x}
	\end{subfigure}
	\hfill
	\begin{subfigure}[b]{0.48\textwidth}
		\includegraphics[width = \textwidth]{img/chapters/appendix/magneticParticleInMagneticFieldAgglomerates_20x.pdf}
		\caption{$20\times$ magnification}
		\label{fig:interactionBetweenSuperparamagneticAgglomerates20x}
	\end{subfigure}
	\label{fig:interactionBetweenSuperparamagneticAgglomerates}
	\caption[Interaction between magnetic particle and particle agglomerates]{(a) Magnetic particles interact with each other and form chains or more complicated 3D structures. (b) Magnified view of the framed section in Figure~\ref{fig:interactionBetweenSuperparamagneticAgglomerates10x}. The particle agglomerates are mostly out of focus since they align along the field lines of the external magnetic field, $H_{ext}$, which points out of plane.}
\end{figure}
