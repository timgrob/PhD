\chapter{Hydrodynamic flow in a microfluidic device}\label{ch:hydrodynamicFlowInAMicrofluidicDevice}

\section{Introduction}
The hydrodynamic flow in the two microfluidic devices described in Section~\ref{subsec:microfluidicHydrodynamicFocusingChannel} and Section~\ref{subsec:microfluidicSeparationDevice} have been numerically modelled using the simulation software ANSYS Fluent and experimentally tested and verified. Here, only results for the microfluidic separation device (see Section~\ref{subsec:microfluidicSeparationDevice}) are presented, since this device will be used further throughout this thesis.

\section{Hydrodynamic flow in a microfluidic separation device}\label{sec:hydrodynamicFlowInAMicrofluidicSseparationDevice}
The solution to the Stokes equation (Equation~\ref{eqn:stokesEquation}) in a straight and rigid channel is the pressure-driven, steady-state flow, also known as Poiseuille flow. Figure~\ref{fig:velocityProfileInMicrofluidicSeparationDevice} shows the 3D velocity profile of the Poiseuille flow across the rectangular cross section of the microfluidic separation device at a Reynolds number of approximately $\text{Re}\approx 0.3$. The dimensions of the fluid channel in the simulation are set equal to the actual dimensions of the microfluidic separation device (iBidi $\mu$-Slide III 3in3) as described in Section~\ref{subsec:microfluidicSeparationDevice}. In Figure~\ref{fig:velocityProfileInMicrofluidicSeparationDevice} the analytical solution based on Equation~\ref{eqn:velocityProfileRectangularDuct} as well as the numerical simulation is shown. For the numerical simulation, however, only half of the channel height is computed.

\begin{figure}[htb]
\centering
   \includegraphics[width=0.8\textwidth]{img/chapters/chapter_5_hydrodynamics/velocityProfile3D_theory_and_simulation_ibidiDevice_cgr.eps}
\caption[Velocity profile across half the cross-section of the microfluidic separation device]{Velocity profile across half the cross-section of the microfluidic separation device at Re $\approx 0.3$. The continuous profile shows the analytical solution. The black dots are the results of the numerical solution computed with ANSYS Fluent. For the numerical simulation, only half of the channel height is simulated due to symmetry. The channel dimensions are representative of the microfluidic separation device (iBidi $\mu$-Slide III 3in3) described in Section~\ref{subsec:microfluidicSeparationDevice}. Note, that the cross section of the microfluidic separation device has a large aspect ratio, which the illustration does not represent accurately.}
\label{fig:velocityProfileInMicrofluidicSeparationDevice}%
\end{figure} 

This particular flow profile is of major importance for the understanding of liquid handling and thus particle manipulation in suspension in microfluidic systems. An important characteristic of the fluid dynamics is the parabolic flow through a rectangular channel and the resultant zero velocity at the internal surfaces of the channel (no-slip) as described in Chapter~\ref{ch:chapter2_theory}. Note, how the flow profile in the rectangular case remains constant along a large part of the channel's wide width, but parabolic across the short height. The shape of the velocity profile, and whether it has a parabolic shape in all directions or tends towards a plug-like form (similar to the electroosmotic flow~\cite{Yang2005}), depends on the aspect ratio ($h_{0}/w_{0}$) of the channel~\cite{Lee2006}. The plug-like flow profile seen here is also well documented in the literature for different channel cross sections~\cite{White2006,Bruus2007}. 

The magnetic beads experience a force from the magnetic field and a force from the fluid. The pull towards the magnet must be sufficient for the magnetic beads to be separated from the main sample flow and drawn into the sheath flow, whilst beads should be prevented from arriving at the internal surfaces of the fluidic channel. Beads at the channel surfaces will experience almost zero drag and thus will not flow out of the fluidic device. Such beads will be considered as \textit{lost}, as they are difficult to recover using continuous flow at low flow rates. In order to minimise beads arriving at an internal surface and to control their path through the device, the fluid stream can be manipulated using hydrodynamic focusing as described in the next section.

\section{Hydrodynamic focusing in a microfluidic separation device}\label{sec:hydrodynamicFocusingInAMicrofluidicSeparationDevice}
The hydrodynamic focusing effect is simulated and experimentally tested for different flow rates as well as different inlet flow rate ratios.  The inlet flow rate ratio describes the ratio between the inlet flow rate of one of the sheath flow channels to the inlet flow rate of the centre channel. In this work, the two sheath flows are always identical. 
%All simulations and experiments are performed in the microfluidic iBidi separation device described in Section~\ref{subsec:microfluidicSeparationDevice}.

Figure~\ref{fig:hydrodynamicFocusingSimulationAndExperiments} shows a comparison of the numerical simulations and the experiments of the hydrodynamic focusing for different flow rates. In order to make the sample stream visible, the water is dyed with black ink. It can be seen that the simulations accurately represent the experimentally found results for different aspect ratios, which also verifies the results of the numerical simulation.

\begin{figure}[htb!]
	\centering
	\begin{subfigure}[b]{0.4\textwidth}
				\includegraphics[width=\textwidth]{img/chapters/chapter_5_hydrodynamics/HDF_Simulation_01_01_01_mod-1.png}
				\caption{1:1}
	\end{subfigure}
	\begin{subfigure}[b]{0.4\textwidth}
				\includegraphics[width=\textwidth]{img/chapters/chapter_5_hydrodynamics/100_100_100-2.pdf}
				\caption{1:1}
    \end{subfigure}
        %%%%%%%%%%%%%%%%
	\begin{subfigure}[b]{0.4\textwidth}
				\includegraphics[width=\textwidth]{img/chapters/chapter_5_hydrodynamics/HDF_Simulation_02_01_02_mod-1.png}
				\caption{2:1}
    \end{subfigure}
	\begin{subfigure}[b]{0.4\textwidth}
				\includegraphics[width=\textwidth]{img/chapters/chapter_5_hydrodynamics/200_100_200-2.pdf}
				\caption{2:1}
    \end{subfigure}
        %%%%%%%%%%%%%%%%
	\begin{subfigure}[b]{0.4\textwidth}
				\includegraphics[width=\textwidth]{img/chapters/chapter_5_hydrodynamics/HDF_Simulation_03_01_03_mod-1.png}
				\caption{3:1}
    \end{subfigure}
	\begin{subfigure}[b]{0.4\textwidth}
				\includegraphics[width=\textwidth]{img/chapters/chapter_5_hydrodynamics/300_100_300-2.pdf}
				\caption{3:1}
    \end{subfigure}
        %%%%%%%%%%%%%%%%
	\begin{subfigure}[b]{0.4\textwidth}
				\includegraphics[width=\textwidth]{img/chapters/chapter_5_hydrodynamics/HDF_Simulation_04_01_04_mod-1.png}
				\caption{4:1}
				\caption*{Simulation}
    \end{subfigure}                        
	\begin{subfigure}[b]{0.4\textwidth}
				\includegraphics[width=\textwidth]{img/chapters/chapter_5_hydrodynamics/400_100_400-2.pdf}
				\caption{4:1}
                \caption*{Experiment}
    \end{subfigure}         
	\caption[Hydrodynamic focusing comparison between simulations and experiments]{Comparison of the hydrodynamic focusing between the simulation and the experiment for different inlet flow rate ratios. The inlet flow rate ratio between the sheath and the sample flow ranges from 1 to 4, whereas the centre sample flow is kept at 100 $\mu$L/min. The sample flow is dyed with black ink for visibility. The directions of the flow streams are indicated by the arrows.}
        \label{fig:hydrodynamicFocusingSimulationAndExperiments}
\end{figure}

It is found that higher flow rates and thus also higher Reynolds numbers, produce a more narrowly focussed sample stream. This effect is also visible in the numerical simulation. 

Table~\ref{tab:hydrodynamicFocusingFlowConditionSampleStreamWidth} lists the measured sample stream width determined by experiment result and the corresponding flow conditions. The Reynolds number ranges within two orders of magnitudes ($O(10^{-1})-O(10^{1})$). It is assumed that the range of values of Reynolds number are sufficiently low to provide laminar flow conditions. In general, all values of Reynolds number relevant to this thesis can be considered as sufficiently low to provide laminar flow.

\begin{table}[htb]
\begin{center}
\caption[Sample stream width and flow conditions used in the hydrodynamic focusing experiment]{Experimentally measured width of the hydrodynamically focused sample stream and the corresponding flow conditions, such as flow rates, Reynolds number and back pressure, calculated for the iBidi separation device geometry. The flow rate of the two sheath flows are kept the same. The Reynolds number and back pressure are calculated according to Equation~\ref{eqn:reynoldsNumber} and Equation~\ref{eqn:hagenPoiseuille}, respectively.}
\vspace{1ex}
\label{tab:hydrodynamicFocusingFlowConditionSampleStreamWidth}
\begin{tabular}{ccccc}\hline
\multicolumn{1}{l}{Sample flow} & \multicolumn{1}{l}{Sheath flow} &  \multicolumn{1}{l}{Reynolds} & \multicolumn{1}{l}{Back} & \multicolumn{1}{l}{Sample stream} \\ 
\multicolumn{1}{l}{rate} & \multicolumn{1}{l}{rate} &  \multicolumn{1}{l}{number} & \multicolumn{1}{l}{pressure} & \multicolumn{1}{l}{width} \\ 
$[\mu$L/min$]$ 		& $[\mu$L/min$]$  	&	$[-]$	& $[$Pa$]$  & $[$mm$]$\\
\hline
10 		& 10 			& 0.29 & 0.96 & 1.20 \\
10 		& 20 		& 0.49 & 1.59 & 0.72 \\
10 		& 30 		& 0.68 & 2.23 & 0.54 \\
10 		& 40 		& 0.88 & 2.87 & 0.44 \\
20 		& 20 		& 0.59 & 1.91 & 1.04 \\
20 		& 40 		& 0.98 & 3.19 & 0.65\\
20 		& 60 		& 1.37 & 4.47 & 0.48 \\
20 		& 80			& 1.76 & 5.74 & 0.38 \\
50 		& 50 		& 1.47 & 4.78 & 1.02 \\
50 		& 100 		& 2.44 & 7.97 & 0.61 \\
50 		& 150 		& 3.42 & 11.16 & 0.44 \\
50 		& 200 		& 4.40 & 14.35 & 0.36 \\
100 		& 100 		& 2.93 & 9.57 & 0.93 \\
100 		& 200 		& 4.88 & 15.94 & 0.61 \\
100 		& 300 		& 6.84 & 22.33 & 0.45 \\
100 		& 400 		& 8.79 & 18.70 & 0.36 \\
200 		& 200 		& 5.86 & 19.14 & 0.94 \\
200 		& 400 		& 9.77 & 31.89 & 0.62 \\
200 		& 600 		& 13.67 & 44.65 & 0.44 \\
200 		& 800 		& 17.58 & 57.41 & 0.40\\ 
\hline
\end{tabular}
\end{center}
\end{table}

The data in Table~\ref{tab:hydrodynamicFocusingFlowConditionSampleStreamWidth} is also visually shown in Figure~\ref{fig:hydrodynamicFocusingNormalizedWidth}. Figure~\ref{fig:hydrodynamicFocusingNormalizedWidth} shows the width ratio between the sample stream width and the width of the fluid channel for four inlet flow rate ratios at various sample flow rates. As predicted by the model described in Equation~\ref{eqn:hydrodynamicFocusingWidthRatio}, a non-linear relationship is observed where the focus width of the sample flow decreases rapidly with increasing flow ratio. The experimental results, as well as the simulation, compare well with the theoretical values.

\begin{figure}[htb]
\centering
\includegraphics[width=0.7\textwidth]{img/chapters/chapter_5_hydrodynamics/hydrodynamicFocusingIBidiDevice_errBar_withSimulation.eps}
\caption[Hydrodynamic focusing in iBidi $\mu$-slide: normalized width ratio for different inlet flow rate ratios at various centre inlet flow rates]{The normalized width ratio of the sample flow ($w_{c}/w_{0}$) plotted against multiple volumetric inlet flow rate ratios for different sample flow rates. The flow rates of the sample stream are $10$ $\mu$L/min, $20$ $\mu$L/min, $50$ $\mu$L/min, $100$ $\mu$L/min and $200$ $\mu$L/min and the sheath flow rates are set according to the volumetric inlet flow rate ratio $\dot{Q}_{c}/\dot{Q}_{s}$. The channel's height to width ratio is $h_0/w_0 = 0.13$.
The error bars show one standard deviation of the measured width of the sample stream for the different sample flow rates.}
\label{fig:hydrodynamicFocusingNormalizedWidth}
\end{figure} 

From the experimental results presented in Table~\ref{tab:hydrodynamicFocusingFlowConditionSampleStreamWidth} and Figure~\ref{fig:hydrodynamicFocusingNormalizedWidth}, as well as Equation~\ref{eqn:hydrodynamicFocusingWidthRatio}, it can be seen that the width of the focused stream mainly depends on the inlet flow rate ratio but not the overall volumetric flow rate. 

\subsection{Entrance length}\label{subsec:entranceLength}
Where a fluid enters a channel, or where several flow streams are brought together at the entrance region of a microfluidic device, as it is the case in hydrodynamic focusing, the following question arises: At which distance $L_{H}$ down-stream from the entrance point is the laminar flow a fully developed steady-state Poiseuille flow profile as described in Section~\ref{sec:hydrodynamicFlowInAMicrofluidicSseparationDevice}. The entrance length $L_{H}$ is calculated according to Equation~\ref{eqn:entranceLength} (see Chapter~\ref{ch:chapter2_theory}). In the microfluidic iBidi separation device, where the hydrodynamic diameter is calculated as $D_{H} \approx 0.7$ mm, the entrance length $L_{H}$ does not exceed $0.7$ mm for all the flow rates tabulated in Table~\ref{tab:hydrodynamicFocusingFlowConditionSampleStreamWidth}. For smaller flow rates this entrance length is also smaller. Typically used flow rates in this thesis are between $50$ $\mu$L/min and $400$ $\mu$L/min which results in an entrance length of $L_{H} \approx 21$ $\mu$m and $L_{H} \approx 165$ $\mu$m, respectively. Thus, the velocity profile is fully developed and remains steady shortly after the entrance of the fluid separation device, which has a total length of $24$ mm. Also particles in suspension can be assumed to travel at a constant velocity down-stream when being magnetically manipulated, as long as the magnet sits $0.7$ mm or more down-stream from the entrance. 

\section{Conclusion}\label{sec:conclusionChapter5Fluiddynamics}
The micromachined microfluidic iBidi separation device, designed and fabricated on a plastic susbstrate, which meets the requirements for hydrodynamic focusing as well as for particle separation, is validated based on its fluid dynamic properties.

The flow field inside the microfluidic device is modelled numerically. The simulated velocity profile inside the rectangular duct meets the theory accurately and is in line with the vast literature published on this topic. Due to the straight uniformity of the channel, analytical results would have been sufficient for the analysis presented in this chapter. The numerical model approach adopted, however, gives more flexibility for future analysis and can easily be extended to consider more complex channel designs.

Changing the flow rate alters the pressure drop along the channel, back pressure (i.e., how hard the pump must work) and the magnitude of the fluid velocity but not the path of the streamlines. This means a laminar regime can be assumed for all flow rates simulated and experimentally tested, which is also suggested by the Reynolds number calculated.

Hydrodynamic focusing is verified with the use of microscopic visualization of water sheath flow and dye containing sample flow. The width of the focused stream is measured for different volumetric inflow ratios and magnitudes. Experimental data matched the proposed model and were also in-line with the numerical results. 

The results presented here can be compared to a similar study presented by Lee \etal{}~\cite{Lee2006}. They successfully demonstrated the hyrodynamic focusing effect within a micro flow cytometer device and compared the normalized width of the hydrodynamically focused stream to the same model used in this thesis. They reduced the focused stream width within a $500$ $\mu$m wide fluid device to approximately $20$ $\mu$m. The channel used in their work is $6$ times smaller than the $3$ mm wide channel used in this thesis, but the aspect ratio as well as the Reynolds number is comparable. Thus, the normalized width of the hydrodynamically focused stream presented in their work for the different volumetric inflow ratios compares well with the model and the normalized focused stream width presented here.

The symmetric hydrodynamic focusing works well in a wide channel with the sheath flows in the same horizontal plane. In a single plane, however, hydrodynamic focusing is intrinsically problematic due to the lack of vertical focusing. The non-uniform (parabolic) velocity distribution of vertically spread beads is known to cause complications in particle separation~\cite{Wolff2003}; because beads near the top or the bottom channel wall will experience almost zero flow rate and thus will be effectively stuck in the system. This problem could be solved by using an axis-symmetric focusing device, which consists of two concentric capillaries~\cite{Takeuchi2005,Paie2014,Harold2014}. However, in practice, such structures are a challenge to fabricate using photolithography based microfabrication techniques. As a result, the mass production of such devices might not be cost effective. Another approach is to use Dean vortices to focus the beads orthogonal to the sheath flow~\cite{Mao2007,Daniele2015}. This requires the centre inlet to be curved and also requires high velocities (Re $>> 1$) to have a noticeable focusing effect. For these reasons an axis-symmetric hydrodynamic focusing will not be explored further in this work. Not focussing the beads in both directions orthogonal to the central flow will have a predicable impact on the efficiency of the microfluidic separation device. 
