\chapter{Summary and Outlook}\label{ch:summaryAndOutlook}
\section{Summary and Contributions}\label{s:SummaryAndContributions}

This work simulated the trajectories of superparamagnetic microparticle in an external magnetic field. The magnetophoretic motion of the particles has been simulated by developing a 3D Lagrangian tracking model. The particle trajectory simulations were used to analyse the separation efficiency of two novel magnet configurations under continuous flow and pulse flow mode. The simulations were also experimentally tested, which yielded benchmarks for validating the magnetophoretic particle transport model on a practical microfluidic platform.

%More particularly the emphasis, leading to this choice of system, was on reducing the effect of unbalanced forces arising in more common planar geometries. The first step was to conduct simulations and numerical studies to verify that beads could be decentralised uniformly from a central focus point, whilst in a continuous flow. It was then necessary to find flow regimes and geometrical parameters that could best optimise the separation efficiency.

The model has shown that the double magnet configuration, used under continuous flow conditions, could achieve a relative separation efficiency of $100\%$ at a flow rate of $10$ $\mu$l/min or smaller. If the flow rate is increased, the relative separation efficiency was found to decrease. Also a negative correlation could be seen between the relative separation efficiency and the particle throughput when the flow rate was increased. Only $85\%$ of all the introduced particles could be recovered in the simulations, even at the highest flow rate explored ($\dot{Q}=375$ $\mu$l/min). 

The double magnetic configuration has also been experimentally evaluated under the same conditions. The experimentally determined separation efficiencies could not be matched to the simulation results, where different scenarios were tested. This indicated that other forces, e.g. particle-particle interaction, need to be taken into account in further analysis. 

The particle separation efficiency and throughput could be improved using a quadrupole magnet configuration operated in pulse mode. An absolute separation efficiency of at least $80\%$ could be achieved in the experiments and simulations. The pulse mode also has the advantage to be able to recover all of the introduced particles, which improves the throughput. This is especially important for cell separation where the targeted cells only appear in small quantities. 

The driving force for miniaturization is an increasing demand for low-cost instruments capable of rapidly analysing compounds in very small sample volumes with a high level of automation. The magnet configurations investigated have the unique characteristics to be able to align magnetic particles along a customized path depending on the magnets' shape. This allows for a simple geometry of the microfluidic device with no expensive integrated magnetic components because the particles will be guided by an external magnetic field. Thus, the microfluidic device, due to its small size and simple geometry, can be cheaply manufactured; whereas the magnetic setup, which is required to generate a large enough magnetic field, can be kept externally and separately. As the cost per microfluidic device is lowered, it becomes economically feasible to use such devices as disposable items.

The increasing interest in magnetic microparticles and their application in bioanalysis fuels the development of Lab-on-a-Chip (LOC) or Micro Total Analysis System ($\mu$TAS) employing them as active components. However, to have accurate prediction on magnetic particle separation, their magnetic properties need to be known. Therefore, the magnetic response of commercially available superparamagnetic particles was measured. More particularly, the magnetophoretic mobility and the corresponding susceptibility of these particles was determined. The particles showed a nonlinear behaviour between the applied magnetic field ($30-65$ mT) and their magnetization. This nonlinearity might need to be taken into account in HGMS devices, because these separation devices often produce magnetic fields sufficiently strong ($\geq 0.5$ T) to saturate the magnetic particles. 

The devices proposed here could be a valuable alternative for magnetic cell separation by magnetically guiding particles along a customized path, with high potential for further development in a chain of downstream analysis of the sorted particles.  

%The purified sample could then be used for further development in a chain of downstream analys 


%We determined the magnetic field in the device by numerical simulations, which were consistent with experimetnal measures of the magnetic force exerted on calibrated magnetic beads % Derec

%Here we extended the concept of particle mobiliy to magnetic separations and demonstrated its usefulness in designing a novel type of magnetic separator based on a continuous flow through process. % Zborowski
%%%%%%%%%%%%%%%%%%%%%%%%%%
%%%%%%%%%%%%%%%%%%%%%%%%%%
%This thesis work looked into the theoretical and numerical analysis of magnetic particle separation.

%An analytical model has been developed for the Lagrangian tracking of magnetophoretic motion of magnetic microspheres in a fully developed flow inside a microchannel under the influence of a magnetic field.  

%Magnetic separation permits the isolation of target cells from heterogeneous, crude samples [231]. Microfluidics has increased in popularity within scientific research over the past decade.  The versatility they offer in design allows for an array of applications. Laminar flow that is achievable at the micro-scale enables highly predictive characteristics and for continuous separation to be achieved. The ability to incorporate automation and analysis on-chip is an attractive characteristic. Microfluidic separations are still troubled by trapped air-bubbles and blockages. Such phenomena are mitigatable, however are still an occurrence that can disrupt separation. The low flow limits observed in this experiment will restrict the amount of sample that can be processed. Running multiple MFDs, alongside each other will increase the concentration and separation of larger volumes, and could also facilitate multi-analyte detection, which is preferable when monitoring water for contaminants.% Naomi

%Laminar flow that is achievable at the micro-scale enables highly predictive characteristics and for continuous separation to be achieved. The ability to incorporate automation and analysis on-chip is an attractive characteristic. Microfluidic separations are still troubled by trapped air-bubbles and blockages. Such phenomena are mitigatable, however are still an occurrence that can disrupt separation. The low flow limits observed in this experiment will RESTRICT the amount of sample that can be processed. Running multiple MFDs, alongside each other will increase the concentration and separation of larger volumes, and could also facilitate multi-analyte detection, which is preferable when monitoring water for contaminants. % naomi

%Devices for biosensing often contain high-surface-area elements for efficient reaction kinetics. Within microfluidic applications it is a challenge to find an integrated fluid actuation technique that facilitates pumping with reasonable flow rates, while having a robust control on local fluid replenishment and mixing near no-slip surfaces. 

%It was found that an easy to manufacture system could be fabricated to accurately, hydrodynamically focus a core stream in three-dimensions.

%More particularly the emphasis, leading to this choice of system, was on reducing the effect of unbalanced forces arising in more common planar geometries. The first step was to conduct simulations and numerical studies to verify that beads could be decentralised uniformly from a central focus point, whilst in a continuous flow. It was then necessary to find flow regimes and geometrical parameters that could best optimise the separation efficiency.

%This study concluded that cylindrical magnets with higher separation ratios were necessary to obtain a more uniform magnetic field around the circumference of channels. This was seen as an important factor because it led to a more uniform radial transport of beads in all directions around the axis of the channel. This in turn allowed the magnetically induced transport of beads to be predicted more accurately.

%Numerical analysis showed how hydrodynamic focussing could be used to control two critical cases of bead mobilities which would result in successful separation. Calculations helped identify the flow regimes leading the widest range between these two critical mobility cases.

%Experiments also showed the effects that higher concentration of beads focussed to relatively small radii had on their agglomeration. The direct effect of which was to increase the overall mobility of beads. It was found that a precise control over the viscosity of the buffer and the suspension solution and of the throughput of the flows could shift the range of bead mobilities. Centring this range around the mean value of mobility acquired by agglomeration of beads could guarantee $100\%$ separation.

%Finally it is worth noting that an accurate model was created in MATLAB which encompasses as many deciding factors for the separation of particles using cylindrical channels and quadrupole configurations of magnets. It is recommended to and openly shared with anyone who is willing to carry on future work on this particular system.

%This project has demonstrated the successful isolation of magnetic beads in hydrostatic conditions using a novel dipole configuration of rectangular permanent magnets. In particular, the focus was on controlling the region in which the beads were isolated. The first step was to conduct numerical studies to verify that there is a region where isolation was feasible, and also to predict how the geometric parameters (width and height of the magnets, and the separation between them) affected the separating power and the location in which the beads were isolated. The studies concluded that flat magnets with low aspect ratios were preferable for the task because of a stronger magnetophoretic force and better compactness. % ciaran

%This project successfully demonstrated the isolation of magnetic beads in continuous flow conditions using a novel quadrupole configuration of magnets. It was found that an easy to manufacture system could be fabricated to accurately, hydrodynamically focus a core stream in three-dimensions. % Harold

%More particularly the emphasis, leading to this choice of system, was on reducing the effect of unbalanced forces arising in more common planar geometries. The first step was to conduct simulations and numerical studies to verify that beads could be decentralised uniformly from a central focus point, whilst in a continuous flow. It was then necessary to find flow regimes and geometrical parameters that could best optimise the separation efficiency. % Harold

%Our measurements in open fluid volumes revealed a large spread in single particle velocities. Recently published research on this matter supports the indications we found for the polydisperse nature of superparamagnetic particles, and shows that particles can differ a lot in susceptibility [79] and even show a remanent magnetic moment [40]. The particle dynamics and interaction effects we have analyzed will be affected by a spread in particle properties.

%This work has highlighted the various characteristics of the developed MPC. The MPC has a variety of interesting properties such as superparamagnetic characteristics, bio-compatibility, and high mechanical stability. It can be remotely heated by applied magnetic fields. A key advantage of the composite is the homogeneous dispersion of the nanoparticles with low particle agglomerations, which enables the fabrication of microstructures with small feature sizes. In the following possible future applications with the composite are presented % marcel suter

\section{Future Work}\label{sec:futureWork}

The key components of magnetophoretic systems are the magnetic particles, the magnetic field and the microfluidic channel. Therefore, these are the three things which need further investigation.

\subsection{Magnetic microparticles}
Magnetic microparticles, which present the central element of the magnetic separation, should be investigated more thoroughly. Only when their physical structure is fully understood, one can make accurate predictions about their magnetic behaviour. Especially the iron content and its variation among different particles is of great interest in magnetic particle separation. Scanning electron microscope (SEM) or transmission electron microscope (TEM) can give further insights about the particle structure. However, in order to gain full control, it would be best to synthesise the particles \textit{in house} or obtain them through a collaboration with an external group, like it is done in~\cite{Zborowski2002}. There is a lot of ongoing research in material science on how to fabricate superparamagnetic microparticles. Various people have successfully synthesised such particles in the past~\cite{Ma2003,Liu2003,Liu2004}.

As seen in this work, the susceptibility of magnetic particles does not depend linearly on the applied magnetic field. The susceptibility might be assumed as constant, since the magnetic field variation across the region of interest in the microfluidic systems is small, but the susceptibility value should be adjusted according to the magnetic field strength. The magnetic field adjusted value should then be used in further simulations.

\subsection{Magnetic field}
This work has shown that it is possible to guide magnetic particles along a well-defined trajectory by using a double magnet or quadrupole configuration. However, vertical forces negatively effected the particle separation efficiency. It could already be shown that by tapering the magnets the gathering points can be adjusted to specific needs or microfluidic channel geometries.

This work has already shown that by tapering the magnets the gathering points can be adjusted to specific needs or microfluidic channel geometries. Thus, having the possibility to accurately shape the magnet one could reduce vertical forces by shaping the magnet edges hyperbolically, as shown in Figure~\ref{fig:isodynamicMagneticField}. 

\begin{figure}[htb]
\centering
\includegraphics[width=0.75\textwidth]{img/chapters/summary/isodynamicField.png}
\caption[Isodynamic magnetic field]{Hyperbolically shaped magnet edges to generate an isodynamic magnetic field.}
\label{fig:isodynamicMagneticField}
\end{figure}

This generates an isodynamic magnetic field, which produces nearly constant force vector on linearly polarizable particles~\cite{Zborowski2011,Hatch2001}. Thus, the magnetic particle velocity remains constant and does not depend on the particle's position. This allows for an easier control of the particle trajectory and makes magnetic particle testing more reliable~\cite{Moore2000}.

Unfortunately, a perfectly isodynamic magnetic field, where magnetic force and its direction is constant, can only be achieved along the $x$ axis ($y=0$), which was proven in~\cite{Gogosov1983,Smolkin2006}. However, any reduction of the vertical force should potentially improve the separation efficiency and throughput and should particularly help in continuous separation.

A large magnetic field gradient is also desirable whilst keeping the magnetic field strength small. This guarantees that the particles do not saturate but still experience a large force. In the literature Halbach configurations have been used to generate effective HGMS devices~\cite{Hatch2001}. The Halbach configuration arranges permanent magnets in such a way that the magnetic field gets augmented on one side while getting cancelled on the other side. 

Combining the two things, shaping and arranging magnets, has not been explored to the fullest yet and further investigation is necessary.

\subsection{Microfluidic device}
The particles' trajectory is influenced by their initial release positions. The variation in position results in different magnetic drift forces and interaction time with the magnet and is mainly responsible for particles getting lost in the system. A more accurate localized particle injection could increase their separation efficiency especially in continuous particle separation, where the interaction time with the magnet is key. 

Particles could be accurately injected into the system through a nozzle. Figure~\ref{fig:nozzleInjection} shows the \textit{sweet} spot where particles should be continuously injected in order to avoid particle-wall interaction. However, in order to achieve a high separation efficiency, position, velocity and magnetophoretic mobility of the particles need to be in balance.

\begin{figure}[htb]
\centering
\includegraphics[width=0.75\textwidth]{img/chapters/summary/nozzleInjectionPosition.png}
\caption[Particle injection through a nozzle]{Preferred initial release position of the particles to achieve a high continuous separation efficiency in a double magnet configuration. The colour code differs between continuously separated and non-separated particles whereas the open and filled circles represent starting and end position, respectively. This figure shows the continuous separation at $\dot{Q}=10$ $\mu$l/min. Only upper half of the fluidic channel is shown due to symmetry.}
\label{fig:nozzleInjection}
\end{figure}
