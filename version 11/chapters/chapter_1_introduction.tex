\chapter{Introduction}\label{ch:introduction}

\section{Microfluidics and micro total analysis systems}\label{sec:microfluidicsAndMicroTotalAnalysisSystems}
This work is aimed to develop a 3D model to predict the trajectories of individual magnetic beads in a magnetic field. The model is used to understand the transport of the beads in a microchannel during magnetic separation. Simulations and experiments are combined to perform a comprehensive study of the effects of multiple parameters on the efficiency of magnetic particle separation. 

The separation and sorting of biological material from complex mixtures represents a critical task in a variety of biomedical applications including diagnostics, therapeutics, and fundamental cell biology. As biological samples are often heterogeneous, containing  different populations of cells, vesicles, proteins, etc., techniques are required to isolate specific components.

One possible approach to sort heterogeneous mixtures is with the help of microfluidics. Microfluidics involves the techniques to actuate, control and manipulate small volumes of fluids ($10^{-9}$ to $10^{-18}$ litres) in channels, where the channel dimensions are of microscopic scale (typically in the order of tens to a couple of hundred of micrometres)~\cite{Whitesides2006}. Figure~\ref{fig:microfluidicDeviceExample} shows a schematic and a photograph of a microfluidic device example from the literature~\cite{Geczy2019}.

\begin{figure}[htb]
\centering
\includegraphics[width=0.9\textwidth]{img/chapters/chapter_1_introduction/microfluidicchannelSchematicAndPhotograph.eps}
\caption[Example of a microfluidic device]{Schematic and photograph of an example for a microfluidic device. The device features channels with a width of $100$ $\mu$m and a height of $50$ $\mu$m. Figure has been adapted from~\cite{Geczy2019}.}
\label{fig:microfluidicDeviceExample}
\end{figure}

The science of miniaturization was initially fuelled by the microelectronics industry during the development of miniature silicon-based electronic devices. Techniques for silicon microfabrication and miniaturization were then extended to the fabrication of mechanical devices that became known as microelectromechanical systems (MEMS)~\cite{Petersen1982}. A later trend in MEMS technology was the development of devices for applications in the medical and life science areas. The term biological microelectromechanical systems (BioMEMS) was coined to describe devices and systems with a biolocial application, although subsequently they did not all necessarily have the components normally found in traditional MEMS devices. Hence, a broad definition of BioMEMS would include some devices and applications made using the modern implementation of microfluidics, which was developed by Manz \etal{} in 1990~\cite{Manz1990}.

With the reduced size came the concept of a micro total analysis system ($\mu$TAS) or lab-on-a-chip (LOC). The two terms essentially mean the same and thus, are used interchangeably in this work. Both can be regarded conceptually as a biological equivalent of conventional silicon integrated circuits, which involve miniaturization and integration of electronics. Instead of an electronic circuit, LOC and $\mu$TAS have microchannels for guidance and manipulation of fluids. As the name suggests, an ideal LOC or $\mu$TAS integrates multiple laboratory functions on a single microfluidic device. This includes sample preparation, incubation, detection and analysis procedures on a single device, no bigger than the size of a microscope slide~\cite{Demello2006,Yager2006}. Consequently, labour intensive and expensive benchtop processes can be replaced and eliminated. LOC can be thought of as the shrinking of an entire laboratory into a miniature device. The LOC concept is schematically illustrated in Figure~\ref{fig:labOnAChipConcept}.

\begin{figure}[htb]
        \centering
        \begin{subfigure}[b]{0.42\textwidth}
        		 \includegraphics[width=\textwidth]{img/chapters/chapter_1_introduction/labOnAChipConcept.png}
                \caption{}
                \label{fig:labOnAChipConceptIdea}
        \end{subfigure}
        \hfill
        \begin{subfigure}[b]{0.57\textwidth}
                \includegraphics[width=\textwidth]{img/chapters/chapter_1_introduction/LabOnAChip.eps}
                \caption{}
                \label{fig:labOnAChipConceptExample}
        \end{subfigure}
        \caption[Concept of a lab-on-a-chip]{(a) Schematic illustration of a lab-on-a-chip, where laboratory processes are integrated onto a microfluidic device. Figure has been reproduced from~\cite{Chow2002}. (b) Conceptual example of a lab-on-a-chip device where various processes, e.g. filtering, mixing, sorting and the analysis, are all integrated on a single device. Photographs adapted from~\cite{Englert2009} and~\cite{Lillehoj2010}.}
        \label{fig:labOnAChipConcept}
\end{figure}

%\begin{figure}[htb]
%\centering
%\includegraphics[width=0.6\textwidth]{img/chapters/chapter_1_introduction/labOnAChipConcept.png}
%\caption[Concept of a lab-on-a-chip]{Schematic illustration of a lab-on-a-chip, where laboratory processes are integrated onto a microfluidic device. Figure has been reproduced from~\cite{Chow2002}.}
%\label{fig:labOnAChipConcept}
%\end{figure}

With the development of LOC and their integrated micro-scale fluidic systems, it is now possible to manufacture inexpensive, fast and high-throughput biomedical devices which need little sample volume and reagents. These benefits are not readily available at low-cost for many benchtop techniques~\cite{Kumar2010,Hosokawa2013}. Therefore, LOC has shown considerable potential for a wide variety of bioanalysis applications including: organic synthesis~\cite{Haswell2001,Watts2005}, forensic analysis~\cite{Verpoorte2002,Horsman2007}, DNA extraction~\cite{Oakley2009,Shaw2009} and analysis~\cite{Tegenfeldt2004}, polymerase chain reaction (PCR) amplification~\cite{Tegenfeldt2004}, chemical and biological assays~\cite{Ng2010}, environmental monitoring~\cite{Gardeniers2004,Marle2005} and various separation and detection techniques in analytical chemistry~\cite{Greenwood2002,Rios2006}.

%The major advantages of these microfluidc devices and systems are a significant reduction of the chemical reaction time and a lower consumption of expensive chemical reagents and samples, as well as the capability of automating the entire process. 

%Microfluidic systems known as lab-on-a-chip and the micro–total analysis system have recently attracted a great deal of attention as immunosensor platforms. Microfluidic immunosensors provide several advantages over conventional immunoassay methods, including an increased surface-to-volume ratio for efficient mass transport in the immunoreactions, leading to fast analysis; a miniaturized microchannel dimension to reduce the consumption of samples and reagents; and automated integration with other functions, such as valves, pumps, mixers, and detectors, to achieve a point-of-care (POC) goal (4, 5).

%With these numerous advantages it is clear why microfluidic devices have received such interest in recent times, having great potential for revolutionising chemical and biological processes.

%Therefore, there is a demand to develop an automated and miniaturized platform for immunoassays. Such a platform must be capable of simplifying procedures, reducing the assay time and sample/reagent consumption and enhancing the reaction efficiency. The advantages of the microfluidic systems described previously fulfill these important criteria for immunoassays. Therefore, extensive investigations using microfluidics for performing immunoassays have been reported recently.

\subsection{Microfluidics bio-sensors for point-of-care diagnostics}
\label{subsec:microfluidicsForBiomarkerAnalysis}
A biomarker is a measurable characteristic that reflects the severity or presence of a disease state. In this thesis, the term \textit{biomarkers} refers to molecules in body fluids, typically proteins in blood~\cite{Strimbu2010}. 

%There is also a raising interest in searching for new biomarkers for various disease states. Among them, cancers are the most intensively studies targets, not only because they are major cause of mortality in developed countries, but also because early-stage diagnosis of a cancer usually decreases the likelihood of mortality.

Biomarkers are conventionally detected using immunoassays. Clinical immunoassays are traditionally based on 96-well microtiter plates and are widely used for the quantification of target molecules in many applications such as clinical diagnostics, pharmaceutical research, and basic biological investigations~\cite{Sauer2005,Spisak2007,Delamarche2005}. Immunoassays, such as enzyme linked immunosorbent assays (ELISA) for the detection of proteins~\cite{Yalow1996} or quantitative polymerase chain reaction (qPCR) for the detection of nucleid acids~\cite{Chen2005}, have been established as \textit{gold standard} detection technologies for specific biomarkers. However, most conventional immunoassays involve time consuming and labour intensive protocols, because they include a series of washing, mixing and incubation steps~\cite{Sittampalam2004}. It often takes several hours, and in some cases days, for a single assay to be processed~\cite{Plitnick2013}. Much of the time taken in immunoassays is due to the long incubation time, because analyte molecules must rely on diffusion to encounter antibodies on the surface where the conjugation occurs~\cite{Rossier2001}. Moreover, these protocols require highly trained personnel, bulky and expensive instruments, and expensive reagents~\cite{Gossett2010}. This ultimately makes this approach costly and due to its size not well suited for point-of-care (POC) analysis.

However, there are POC diagnostic tests commercially available and to provide context relevant to the thesis a few breakthrough examples will be discussed. A current major class of such tests is the Lateral Flow Immunoassay test (LFT), which uses a membrane or paper strip to indicate the presence of biomarkers such as pathogen antigens or antibodies~\cite{Ching2015}. On a membrane, addition of a sample induces capillary action which lets the sample migrate along the membrane. As the sample flows across the membrane, it gathers detection reagents embedded in the membrane, and then flows over a band that contains capture molecules (antibodies). The captured antibodies and detection reagent labelled analytes form a visible band which can be interpreted by eye. This diagnosis is most notably used to test for pregnancy, infections with streptococcus, the flu or HIV~\cite{Ching2015}. LFTs are simple to perform and are extremely versatile, consequently, they have found clinical, veterinary, agricultural, food industry and environmental applications. The single-flow action, however, does not mimic the multi-step procedures of laboratory-based assays that are crucial for highly reproducible, quantitative, and sensitive results~\cite{Sia2008}.

The other major class of successful POC tests is the blood glucose test, which almost serves as a textbook example for a high-impact POC product that has improved diabetic patients' lives, and now acts as a pillar of the entire diagnostics industry. The glucose test is also performed on membranes, but is typically classified differently compared to lateral flow immunoassays as the analytical method is altogether different. It uses signal amplification by a redox enzyme and is typically analysed via an electrochemical readout~\cite{Sia2008,Luppa2011}. However, testing for glucose is somewhat unique in a sense that the analyte concentration is in the range of millimole per litre, which is well above the concentration of most biomarkers. Additionally, the test typically needs to be taken multiple times a day, which outnumbers that of most other biological tests.

With the advance in microfluidics technology, the miniaturized fluidic systems (LOC or $\mu$TAS) became an attractive alternative for running multiplexed on-chip immunoassays due to the integration of various compatible components on a single device, such as pumps, mixers or valves. Researchers have successfully developed micropumps~\cite{Yobas2008,Nisar2008,Wang2010,Das2017}, microvalves~\cite{Oh2006}, micromixers~\cite{Zhang2017,Nguyen2004} and other active as well as passive microfluidic liquid handling components~\cite{Geschke2004,Lin2010}. With the integration of these components on a single microfluidic bio-sensor, it is possible to perform full sample analysis including continuous sampling~\cite{Auroux2002}, sample separation and mixing~\cite{Manz1990}, and sample pre-treatment and pre-concentration~\cite{Stone2001}. Integrating and combining multiple microfluidic components offers enhanced analytical performance, high throughput, real-time detection, fast reaction rates and portability~\cite{Liu2010,Wu2012}; making detection adaptable to POC applications.

For instance, Lee \etal{}~\cite{Lee2007} successfully developed an automated microfluidic biosensor for the detection of breast cancer. The device uses multiple channels to detect the interaction between anti-rabbit immunoglobulin G and rabbit immunoglobulin G using a micro-arrayed immunoassay. The experimentally achieved sensitivity was found to be $10^{-4}$ mg/mL ($0.67$ nM). The total time needed for the on-chip analysis was approximately $8$ hours. 

Kong \etal{}~\cite{Kong2009} demonstrated a LOC design for the rapid detection of clenbuterol which has been used to treat chronic breathing disorders but also for performance-enhancing in bodybuilding. They performed an ELISA in the interrogation section of the microchannel network. Pneumatic valves were utilized to control the introductions of different reagents. Under optimal conditions, $5$ ng/mL could be detected in less than $30$ minutes.

Another example for the integration of a microfluidic device and biosensor is demonstrated by Yao \etal{}~\cite{Yao2016}. They reported a fully automated microfluidic-based detection system for the rapid determination of insulin concentration based on a chemiluminescence immunoassay. The system is a double-layered PDMS device embedded with interconnecting micropumps, microvalves, and a micromixer. The chemiluminescence assay can detect insulin in the range of $1.5$ pM to $391$ pM in less than $10$ minutes. These results were achieved with the help of magnetic microparticles, which will be explained in the next section.


%%%%%%%%%%%%%%%%%%%%%%%%%%%%%%%
%One problem microfluidic systems have is their power supply. Even if the power consumption might be small, incorporating actuators for liquids on a micron-sized device is challenging. Many strategies for on-chip liquid transport in microfluidic immunoassays have been reported in the literature, such as electroosmotic~\cite{Hayes1993,Gao2005}, electrowetting~\cite{Mugele2005,Sista2008,Sista2008a}, pneumatic~\cite{Wang2005,Wang2006,Yang2008,Yang2009}, centrifugal~\cite{Madou2001,Madou2006,Honda2005,He2009,Lee2009}, power-free~\cite{Hosokawa2006}, piezoelectric~\cite{Kim2009}, and thermopneumatic~\cite{Ha2009} approaches.

%New concepts and technologies, such as integrated optical~\cite{Pereira2011} and electrical~\cite{Choi2002} detection or centrifugal force technologies~\cite{•} have been continuously applied in on-chip immunoassay applications as opposed to traditional immunoassays.

% Naomi thesis
%ELISA: The reported detection limits are not sensitive enough (106 CFU/mL) and there is a lengthy procedure between sample preparation to final analysis, this technique will not be suitable for the biosensor proposed here [131]

% Naomi thesis
%Polymerase chain reaction (PCR) is an in vitro technique used to enzymatically amplify specific fragments through a series of repetition reaction cycles. Ultra-fast PCR methods can detect as low as 5 CFU in a single assay with a 20 minute detection time. Contrary to the short detection time, the process is lengthy as several pre-treatment steps are required in order to establish suitably conditioned samples suitable for the extraction of target DNA [15]. Ultra-fast PCR has been reported to be able to detect 1 CFU/mL of V. cholerae in seawater samples; the enrichment step took four hours [38]. The technique also offers possibility for automation [132], but requires highly trained users as the enrichment step is complex.

%%%%%%%%%%%%%%%%%%%%%%%%%%%%%%%%%%%%%%%%%%%%
%%%%%%%%%%%%%%%%%%%%%%%%%%%%%%%%%%%%%%%%%%%%
%%%%%%%%%%%%%%%%%%%%%%%%%%%%%%%%%%%%%%%%%%%%
%%%%%%%%%%%%%%%%%%%%%%%%%%%%%%%%%%%%%%%%%%%%

%\section{Microfluidic device detection methods}
%Different method of detection, as opposed to traditional imumnoassays, have been developed. For example, a biosensor using diffusion between two laminar streams has been developed. The concentration of the bound antibody/antigen can then be rapidly quantified in the stream using an inverted microscope and fluorescence source. The use of DNA directed self-assembly on the surface of microfluidics chips to immobilize the capture probes allowing a multiplexed immunological detection assay has also been demonstrated. 

%\subsubsection{Optical immunosensors}\label{subsubsec:opticalImmunosensors}
%Optical detection is the simplest and most popular method used in microfluidic immunoassay applications because it can be easily implemented in microfluidic systems~\cite{Kuswandi2007}. Fluorescence measurement is the most widely used optical detection method for molecular sensing in microfluidic systems because of its many advantages over other techniques, including high sensitivity and selectivity and well-established labelling procedures~\cite{Auroux2002,Hata2003}. In fluorescence detection, suitable labels conjugated with the antibody or antigen are excited by a laser or light-emitting diode light source, and fluorescence light emitted from the labelled molecules is detected by the photodetector.

%Organic fluorescent molecules traditionally used for fluorescence detection can be employed for the simultaneous screening of multiple analytes. Although organic fluorescent dyes are reasonably effective in multiplexed analyses, they have several drawbacks, including poor color, susceptibility to photobleaching, and a broad excitation/emission spectrum. To overcome these problems, researchers have replaced traditional fluorophores with semiconductor nanoparticles known as quantum dots (QDs). QDs provide better stability and higher quantum yield and exhibit size dependent fluorescence emission spectra that are easily tuned by synthesis techniques and allow multiplexed analyses across a wide range of accessible wavelengths. However, their size (typically $10-50$ nm), which is larger than that of biomolecules, might be a steric hindrance in the immunoreaction, thereby leading to sensitivity degradation. QDs are also relatively difficult to synthesize and functionalize and are consequently much more expensive than fluorophores~\cite{Han2013}.

%%%%%%%%%%%%%%%%%%%%%%
% http://www.sciencedirect.com/science/article/pii/S0925400517306330#fig0015
% In recent years, with the advance in microfluidics field, new concepts and technologies, such as integrated electrical [1] and optical detection [2], and lab-on-a-CD [3] technologies, have been continuously applied in on-chip immunoassay applications. Yao [4] reported a detection system based on a chemiluminescence immunoassay. This microfluidic system is a double-layered polydimethylsiloxane (PDMS) device embedded with interconnecting micropumps, microvalves, and a micromixer. At a high injection rate of the developing solution, the chemiluminescence signal can be excited and measured within a short period of time. Yang \etal{} [5] proposed a novel sensor for chloramphenicol (CAP) detection based on immunoassay and magnetic nanoparticles. First, CAP was conjugated on functional Au nanoparticles that were labeled Raman reporter molecules. Then the Au nanoparticle and free CAP have a competitive immune reaction for CAP antibody-modified magnetic nanoparticles. The Raman spectra were then obtained with a microscopic Raman spectrometer. Kong \etal{} [6] recently demonstrated a lab-on-a-chip design for the rapid detection of clenbuterol. Enzyme-linked immunosorbent assay (ELISA) was performed in the interrogation section of the microchannel network and pneumatic valves were utilized to control the introductions of different reagents. Soares \etal{} [7] reported a portable microfluidic immunoassay system for detection of Ochratoxin A (OTA). It has an integrated peristaltic pump, Si photodiode and customized PCB in a dark case. This detection system was connected to a laptop computer for signal processing and display. This microfluidic immunoassay system shows the potential to achieve a true point-of-use quantification device compared to conventional technology. Ellinas \etal{} [8] proposed a microfluidic platform for the detection of two proteins (IL-6 and PDGF-2). Their microfluidic system used super hydrophobic, passive valves to simplify its working principle. Obtaining almost same fluorescence intensity for detection, the usage of reagents for this microfluidic platform was at $5$ $\mu$l in comparison with typically $50$ $\mu$l used in conventional methods, while the time consumption was reduced by $3$ times. Though immunoassays based on microfluidic technologies can produce results comparable to their conventional counterparts, the operation of reagents and integration of detection are still difficult to achieve. Shao \etal{} reported a PDMS micro-chip based immunoassay with multiple reaction zones [9]. It used 10 $\mu$l sample and reported a detection limit of 5 ng/mL (33 pm). Similar to Kong's work, they also used external supply of compressed air to pump the biological samples and reagents, and to actuate the pneumatic valves. The need of peripheral devices for pumping and actuation increased the sizes of these reported systems, makes them not very portable. On the other hand, though many designs of micro-pumps have been reported in the field over the last decades, reliable micropumps are still not readily available for uses in microfluidic systems.
%%%%%%%%%%%%%%%%%%%%%%%%%%%%%%%%%%%%%%%%%%%%
%%%%%%%%%%%%%%%%%%%%%%%%%%%%%%%%%%%%%%%%%%%%
%%%%%%%%%%%%%%%%%%%%%%%%%%%%%%%%%%%%%%%%%%%%
%%%%%%%%%%%%%%%%%%%%%%%%%%%%%%%%%%%%%%%%%%%%

\section{Magnetic micro- and nanoparticles}\label{sec:magneticMicroparticleAndNanoparticle}
Micron-sized magnetic beads are intensely investigated due to their physiological properties and diagnostic potential, since they are very promising carriers for biomarkers in biomedicine or bioanalysis, e.g. for drugs~\cite{Safarik2002,Schuster2000}, DNA~\cite{Vuosku2004,Yeung2006}, antibodies~\cite{Guesdon1977} or cells~\cite{McCloskey2003,Zborowski2011}, and due to their size can be seamlessly integrated in microfluidic devices.

Magnetic beads usually consist of an ensemble of nano-sized ferromagnetic grains in the order of $5-15$ nm, embedded in a nonmagnetic host matrix. The nonmagnetic matrix seals the magnetic grains and prevents them from further chemical reactions and degradation. The inner morphology of the particles, which is determined by the size and the number of enclosed magnetic grains, influences the overall magnetic behaviour~\cite{Osawa1988,Leslie-Pelecky1996}. 

The ferromagnetic grains are generally an iron oxide, such as magnetite ($\textrm{Fe}_{3}\textrm{O}_{4}$) or maghemite ($\gamma-\textrm{Fe}_{2}\textrm{O}_{3}$), because of a high saturation magnetization and non-toxicity~\cite{Haefeli1999,Mueller2004}. Other magnetic materials such as cobalt, nickel and neodymium-iron-boron may offer improved magnetic properties, however, these materials may be toxic and/or susceptible to oxidation~\cite{Vadala2005}.

The nonmagnetic host matrix is often a polymer or silica. These materials allow the production of a wide range of different particle sizes, ranging from as small as $6$ nm to more than $20$ $\mu$m and a uniform size distribution can be achieved~\cite{Park2005,Horak2007,Bourlinos2001}. These polymer and silica hosts also permit a wide choice of possible surface coatings, which ensure that the magnetic particles are compatible with biological materials~\cite{Yeung2006,Jaffrezic-Renault2007,Tartaj2003}.

The loading of iron oxide within the beads occurs either as fixed dispersion of single domain nano-sized grains within the host matrix or concentrated in the core of the bead, as shown in Figure~\ref{fig:superparamagneticParticleIronOxideTypes}. If the magnetic grains are small enough and spaced sufficiently far apart inside the host matrix, then the bead will behave superparamagnetically and will have a large induced dipole moment due to the collective response of the magnetic grains. Due to their superparamagnetic behaviour the beads are attracted to magnetic fields but do not retain any magnetism after removal of the magnetic field, which will be explained in more detail in Section~\ref{sec:magnetism}. This property has revolutionized particle based separation because they can be easily manipulated through the control of an external magnetic field~\cite{Ganguly2010}.

\begin{figure}[htb]
        \centering
        \begin{subfigure}[b]{0.48\textwidth}
        		 \includegraphics[width=\textwidth]{img/chapters/chapter_1_introduction/RS_Beauty_1.jpg}
                \caption{}
                \label{fig:superparamagneticParticleIronOxideCore}
        \end{subfigure}
        \hfill
        \begin{subfigure}[b]{0.48\textwidth}
                \includegraphics[width=\textwidth]{img/chapters/chapter_1_introduction/RS_Beauty_2.jpg}
                \caption{}
                \label{fig:superparamagneticParticleIronOxideDisperse}
        \end{subfigure}
        \caption[Two composition structure of superparamagnetic beads]{Two types of superparamagnetic beads. The iron oxide can either be found as (a) nano-sized magnetic grains dispersed in a core with a surrounding shell containing no magnetic material, or (b) a fixed dispersion of single domain nano-sized magnetic grains within the nonmagnetic host matrix.}
        \label{fig:superparamagneticParticleIronOxideTypes}
\end{figure}

Further, magnetic beads have become popular mobile carriers due to their large specific surface area (total surface area per unit of mass). Beads can be coated with antibodies to bind to biomolecules and thereby providing a controllable means of identifying a specific target species~\cite{Gijs2004}. These antibodies are shown as surface extensions in the schematic of Figure~\ref{fig:magneticParticleFunctionalized}. Hence, the beads act as mobile reaction substrates, able to directly interact with biological entities over a wide size range, e.g. cells ($1-100$ $\mu$m), viruses ($20-450$ nm), proteins ($5-50$ nm) or genes ($2$ nm wide and $10-100$ nm long)~\cite{Pankhurst2003,Campbell2006,Collier2011}. Several excellent reviews on the synthesis and the use of magnetic particles in medical and biological applications exist in the literature~\cite{Osaka2006,Lu2007}.

\begin{figure}[htb]
\centering
\includegraphics[width=0.55\textwidth]{img/chapters/chapter_1_introduction/immunoLabelledParticle.pdf}
\caption[Magnetic bead labelling via antibodies]{Schematic of an immunomagnetically labelled magnetic beads. Antibodies (red Y-shapes) are used to target specific biological cells or material (green cubes) and bind them to the surface of the bead (blue sphere). Non-targeted molecules (yellow dodecahedrons) do not interact with the antibodies and thus are not bound to the magnetic beads.}
\label{fig:magneticParticleFunctionalized}
\end{figure}

%%%%%%%%%%%%%%%%%%%%%%%%%%%%%%%%%%%%%%%%%%%%%%%%%%%%%%%%%%%%%%%%%%%%%%%%%%%%%%%%%%%%%%%%%%%%
%%%%%%%%%%%%%%%%%%%%%%%%%%%%%%%%%%%%%%%%%%%%%%%%%%%%%%%%%%%%%%%%%%%%%%%%%%%%%%%%%%%%%%%%%%%%
%% Continue here with particle characteristics
%%%%%%%%%%%%%%%%%%%%%%%%%%%%%%%%%%%%%%%%%%%%%%%%%%%%%%%%%%%%%%%%%%%%%%%%%%%%%%%%%%%%%%%%%%%%%
%Due to the advantages of magnetic microparticles as well as their popularity as mobile carriers in biomedical applications, several groups have characterised commercially available magnetic beads. Table~\ref{tab:liteartureTestedMagneticBeads} lists the properties of the most popular magnetic beads used in the literature. 
%
%\begin{table}[htb]
%\begin{center}
%\caption[Magnetic beads used in the literature]{List of the five most popular magnetic beads used in the literature and their physical characteristics.}
%\vspace{1ex}
%\label{tab:liteartureTestedMagneticBeads}
%\begin{tabular}{llccl}
%\hline
%Bead Type 	& 	Supplier 		& Size 			& Susceptibility 	& Reference\\ 
%					& 						& [$\mu$m] 	&  						& \\
%\hline
%Dynabeads MyOne 	& $1.05 \pm 0.05$ 	& Thermo Fisher Scientific & $7-10\times 10^{9}$ \\
%Dynabeads M280 		& $2.8 \pm 0.08$ 		& Thermo Fisher Scientific & $6-7\times10^{8}$ \\
%Dynabeads M270 		& $2.8 \pm 0.08$ 		& Thermo Fisher Scientific & $6-7\times10^{8}$ \\  
%ScreenMAG-Silanol  	& $1 \pm 0.15$				& Chemicell GmbH 	& $9\times10^{10}$ \\
%SiMAG-Silanol 			& $1 \pm 0.15$				& Chemicell GmbH  	& $9\times10^{10}$ \\
%\hline
%\end{tabular}
%\end{center}
%\end{table}

\section{Cell separation in microfluidic systems for biomedical applications}
Cell separation underpins many discoveries in biology and is further enabling research in areas as diverse as regenerative medicine~\cite{Guo2006,Tomlinson2013}, cancer therapy~\cite{Takaishi2009} and HIV pathogenesis~\cite{Terry2009}. Thus the ability to isolate and enrich a population of cells extracted from their complex native environment is often the first processing step in the analysis~\cite{Tomlinson2013}.

Numerous cell isolation and sorting techniques have been developed for clinical settings that are based on the differences in cell size, shape, density or the cells' dielectric and magnetic properties~\cite{Gossett2010,Tsutsui2009,Lenshof2010}. 

Density-based separation techniques such as differential centrifugation techniques are commonly used in laboratories and are also routinely used clinically~\cite{Miller1969}. Differential centrifugal separation relies upon sedimentation that is accelerated by centrifugal force and requires a difference in density between the constituent phases. Cells of different densities and volume in a suspension will settle at different rates, with the larger and denser cells pelleting more rapidly~\cite{Berg1993}. Cell separation based on differential centrifugation allows for the isolation of cells from a large initial cell sample, which makes this technique particularly applicable for separations involving the use of blood. Indeed, the most commonly used clinical cell separation method is apheresis of whole blood to isolate mononuclear cells for treatment of a variety of conditions, including leukaemia~\cite{Buckner1969}. Furthermore, centrifugation is also widely employed for the isolation and purification of viruses, subcellular organelles, proteins, and nucleic acids that are dissolved or dispersed in biologically relevant solvents~\cite{Duve1975}. In situations where cell size or density differences are not significant, however, effective cell separation of individual cell types is not straight forward using centrifugation. Problems can be overcome, however, by performing repeated centrifugations using differing concentrations of centrifugation medium and differing angular velocities. By using this more elaborate approach, it is possible to isolate different cell types from a complex mixture, such as disrupted solid tissues as demonstrated for mouse liver~\cite{Liu2011}. However, although technically feasible, this is still challenging to perform with high specificity. As such, centrifugation methods are generally used if specificity is not absolutely necessary, as in apheresis, or as a pre-enrichment stage to remove cells like red blood cells and platelets.

% Effective cell sorting to support numerous biomedical pursuits relies upon optimal matching between the target cell attributes, desired outcomes, and the parameters of the sorting technique.
Advances in bioanalysis, along with the promise of POC and personalized medicine, have increased the need to separate specific biological entities from their native environment in order to prepare concentrated samples for subsequent analysis and testing in both basic research and clinical applications. Thus, the need to separate cells and other biological material is rapidly expanding toward the isolation of rare target populations (cell abundance is less than $10^{3}$ in $1$ mL~\cite{Dharmasiri2010}), examples include the enrichment of circulating tumour cells (CTC), hematopoietic stem cells, and circulating fetal cells (CFC) from blood~\cite{Armstrong2011,Chen2014}.

The low abundance of rare cell populations preclude the use of conventional benchtop separation techniques, such as centrifugation, because of their low selectivity and the impact of sample loss. In order to improve selectivity, cells can be labelled with tags using antibody binding and isolated by other means. 

The fluorescent-activated cell sorting (FACS) technique harnesses the ability to label specific cells with fluorescent dye tags, which allows for cell sorting based on the individual labelling profile of a particular cell population~\cite{Bonner1972}. The cell suspension is entrained in the centre of a narrow, rapidly flowing stream of liquid. The flow is arranged so that all the cells are in a line with a large separation (relative to their diameter) between each cell. A vibrating mechanism causes the stream of cells to break into individual aerosol droplets. Just before the stream breaks into droplets, the cells in the flow encounter a focused laser beam that scatters light into a detector where the fluorescent signal of each cell of interest is measured. Depending on the scattering and fluorescent profile, each cell encapsulated in an aerosol droplet is then given an electric charge and then physically sorted using electrostatic deflection plates~\cite{Bonner1972}. Current state-of-the-art FACS devices typically use up to seven lasers which enables them to sort six different types of cell based on morphological and fluorescent cell signatures (e.g., cell surface labels, cell size, or granularity)~\cite{Piyasena2014,Gossett2010}.

Although still one of the most highly-used cell isolation platforms, a serious limitation to FACS systems remains their price and complexity. The complexity of operation also requires dedicated highly-trained personnel to ensure reliable cell sorting efficiency and purities. It is reported in the literature that FACS systems are susceptible to cross-contamination, clogging in the nozzle, require high reagent consumption and have safety concerns due to the sorting of aerosol samples~\cite{Holmes2014}. Additionally, even if modern flow cytometers offer sample rates of upwards of $50,000$ cells/s and multichannel detector analysis, it is not a feasible technique for processing clinical samples ($>500$ million cells). These limitations must be overcome to enable more efficient clinical application and further commercialization.

Consequently, the next generation of cell sorting devices should ideally meet higher standards of performance, versatility and convenience. They should include: faster sorting rates, improved sorting accuracies, improved sorting purity, high cell recovery, ability to process native biological fluids, ability to process diverse cell types, enhanced capabilities for multiplexed sorting, simpler operating procedures enabling fully automated systems, reduced hazardous risk by eliminating aerosols, reduced cost and reduced size for operational convenience and portability.
% purity
%Assessment of the efficacy of any separation technique involves three paramount considerations: purity, recovery and when separating cells also viability~\cite{Sharpe1988}. 
% Recovery is usually used to describe the percentage of cells that are obtained post sorting compared to the number of total cells or target cells in the original suspension. As such, the recovery, in conjunction with the purity, gives information on the efficiency of the cell separation.

With the rapid development in lab-on-a-chip technology, recent advances in microfluidics have been providing robust solutions for rare cell analysis. Microfluidic devices capable of cell separation and sorting are emerging at an increased pace~\cite{El-Ali2006}. The ability to design advanced microfluidic networks in combination with both actuation and sensing/monitoring is driving novel concepts for advanced on-chip cell studies, as well as new means to capture/retain, separate and/or concentrate cell species from complex biological fluids. The advance in microfabrication technology has made it possible to design microfluidic devices with design features that are commensurate with single cells~\cite{Sackmann2014}.

Current developments in cell separation and cell enrichment in microfluidic devices encompass the employment of active separation methods, where cells are manipulated by an external field, and passive separation methods which induce a force onto the cells by controlled hydrodynamics. Most cell separation methods rely on the use of active manipulation techniques, such as optical~\cite{Ashkin1987,Grier2003}, electrical~\cite{Huang2001}, acoustical~\cite{Neild2007}, or magnetic fields~\cite{Helseth2003,Yellen2007,Erb2008}. There is also research in passive separation methods, where cells are separated solely based on hydrodynamics effects~\cite{Sajeesh2014}. This enables the sample preparation step to be avoided because no specific labelling is needed (\textit{label-free separation}). The various methods are summarised in Table~\ref{tab:variousSeparationMethods} and each method with its advantages and disadvantages is briefly reviewed and discussed below.

\begin{table}[htb]
\begin{center}
\caption[Separation method summary]{Summary of the different separation methods.}
\vspace{1ex}
\label{tab:variousSeparationMethods}
\begin{tabular}{clll}
\hline
& Separation method & Advantages & Disadvantages \\
\hline
\multirow{6}{*}{\rotatebox[origin=c]{90}{Passive}} 
& Filtration
&  \begin{tabular}[c]{rl} - & Label-free\\ - & No need for external \\ & forces \end{tabular}	
&  \begin{tabular}[c]{rl} - & Risk of blockages \\ - & Complex fabrication \end{tabular} \\ \cline{2-4}
& Hydrodynamic 	
&  \begin{tabular}[c]{rl} - & Label-free\\ - & No need for external \\ &  forces \end{tabular}	
&  \begin{tabular}[c]{rl} - & Complex chip design \\ - & Flow needs to be  \\ & carefully controlled \end{tabular}	\\ \cline{2-4}
& Centrifugation
&  \begin{tabular}[c]{rl} - & Label-free\\ - & Large sample volumes \end{tabular}	
&  \begin{tabular}[c]{rl} - & Lacking specificity \end{tabular}	\\ 
\hline
\multirow{10}{*}{\rotatebox[origin=c]{90}{Active}} 
& Optical 		
&  \begin{tabular}[c]{rl} - & Label-free\\ - & Contactless \\ - & Precise control over \\ & individual particles \end{tabular}	
&  \begin{tabular}[c]{rl} - & External power supply \\ & and laser needed \end{tabular} \\ \cline{2-4}
& Electrical
&  \begin{tabular}[c]{rl} - & Label-free\\ - & Electric force can \\ & be adjusted \end{tabular}	
&  \begin{tabular}[c]{rl} - & Buffer solution needed \\ - & External power supply \\ & needed \\ - & Joule heating \end{tabular} \\ \cline{2-4}
& Acoustic
&  \begin{tabular}[c]{rl} - & Label-free\\ - & Contactless \end{tabular}	
&  \begin{tabular}[c]{rl} - & External power supply \\ & needed \end{tabular} \\ \cline{2-4}
& Magnetic
&  \begin{tabular}[c]{rl} - & Contactless \\ - & Inexpensive magnetic \\ &  setup \end{tabular}	
&  \begin{tabular}[c]{rl} - & Requires magnetic\\ & labelling \end{tabular} \\ \cline{2-4}
\hline
\end{tabular}
\end{center}
\end{table}

% https://www.ncbi.nlm.nih.gov/pmc/articles/PMC4310825/
% Conventional cell separations are often achieved on the basis of the differences in cell physical properties, such as density and size, or by exploiting more specific biochemical properties, such as surface antigen expression (Radisic et al., 2006, Pratt et al., 2011, Bhagat et al., 2010, Recktenwald and Radbruch, 1997). Rather than providing a comprehensive review of all cell separation techniques, representative examples of different methods are briefly described here to illustrate each isolation technique. These techniques are divided into three categories: culture-based cell separation, separation based on physical properties and biochemical affinity-based cell separation. The last class of cell separation technologies includes the important techniques of fluorescent-activated cell sorting (FACS) and magnet-activated cell sorting (MACS). All of these techniques are summarized in Table 1, which provides a general overview of the specific cell characteristics that are used to achieve isolation, in addition to describing the advantages and disadvantages inherent in each technique. There is no perfect cell isolation technique, and development of such a platform would be a quixotic approach, thus the choice of separation platform is dependent on application and need.

\subsection{Passive particle separation}
% https://www.ncbi.nlm.nih.gov/pmc/articles/PMC4331226/pdf/nihms654772.pdf
Passive separation systems consist of a variety of methods, such as inertial forces, mechanical filtration, and immobilization, that do not solely rely on labels, particles or biochemical markers~\cite{Tsutsui2009}. Instead, these methods rely on the inherent physical differences between particle groups (e.g., size, shape, compressibility, and density). Since these techniques often do not require the addition of an attached particle or ligand (such as a fluorescent or magnetic particle tag), they are attractive methods when affinity ligands are not available~\cite{Sollier2009}. 

%One of the earliest automated cell sorter did not require any labels, as it sorted cells by volume using an impedance technique based on the Coulter principle. The technology was principally developed to count blood cells quickly by measuring the changes in electrical conductance as cells suspended in a conductive fluid passed through a small orifice~\cite{Graham2003}. These key concepts have since been expanded to sort cells in the microfluidic regime. 

In the following, some of the most important passive particle separation techniques for microfluidic devices are described.

\subsubsection{Filtration separation}
Filtration is based on particle size differences and selective segregation by dimensioned pores. This method makes it easy to extract the liquid phase of a suspension and is commonly used for blood applications with varying degrees of success~\cite{Crowley2005,Ji2008,Chen2008}.

A major limitation of filters is that they often clog when trapping larger cells or debris. This means that filter systems are only efficient for a certain time defined by saturation (until clogging occurs). They are best suited for highly diluted samples and for a given volume at low flow rates. For this reason, several designs of size-based microfilters exist including membrane filters, pillar filters, weir filters and crossflow filters~\cite{Ji2008}. The separation performance of different filter types have been experimentally validated and reviewed by Ji \etal{}~\cite{Ji2008}.

One of the most typical separation strategies in microfluidic filters uses micro-fabricated pillars which are equally spaced, or a membrane containing an array of micro-pores on the floor or the ceiling of the channel, which acts as a sieve, to remove or enrich cells from a blood sample~\cite{Zheng2011,Lim2012,Lv2013,Fan2015}.

The simplest filter exploiting microchannels are weir filters that contain a dam or barrier to retain particles. Sato \etal{} fabricated such a microfluidic weir filter which halts the movement of particles through the chip without stopping the fluid flow, after which assays or reactions can be performed on the particle surfaces~\cite{Sato2000}. Particles can also be trapped between two weirs, whereupon solutions can be drawn over them, allowing solid phase extraction and chromatography techniques to be employed~\cite{Oleschuk2000}. 

Crossflow filters, also known as tangential-flow filters, are derived from the concept of pillar filters except that the pillars are placed tangential to the main flow direction. This method of filtration has a major advancement over early types of filters such as weir, pillar, and membrane because they are less prone to clogging because they behave more like a sieve than a conventional filter, where the fluid is directly propelled into the filter~\cite{Murthy2006}. Several cross-flow filter designs have been developed for size-based sorting applications such as the separation of white blood cells from whole blood, plasma from whole blood, and myocytes from non-myocytes~\cite{Sethu2006,VanDelinder2007}.

More recently, sophisticated grids with tiered (i.e., decreasing) spacing as a function of distance or oscillating flow patterns have been used for sorting cells in a non-binary fashion and to reduce the risk of clogging. Mohamed \etal{} developed a size exclusion filter to isolate circulating fetal cells from maternal blood by arranging multiple pillar filters in series with different spacings~\cite{Mohamed2007}. Later, McFaul \etal{} as well as Preira \etal{} showed how structural ratchets can be used to sort cells based on their size in combination with oscillatory applied pressures~\cite{McFaul2012,Preira2013}. Lately, Yoon \etal{} demonstrated a $\mu$-sieve which has a low oscillating frequency added to the fluid flow. This made it possible to release unwanted polystyrene particles trapped between the larger target polystyrene particles. With this structure, a high separation efficiency as well as a high recovery rate could be achieved~\cite{Yoon2016}.

\subsubsection{Hydrodynamic separation}
Hydrodynamic cell separation techniques make use of the fluid flow and the laminar flow profile alone to separate particles based on their size. Therefore, parameters such as flow rate control through one or more inlets, the channel geometry, and the configuration of outlets, dictates the flow character and ultimately the cell isolation performance.

Blom \etal{} developed an on-chip hydrodynamic chromatography device in which particles were introduced into a channel and separated according to size~\cite{Blom2002,Blom2003}. The separation principle is based on the parabolic profile caused by pressure-driven flow, which allows smaller particles to migrate closer to the walls of the channel such that they have a slower average velocity than larger particles that are farther from the wall, thus achieving separation.  

Yamada \etal{} used pinched-flow fractionation to sort different particle sizes~\cite{Yamada2004}. Pinched flow fractionation occurs when a flow stream of cells is \textit{pinched} by a narrow channel cross-section such that cells are constrained and aligned against a side wall. Smaller particles migrate closer to the side wall compared to larger particles and are subsequently separated by size once the channel broadens since they follow different stream lines. The technique was refined by asymmetric amplification of the flow in the system~\cite{Takagi2005}, the incorporation of PDMS valves~\cite{Sai2006}, and by replacing pressure-driven flow with electroosmotic flow~\cite{Kawamata2008}.

Separation methods, where hydrodynamic properties are combined with physical obstacles, as used in filter separation techniques, are known as \textit{deterministic lateral displacement} (DLD). In DLD, cells navigate through an array of posts for sorting by size. In these systems, control over sorting is given by the design of the array features such that cells smaller than a critical radius move with the convective flow and cells larger than a critical radius move in a direction dictated by the arrays~\cite{Beech2012}. Thus, small particles follow the laminar flow streamlines through the chamber, essentially \textit{weaving} around the posts whereas larger particles, above a critical size, are unable to follow the streamlines upon encountering a post, and are instead \textit{bumped} in the direction perpendicular to flow. Using a periodic array of microposts, in which each row of posts is offset by a certain periodic distance, Huang \etal{} showed that smaller cells could more easily traverse between obstacles than larger cells, enabling their separation~\cite{Huang2004}. The critical sizes of cells and microposts were carefully evaluated by Inglis \etal{}~\cite{Inglis2006} and have been used to isolate large cells, e.g. cancer cells or leukocytes, from whole blood samples~\cite{Davis2006,Huang2008,Liu2013}.

\subsubsection{Microfluidic centrifugation separation}
Centrifugation separation, as it is currently used in bulk cell separation techniques, can also be integrated in microfluidic systems. Such centrifugal microfluidic devices are commonly known as lab-on-a-disk (LOD)~\cite{Lai2004,Park2012,Kim2014}. Centrifugation approaches for cell separation in LOD include methods that utilize the physical centrifugation process on microfluidic platforms and the centrifugation effects due to Dean flow~\cite{Dean1959}.

Physical LODs are circular-shaped platforms containing a network of microchannels, chambers and other features. As the platform resembles a compact disk (CD), it is referred to as the centrifugal microfluidic CD or lab-on-a-CD~\cite{Gorkin2010}. A motor is used to rotate the microfluidic CD, transporting fluid radially outwards through the microfluidic device, and manipulating fluid by means of various microfluidic functions and features on the disc. The spinning disc platform is particularly advantageous in biomedical applications where large samples based on their density are to be separated~\cite{Burger2012,Burger2012a}. Amasia \etal{} demonstrated the separation of plasma from a large sample of whole blood~\cite{Amasia2010}. Similarly, isolation of white blood cells from whole blood using density gradient techniques has also been shown~\cite{Kinahan2014,Nwankire2015}.

Physical centrifugation techniques enhance inertial forces. However, inertial forces can also result in curved fluid channels. Typically, inertial forces emanate from boundary effects of fluid flow adjacent to the walls of a microfluidic channel, causing lift. Thus, in curved microfluidic platforms particles can be focused by balancing the inertial lift force and Dean drag force~\cite{DiCarlo2009}. Di Carlo \etal{} demonstrated that cells could be differentially focused and sorted based on size under laminar flow using a serpentine pattern~\cite{DiCarlo2007}. A major benefit of this system is its high throughput (e.g., $1.5$ mL/min) without additional sheath flow or sequential cell manipulation, which is useful for processing native biological fluids~\cite{DiCarlo2008}.

As in the case of serpentine patterns, where cells can be focused into a single streamline, a spiral microfluidic channel can sort cells by size~\cite{Kuntaegowdanahalli2009}. Russom \etal{} created a system with a spiral channel that applied inertial forces to focus and sort cells~\cite{Russom2009}. Similarly, Hou \etal{} and Nivedita \etal{} utilized a spiral fluid channel to continuously isolate CTCs from blood samples and separate blood cells, respectively~\cite{Hou2013,Nivedita2013}. Guan \etal{} and Warkiani \etal{} demonstrated that a channel with a trapezoidal cross section could further enhance the performance of a spiral microfluidic device to focus and sort cells~\cite{Guan2013,Warkiani2014}.

Inertial forces do not only occur in curved channels, they also play a critical role in straight microfluidic channels~\cite{DiCarlo2009}. Parichehreh et al. were able to enrich nucleated cell populations in blood using inertial forces in straight microfluidic channels of large aspect ratios~\cite{Parichehreh2013}.

\subsection{Active particle separation}
Figure~\ref{fig:particleManipulationMethodRange} shows the different active manipulation methods and visually depicts the particle size regime to which the different methods have access to, as well as the force range each method can apply.

\begin{figure}[htb]
        \centering
		\includegraphics[width=\textwidth]{img/chapters/chapter_1_introduction/particleManipulationMethodRange.pdf}
        \caption[Particle manipulation methods and manipulation range]{Diagrams showing (a) the relative sizes of particles and the associated technologies that provide access to manipulation in these size regimes and (b) the relative forces to which different manipulation schemes have access. Figure reproduced from~\cite{Erb2009}.}
        \label{fig:particleManipulationMethodRange}
\end{figure}

%%%%%%%%%%%%%%%%%%%
% http://iopscience.iop.org/article/10.1088/1464-4258/9/8/S02/meta
%In single cell experiments it is now common to use microfluidic devices to achieve reliable control over the environment of the cell. This miniaturization is particularly important when rapid environmental changes are required. For this purpose, a number of miniaturized systems, often called lab-on-a-chip devices, have been developed [6, 7]. In miniaturized fluidic devices the flow becomes extremely laminar [8]. The flow is therefore non-turbulent and mixing between two adjacent fluids will mainly depend on diffusion. This opens up the possibility to create devices where different media can flow in adjacent streams without extended mixing [9]. By manipulating a cell within such a microfluidic device, it is possible to completely change the environment around the cell in a highly repeatable manner.

% http://dukespace.lib.duke.edu/dspace/bitstream/handle/10161/1193/D_Erb_Randall_a_200904.pdf?sequence=1
%The manipulation of particle suspensions is an essential capability for various engineering applications ranging from self-assembled nano-manufacturing to life science analysis tools. Methods for manipulating the particles in parallel have mainly relied on the use of optical~\cite{Ashkin1987,Grier2003}, electrical~\cite{Pethig1996,Huang2001}, acoustical~\cite{Groschl1998,Neild2007}, or magnetic field traps~\cite{7,8,9,10,11,12}, which have the advantage of being shaped remotely through the use of lasers, electrodes, sonic waves, or external coils, respectively.

\subsubsection{Optical manipulation}\label{subsubsec:opticalManipulation}
Optical manipulation techniques present an effective method for controlling the three degrees of spatial freedom for particles from tens of nanometers in size~\cite{Svoboda1994} to tens of microns~\cite{Leach2006,Greulich2016}. Optical forces produced by a highly focused optical beam impose scattering and gradient forces sufficiently large to manipulate micron-sized objects (e.g. cells)~\cite{Neuman2008}.

Thus, optical forces have been used for cell manipulation or to trap cells in a liquid due to a mismatch in the refractive index between the cell and its surrounding fluid. Optical manipulation is considered advantageous for biomaterial manipulation due to its preservation of cell functionalities, if the laser intensity is below the level that causes damage~\cite{Jonas2008}.

Optics has first being used to levitate and transport individual particles in a fluid by Ashkin in 1970 \cite{Ashkin1970}. Eventually, optical radiation forces were applied to cell sorting by propelling single cells through a microfluidic channel, whereby the long traversal times generally resulted in low throughputs~\cite{Buican1987}. However, recent innovations in miniaturized optical devices have reinvigorated the use of optics as switches, which have made optics-based methods more feasible for fluorescent label-based cell sorting~\cite{Wang2005a,Perroud2008,Wang2011}. Several groups have used focused optical beams to impose switchable radiation forces on a continuous stream of focused cells for on-chip sorting and cell characterization~\cite{Wang2005a,Jr2007}. Recently, Wu \etal{} developed a pulsed laser fluorescence-based cell sorter that rapidly forms a stationary bubble due to localized heating to controllably deflect target cells at $20,000$ cells/sec~\cite{Wu2012a}.

However, optical techniques have limiting characteristics making them unsuitable for many applications. First, optical manipulation can only exert forces in the order of $O(10^{1})$  piconewtons~\cite{Simmons1996}, which removes this technique from many force measurement platforms. Second, for many materials lasers often induce charging or heating which is generally detrimental to system performance~\cite{Peterman2003}. Third, sample throughput is limited due to the need to have a focused beam. Finally, optical techniques require the use of expensive, bulky equipment that is currently difficult to integrate into LOC technologies and mobile units. 

\subsubsection{Electrical manipulation}\label{subsubsec:electricalManipulation}
Electrical forces can either be exerted on a particle in a uniform electrical field due to the particle's surface charge (electrophoresis) or on a dielectric particle when it is subjected to a non-uniform field (dielectrophoresis).

Electrophoresis refers to the movement of suspended particles toward an oppositely charged electrode in direct current (DC) fashion. Since most biomaterials have a slight negative net surface charge due to chemical groups on their surface, they migrate toward the positive electrode during electrophoresis~\cite{Voldman2006}. Takahashi \etal{} applied electrophoresis to sort cells on a microchip in which an upstream fluorescence detector identified labelled cells for rapid electrostatic sorting downstream~\cite{Takahashi2004}. Yao \etal{} developed a similar device based on gravity that operated in an upright orientation to process cells without convective flow~\cite{Yao2004}. A more recent example by Guo \etal{} showed electrophoretic sorting with much higher throughput by sorting water-in-oil droplets under continuous flow~\cite{Guo2010}. In this system, pre-focused cells were encapsulated into droplets such that droplets containing single cells were sorted from droplets containing no cells or multiple cells.

In contrast to electrophoresis, where particles move in a uniform electric field due to their surface charge, dielectrophoresis (DEP) refers to the movement of particles in a non-uniform electric field due to their polarizability. This force does not require the particle to be charged, but relies on an induced dipole moment across a cell~\cite{Voldman2006}. Once exposed to an AC field, particles migrate either toward or away from the region of strongest field intensity depending on the electrical permittivity of the particle and the fluid. Particles with a higher permittivity than the fluid are attracted toward the field maxima, which is known as positive DEP (pDEP)~\cite{Lenshof2010}. The opposite is true for negative DEP (nDEP). The strength of the force depends on the medium and particles' electrical properties, on the particles' shape and size, as well as on the frequency of the electric field~\cite{Voldman2006}. 

Since cells exhibit a dielectrophoretic force in the presence of an electric field and because integrating electrical systems into a microfluidic device is straightforward (drawing from the highly developed science of chip technology for MEMS) DEP has become a popular choice for cell separation~\cite{Lackie2007}.

Huang \etal{} took advantage of the intrinsic dielectric properties of cells and successfully sorted five different cell types based on their individual distinct dielectric properties on a $5 \times 5$ microelectronic chip array~\cite{Huang2002}. Also, Cummings \etal{} designed a microchip containing arrays of insulating posts where the resulting non-uniform electric field across the posts exerted different magnitudes of force on cells as they passed~\cite{Cummings2003}. Later, this principle was used to generate non-uniform electric fields near insulating ridges on a microchip to filter and concentrate cells~\cite{Barrett2005}. Vahey \etal{} introduced an isodielectric separation system for sorting cells without labels~\cite{Vahey2008}. In this system, cells experience dielectrophoretic forces that displace them to their natural equilibrium point across a conductivity gradient to where their net polarization vanishes (i.e. isodielectric point). With this principle, cells with different conductivities are displaced along the gradient for efficient on-chip sorting. Wang \etal{} developed a system using lateral nDEP, whereby a set of interdigitated electrodes aligned along both sides of a microfluidic channel provides repulsive forces to organize cells at precise distances from the microfluidic channel walls, to enable enhanced on-chip cytometry and sorting across five channels~\cite{Wang2007}. Doh and Cho developed a microchip to separate viable from non-viable cells due to their dissimilar polarizabilities, resulting in pDEP and nDEP, respectively~\cite{Doh2005}.

An interesting extension of dielectric cell sorting uses field-flow fractionation (FFF), which exploits a combination of approaches (e.g., gravitational, centrifugal, thermal, magnetic, and electrical) to position cells at precise distances from the microchannel floor and thus to distinct regions of the parabolic flow profile for separation by flow~\cite{Roda2009}. Vykoukal \etal{} first developed a DEP FFF system to isolate stem cells from enzyme-digested adipose tissue~\cite{Vykoukal2008}. This system contained patterned microelectrodes along the bottom of a microfluidic channel that exerted dielectric forces in the opposite direction to gravity, which let the cell types levitate at different heights based on their weight.

These advances hold promise for creating the next generation of cell sorting devices that could also enhance clinical detection, analysis, and diagnosis using a single microchip. However, the electrical permitivities of solvents are not always tunable, such as in the case of cell media where specific salt concentrations and pH concentrations are required. In addition, electric platforms can have deleterious effects such as charging and heating.


%In contrast to directly sorting cells in a buffered suspension, several groups have developed systems to encapsulate single cells into emulsified droplets for sorting using DEP, thus enabling continuous genomic and proteomic analyses downstream.27–29 Unlike FACS, which can generate potentially biohazardous aerosols, water-in-oil droplets provide a safe and rapid way to analyze individual cells post-sorting. Baret \etal{} applied DEP in a fluorescence-activated droplet sorter to separate up to 2,000 cells/sec.27 Agresti \etal{} used emulsions to generate picoliter-volume reaction vessels for detecting new variants of molecular enzymes and dielectrophoretic sorting.28 Mazutis \etal{} showed that cells compartmentalized into emulsions with beads coated with capture antibodies can be used to analyse the secretion of antibodies from cells for downstream sorting using DEP (Fig. 1).29 These advances hold promise for creating the next generation of cell sorting devices that could also enable clinical detection, analysis, and diagnosis using a single microchip.

%Some groups have coupled hydrodynamic lift forces, which act on cells exposed to a shear flow, with DEP to sort cells by size or electric permeability\cite{Karimi2013}. Moon \etal{} and Shim \etal{} used hydrodynamic lift coupled with DEP to sort breast cancer cells and CTCs without labels, respectively 124, 125 Kim \etal{} sorted cells based on their different sizes during the cell cycle using a DEP fractionation technique.126 Hydrodynamic forces were used to concentrate cells to one side of the microfluidic channel whereupon cells encountered an AC field that displaced the cells across the channel by distances according to their size and dielectric properties. This technique was later used to separate viable from nonviable yeast cells.127 An interesting dual function microchip was developed by Sun \etal{} that could simultaneously sort cells using DEP and size cells using the Coulter Principle via an integrated transistor.128

\subsubsection{Acoustic manipulation}\label{subsubsec:acousticManipulation}
% https://www.ncbi.nlm.nih.gov/pmc/articles/PMC4331226/pdf/nihms654772.pdf
Acoustic standing waves guide particles to regions of higher pressure. This movement of an object in response to an acoustic pressure field is known as acoustophoresis. The method takes advantage of a density and stiffness difference between the particle and the surrounding fluid~\cite{Groeschl1998,Groeschl1998a}. Acoustical systems allow for the generation of multiple pressure nodes in a system, without the need for additional structures to create localized distortions to the field, as it is often required in common optical, electrical, and magnetic systems~\cite{Neild2007}. Localized pressure nodes can be created using ultrasound sources and signals. Acoustic forces are contact- and label-free and thus biocompatible because they do not affect cellular viability~\cite{Lenshof2010}.

Acoustic standing waves were first used to sort cells by Johansson \etal{}~\cite{Johansson2009}, in which fluorescently labelled cells detected by a camera triggered an ultrasound transducer that moved the cell from its initial streamline toward a pressure node, thereby modifying its trajectory to the target outlet. More recently, an acoustic wave transmitter has been integrated in a microfluidics system. Acoustic microfluidic technologies have provided new areas of development within analytical flow cytometry, including cell sorting~\cite{Piyasena2014,Ward2009} or particle trapping~\cite{Leibacher2015}.

However, pressure nodes tend to be broad, weak, and exhibit higher order effects. These characteristics allow control of large microparticles ($>10\mu$m), but have reduced effectiveness on smaller particles, which are susceptible to Brownian diffusion. Though this technique has developed into bulk fluid applications of spatial patterning of aggregates~\cite{Woodside1998} and sized-based separation schemes~\cite{Petersson2007}, acoustophoresis does not provide for precise, deterministic control of many useful microparticle sizes. 

\subsubsection{Magnetic manipulation}\label{subsubsec:magneticManipulation}
Magnetic separation (immunomagnetic separation) uses either a permanent magnet or electromagnets to apply an external magnetic field to create a force on targeted particles. It is normally facilitated by binding magnetic particles to specific biomarkers. Once the magnetic particle is attached, the biological material inherits the magnetic properties of the magnetic particle. Magnetic forces can also be exerted on magnetically responsive cells or biomaterial suspended in a ferrofluid. Hence, the magnetic biomaterial (labelled or intrinsic) can be effectively manipulated, sorted, captured or separated using an externally applied magnetic field. 

Immunomagnetic separation actively applies a strong magnetic field, which can exert forces into the tens of nanonewtons and over a wide range of different particle sizes (see Figure~\ref{fig:particleManipulationMethodRange})~\cite{Adair1991,Grahl1996}. Most biological materials have a range of magnetic susceptibility and are either diamagnetic or weakly paramagnetic. This separation method has the potential to be selective and a high throughput can be achieved~\cite{Sole2001,Yung2009,Hoshino2011,Earhart2014}.

Additionally, magnetism has the practical advantage of being biologically and chemically inert. This is why magnetic separation has shown tremendous promise for biomedical and bioanalytical applications~\cite{Gijs2004,Pankhurst2003}. 

In the last decade, magnetic separation techniques have been conducted in microfluidic geometries, which combine the advantages of microfluidics and magnetic microparticles. The combination of the two fields, microfluidics and magnetism, is commonly known as magnetophoresis and will be discussed in more detail in the next section (Section~\ref{sec:magnetophoresis}).

% https://www.ncbi.nlm.nih.gov/pmc/articles/PMC4310825/
% Magnetic forces are unique in that they allow action at a distance, providing the ability to control objects without external wires or contacts. While not the only force that acts at a distance, i.e. gravity, electric forces, optical forces, and acoustics, the magnetic force underlying magnetic cell separation provides for action at a distance based on cell-marker affinity. While the intertwined fields of magnetism and magnetic materials are immense and very old, they have expanded to include biomedicine only rather recently (Krishnan, 2010, Murthy, 2007, Frimpong and Hilt, 2010, Mout et al., 2012, Pankhurst et al., 2009, Roca et al., 2009). The phenomenon of magnetophoresis is the controlled migration of particles, in this case biological cells, upon the application of an inhomogeneous magnetic field. Magnetophoresis may be employed to separate out specific cells from a heterogeneous cell population, with high selectivity, high sensitivity, and good throughput. In this section a brief overview of phenomena and terminology of relevance to magnetophoretic cell separation is provided. This section describes the categories of magnetic materials, the governing forces responsible for the separation and isolation of a target cells population, and the materials and methods choices that impact the overall operation of the desired platforms. More detailed information magnetic force theory and magnetophoretic principles may be found in a selection of excellent textbooks (Aharoni, 1996, Coey, 2010).

% https://www.ncbi.nlm.nih.gov/pmc/articles/PMC4310825/
% Among all functional materials, magnetic materials are singular by virtue of their ability to transfer energy and force through air, vacuum or intervening materials without wires or contacts. This property bestows these materials with a key technological role to enable devices of all types. In particular the magnetic force is well suited to many non-invasive biomedical applications through the phenomenon of magnetophoresis, which is the basis of magnetic-activated cell sorting.

%While the majority of magnetic cell sorting devices require nanoparticles to capture and isolate cells (Fig. 6), erythrocytes can be sorted from blood in a similar fashion solely due to their natural iron content in methemoglobin.129, 130 This label-free method of cell sorting has been used for debulking peripheral blood by removing erythrocytes from leukocytes, platelets, lipids, and plasma. Isolating erythrocytes from blood using external magnetic fields was first demonstrated in 1975 by Melville \etal{} 131 and was later refined by Zborowski \etal{} using cell tracking velocimetry132 and by Han \etal{} embedding a ferromagnetic wire inside of a microdevice.133–135 This method of magnetic manipulation from embedded ferromagnetic microwires was more fully developed later by others.136–138 Furlani presented a method for separating red blood cells from white blood cells in plasma by embedding soft magnetic elements in an array along a microfluidic channel.139 Since these magnetic, label-free approaches are only suitable for blood cells with natural iron content, several groups have employed nontraditional means for continuously sorting other types of cells in magnetic fields. Roberts \etal{} created a device that could sort macrophages from monocytes due to their dissimilar internalization rate of iron nanoparticles.140 Since macrophages can endocytose more iron content per cell than monocytes, macrophages can undergo greater loadings of magnetic material and thus experience greater MAP forces in an external field. Some groups have sorted diamagnetic (i.e., unlabeled) cells by suspending them in ferrofluids, which can separate cells due to differences in their size, shape, or deformability.141 Ferrofluids are liquids that are strongly magnetizable (e.g., water  containing iron oxide nanoparticles) and can thus transport diamagnetic particles in a magnetic field. Zeng \etal{} developed a system where an upstream magnet focused cells suspended in ferrofluid into a single stream, and a downstream magnet displaced those cells according to their size.142 Similarly, Shen \etal{} used a paramagnetic solution (i.e., of paramagnetic ions) to exert repulsive forces to sort cells by size.143

% maybe talk about positive and negative sorting

%% Purity
%The different isolation methods are usually evaluated by criteria such as purity, recovery rate, and viability (Tomlinson et al. 2013). Purity represents the degree of contamination of the recovered targeted cells with unwanted background cells/particles. Recovery rate reflects the ratio of number of recovered targeted cells to the number of cells in the original sample, and viability is the number of recovered live and heathy cells, which is very important for subsequent cell analysis stages. Therefore, ideal cell isolation methods should have short processing time, with high recovery and purity rates that result in viable cells.

\section{Magnetophoresis}\label{sec:magnetophoresis}
Magnetophoresis is an active separation technique and is the main topic of this thesis. Analogous to the term \textit{electrophoresis}, magnetophoresis describes the behaviour of a magnetic particle (see Section~\ref{sec:magneticMicroparticleAndNanoparticle}) moving through a viscous medium under the influence of an external non-homogeneous magnetic field. 

The term free-flow magnetophoresis describes the continuous separation of magnetic particles using the means of magnetophoresis while being pumped through a microfluidic system as depicted in Figure~\ref{fig:freeFlowMagnetophoresis}~\cite{Zborowski2015}. The appealing feature of free-flow magnetophoresis is that under well-controlled conditions it provides a means for non-invasive and low-cost particle separation due to its potential to be integrated on a LOC. However, the full control of particle motion in direction and velocity remains complicated, since particle properties and their magnetic response vary quite significantly depending on the particle manufacturer and its manufacturing processes.

\begin{figure}[htb]
\centering
\includegraphics[width=0.9\textwidth]{img/chapters/chapter_2_theory/magnetophoresis.pdf}
\caption[Schematic of magnetic particle separation principle using free-flow magnetophoresis]{A schematic of the magnetic particle separation principle in continuous flow using magnetophoresis. Magnetic particles or magnetically labelled biomaterial are deflected from the direction of laminar flow by the use of an external magnetic field. Figure is not drawn to scale.}
\label{fig:freeFlowMagnetophoresis}
\end{figure}

Magnetic separation typically uses functionalized magnetic particles with antibody receptors that can recognize and associate with various biological materials by affinity binding. The use of functionalized magnetic microparticles attached to biomarkers lead to a sensitive and relatively inexpensive cell separation method~\cite{Gijs2004,Pankhurst2003} which is a vital component for any magnetic separation device.

Magnetophoresis has gained a lot of attention in recent years due to its potential in bioanalysis systems, e.g. detection of cancerous cells~\cite{Zborowski2011}. Numerous configurations have been demonstrated over the course of the last decade including: capturing, isolation and collection of target biomaterial from mixtures of substances~\cite{Gijs2004,Friedman2005,Pamme2006,Yu2014}. 

Magnetic separation devices use an arrangement of permanent magnets or electromagnets in proximity with the fluid handling parts; in its simplest form a test tube is brought into proximity of a permanent magnet. This basic configuration can be further developed into automated devices. In this thesis, magnetic separators will be divided into two categories, namely \textit{batch} (batch mode) and \textit{continuous} (continuous flow) magnetic separators.

\subsection{Batch mode magnetophoresis}
Magnetic cell separation is conventionally achieved using a test-tube arrangement where the magnetically labelled cells are first immobilized using a magnet and subsequently released (batch mode), as schematically shown in Figure~\ref{fig:magneticSeparationBatchMode}.

\begin{figure}[htb]
\centering
\includegraphics[width=0.8\textwidth]{img/chapters/chapter_1_introduction/magneticSeparationBatchMode.pdf}
\caption[Batch mode magnetic particle separation]{A typical reaction or separation using magnetic particles, where (1) the particles are introduced into the sample or reagent solution, (2) they bind to the target analyte/reagent, (3) the particles are collected using an external magnet, (4) the supernatant is removed, and (5) the particles are resuspended in fresh buffer solution. Steps 3-5 may be repeated several times to ensure unbound material is removed from the particle surfaces.}
\label{fig:magneticSeparationBatchMode}
\end{figure}

This type of cell separation was commercialised by Miltenyi Biotec in 1990 with the development of magnetic activated cell sorting (MACS)~\cite{Miltenyi1990}. In MACS, antibody labelled magnetic particles are introduced into a cell suspension, where they bind to the cell of interest. The suspension is then flushed through a high-gradient magnetic separation column. The column contains steel wool that is magnetised by the presence of an external permanent magnet, which traps the magnetically labelled cells on the wool whilst the unlabelled cells pass through the column, thereby allowing their separation. The magnets can then be removed, allowing the magnetically labelled cells to be eluted from the column (see Figure~\ref{fig:MACS}). The cell suspension either flows through the column due to gravitational forces or can be enhanced by using a syringe plunger to push the suspension through. Miltenyi Biotec reports the isolation of cells from up to $30$ mL of anticoagulated blood in less than $30$ minutes~\cite{MiltenyiBiotec2017}. MACS is a well-established platform in both basic research as well as clinical medicine, as a major attraction is its scalability and very low capital costs relative to FACS~\cite{Thiel1998}. Thus, the concept of MACS has been commercially developed by other companies~\cite{Thanh2012}. 

\begin{figure}[htb]
        \centering
		\includegraphics[width=0.45\textwidth]{img/chapters/chapter_1_introduction/MACS_cuvette.pdf}
		\caption[Magnetic activated cell sorting principle]{Magnetic separation principle of MACS using a magnetic separation column. (a) Unlabelled cells pass through the column, magnetically labelled cells are retained within the high gradient magnetic column. The flow-through can be collected as the unlabelled cell fraction. (b) After the magnetic field is removed, the magnetically labelled cells can be eluted from the column. The magnetically labelled cells can again be collected. Figure reproduced from~\cite{MiltenyiBiotec2017}.}
\label{fig:MACS}
\end{figure}

However, MACS suffers from lost particles in the system. Due to the remnant field in the magnetized columns it is hard to release all magnetic particles, which negatively impacts the recovery rate. In cases of rare cell populations the recovery rate should be as high as possible. Additionally, the MACS systems developed by Miltenyi Biotec still requires operator interaction and therefore is in a macroscopic scale range for ease of manipulation. The concept of MACS, however, can be scaled to micrometers and applied in LOCs or $\mu$TAS. 

Choi and Ahn~\cite{Choi2000,Choi2001}, two of the pioneers in the field of magnetic cell separation in microfluidic systems, investigated the design, fabrication and characterization of a bio-separator composed of micro-machined serpentine and spiral electromagnets in a fluid channel. They successfully immobilized antibody-coated magnetic microparticles ($0.8-1.3$ $\mu$m, Bangs Laboratories) from a bio-buffer solution in a $800$ $\mu$m wide and $250$ $\mu$m deep channel (see Figure~\ref{fig:Choi2000}). They showed that efficient separation can be achieved in a few minutes.

\begin{figure}[htb]
   \centering
   \includegraphics[width=0.8\textwidth]{img/chapters/chapter_1_introduction/ChoiAhn.pdf}
   \caption[Micro-machined bio-separator with an integrated electromagnet]{Conceptual illustration of bio-sampling procedure. (a) Injection of magnetic particles, (b) separation and holding of particles, (c) flowing samples, (d) immobilization of target antigen, (e) release of magnetic particles to biosensor of bioreactor. Figure reproduced from~\cite{Choi2000}.}
   \label{fig:Choi2000}
\end{figure}

Another example of integrated micro-MACS is presented by Deng \etal{} in which arrays of nickel posts are manufactured on the bottom of a $150$ $\mu$m wide and $50$ $\mu$m deep microchannel. The nickel posts are $7$ $\mu$m in height and a diameter of $15$ $\mu$m and were magnetized by an external magnetic field to rapidly trap magnetic particles (Figure~\ref{fig:Deng2002}). A high separation efficiency of more than $95\%$ of the $4.5$ $\mu$m sized magnetic particles (Dynabeads M-450) was achieved when the particle suspension was injected at a flow rate of $2$ $\mu$L/min ($\bar{u}\approx4.4$ mm/s)~\cite{Deng2002}. 

\begin{figure}[htb]
        \begin{subfigure}[b]{0.42\textwidth}
                \includegraphics[width=\textwidth]{img/chapters/chapter_1_introduction/DengMACS.pdf}
                \caption{}  
                \label{fig:Deng2002}
        \end{subfigure}
        \hfill
        \begin{subfigure}[b]{0.55\textwidth}
                \includegraphics[width=\textwidth]{img/chapters/chapter_1_introduction/Saliba.pdf}
                \caption{}    
                \label{fig:Saliba2010}
        \end{subfigure}
        \caption[Examples of $\mu$-MACS integrated on a LOC or $\mu$TAS]{(a) Schematic outline of the micro-fabricated nickel post arrays using soft lithography. An externally applied magnetic field magnetizes the nickel posts. Cells labelled with magnetic particles are trapped within the post array. (b) Operating principle and practical implementation of the cell sorting system proposed by Saliba \etal{}. A hexagonal array of magnetic ink is patterned on the bottom of a microfluidic channel. The application of an external vertically-aligned magnetic field induces the formation of a regular array of magnetic particle columns localized on top of the ink dots. The particle columns then capture Raji cells. Raji cells are the first continuous human cell line from hematopoietic origin~\cite{Karpova2005}. Images in figure reproduced from~\cite{Deng2002} and~\cite{Saliba2010}.}
        \label{fig:microMacs}
\end{figure}

Saliba \etal{} essintially used the same magnetic trapping principle and magnetic particles as Deng \etal{}, however, invented an unconventional immunomagnetic separation method as shown in Figure~\ref{fig:Saliba2010}~\cite{Saliba2010}. Instead of labelling the target cells, antibody conjugated magnetic beads were dynamically assembled into an array of columns in the microfluidic channel with the help of the pre-patterned magnetic traps on the substrate. As a result, when the sample flows through the channel, cells with specific biomarkers are captured by the assembled magnetic particles. A separation efficiency of $>94\%$ was reported at a flow velocity of $\bar{u}=100$ $\mu$m/s. This flow velocity corresponds to a cell throughput in the range of ten to hundred cells/s and a flow rate of approximately $150$ $\mu$L/min in the $500$ $\mu$m wide and $50$ $\mu$m high device. Recently, this device has been applied to CTC isolation by Svobodova \etal{}~\cite{Horak2013,Svobodova2014} where they achieved a $50\%$ separation efficiency of human breast cancerous cells (ratio of collected particles in target outlet to total collected particles).

However, despite the advantages of using magnetic microparticles for performing assays, these batch mode separation techniques suffer from requiring multiple, sequential reaction and washing steps that render the procedures laborious and time-consuming, and often result in using relatively large volumes of potentially expensive reagents~\cite{Thanh2012}. Therefore, continuous separation, where magnetic particles are continuously pumped through a device and simultaneously sorted, is preferred. 

\subsection{Continuous magnetophoresis}
Continuous magnetic separation offers the potential to eliminate some of the inefficiencies of batch separation methods, allowing for an unremitting separation by combining a continuous introduction and guidance of particles through a viscous medium using continuous fluid flow. The problems encountered in the batch separation, where multiple sequential steps are required, do not occur. On the contrary, it allows for multiple staging and recirculation of sorted fractions. In this manner, larger volumes can be processed while allowing continuous monitoring~\cite{Zborowski2011}.

Various designs of continuous cell separators are reported in the literature. A common design directly integrates an electromagnet by micro-machining a wire in the vicinity of the microfluidic channel. Magnetic particles are deviated via an external magnetic field created by a current passing through the wire~\cite{Pekas2005,Derec2010,Siegel2006,Shevkoplyas2007}. 

Pekas \etal{}~\cite{Pekas2005} fabricated a fully integrated micromagnetic particle diverter in a microfluidic system. The resulting magnetic force deflects particles across a flow stream into one of the two channels at a Y-shaped junction, as shown in Figure~\ref{fig:wireIntegratedMagnetophoresis}(a)-(b). At a flow rate of $6$ nL/min ($\bar{u}\approx 1$ mm/s), $85\%$ of the micron-sized particles (Bangs Laboratories) flowed down the desired channel, compared to $50\%$ when the diverter was inactive. Derec \etal{}~\cite{Derec2010} described a similar design for a magnetophoretic bio-diverter, where a copper wire runs either side of the Y-shaped microfluidic channel (see Figure~\ref{fig:wireIntegratedMagnetophoresis}(c)). One wire was capable of producing a magnetic field between $2-7$ mT with a magnetic field gradient ranging from $3-70$ T/m within the channel. The dimensions of the etched channel were $100$ $\mu$m in height and $300$ $\mu$m in width. $5$ $\mu$m sized magnetic particles (PMPEG-5.0COOH, Kisker-Biotech) were used to label tumour cells. The device successfully purified a mixture of two different cell types by removing the labelled tumour cells. Only $7\%$ of the initial magnetically labelled tumour cells were obtained in the outlet of the nonmagnetic side, when a withdrawal rate of $0.8$ $\mu$L/min was applied to both outlets. 

\begin{figure}[htb]
        \centering
		\includegraphics[width=0.9\textwidth]{img/chapters/chapter_1_introduction/PeakasDerecMACS.pdf}
        \caption[Integrated magnetophoretic particle diverter in microfluidic systems]{(a)-(b) Top and cross-sectional view of a micromagnetofluidic diverter. Magnetic particles in the flow stream are magnetized and deflected by an external magnetic field. The magnetic field is applied by a current passing through one of the inlaid wires. As a consequence, particles are diverted into one of the desired channel. (c) Microfluidic bio-diverter with a Y-shaped junction. The copper appears in grey, and the copper-less channel-like regions appear in white. The central channel is used for the microfluidic flow (shown by the arrow), while the other channels serve to interrupt the copper and generate a narrow copper band framing the central channel. Figure reprinted with permission of Elsevier~\cite{Pekas2005} and Springer~\cite{Derec2010}.}
        \label{fig:wireIntegratedMagnetophoresis}
\end{figure}

Siegel and Shevkoplyas~\cite{Siegel2006,Shevkoplyas2007} fabricated a microfluidic channel with a $40$ $\mu\textrm{m}\times$ $40$ $\mu\textrm{m}$ cross-section, which has two metal wires either side. Passing an electrical current through one of the two wires generated a magnetic field up to $2.8$ mT and a magnetic field gradient up to $40$ T/m, which enabled magnetic particles of $6$ $\mu$m diameter (Bangs Laboratories) to be pulled to either side of the channel. By turning the electromagnets either side of the central microfluidic channel on and off, the particles alternated the direction of their movement. This electromagnetic switch could sort the magnetic particles within $1-30$ seconds, depending on the applied current. 

Although electromagnets have the advantage of being easily and rapidly switched on and off, offering system flexibility, integrating them into micro-devices often requires numerous and expensive lithography and etching techniques~\cite{Pamme2006,Ganguly2010}. This makes the fabrication process prone to errors and cumbersome. Electromagnets also suffer from Joule heating problems that have the potential to damage live cells if too extreme. For this reason, currents may need to remain in the milliamp region, therefore limiting the fields generated to $0.01-0.1$ Tesla~\cite{Kong2011}.

An easy way to avoid these disadvantages is by using soft-magnetic element systems, where external biasing is used to magnetize embedded elements or hard-magnetic element systems that essentially have the same structure but do not require current biasing. Such systems are normally referred to as high gradient magnetic separation (HGMS) devices. 

Multiple research groups are working on such HGMS devices. For example, Inglis \etal{}~\cite{Inglis2004,Inglis2006a} and also more recently Kim \etal{}~\cite{Kim2013} presented a microseparator where a ferromagnetic wire array had been inlaid in the bottom substrate of the microchannel to continuously isolate magnetically labelled white blood cells and CTCs from peripheral blood, respectively. Both groups labelled their cells with immunomagnetic nanoparticles ($50$ nm) via specific antibody binding. The microchannel used by Kim \etal{} is an asymmetric Y-junction where the centre channel is $1$ mm in width and $50$ $\mu$m in depth with the two outlet channels being $800$ $\mu$m and $200$ $\mu$m wide, respectively (see Figure~\ref{fig:Kim2013}). The externally applied magnetic field of $0.2$ T generated a magnetic flux gradient of $<40$ kT/m around the microscale ferromagnetic wires. Experimental results showed that under these conditions the microseparator isolated up to $90\%$ of CTCs from human peripheral blood at a flow rate of $5$ mL/h ($\bar{u}\approx 27.8$ mm/s). The overall isolation procedure was completed within $15$ min for $200$ $\mu$L of blood. Park \etal{}~\cite{Park2015} further developed and improved the design by Kim \etal{}. Instead of having an array of ferromagnetic wires at an angle compared to the flow direction, they fabricated an inlaid array of chevron shaped ferromagnetic wires at the bottom of the microdevice, as shown in Figure~\ref{fig:Park2015}. A separation efficiency of $93\%$ at a flow rate of $140$ $\mu$L/min ($\bar{u}\approx 15.6$ mm/s) for the separation of CTCs from human whole blood cells is reported. The time needed to isolate $400$ mL of whole human blood only took $10$ minutes.

\begin{figure}[htb!]
        \centering
        \begin{subfigure}[b]{0.8\textwidth}
                \includegraphics[width=\textwidth]{img/chapters/chapter_1_introduction/Kim.png}
                \caption{}    
                \label{fig:Kim2013}
        \end{subfigure}
        \begin{subfigure}[b]{0.8\textwidth}
                \includegraphics[width=\textwidth]{img/chapters/chapter_1_introduction/Park.png}
                \caption{}    
                \label{fig:Park2015}
        \end{subfigure}
        \caption[Lateral magnetophoresis microseparators]{(a) Perspective view of the CTC microseparator, including the inlaid ferromagnetic wire array. The magnetic force acting on the CTCs is induced by a high magnetic field gradient generated near the ferromagnetic wire array with an external magnetic field. (b) Schematic diagram showing the working principles of a microdevice for separation of CTCs using lateral magnetophoresis and immunomagnetic nanobeads. Figure reprinted with permission of American Chemical Society~\cite{Kim2013} and ETRI Journal~\cite{Park2015}.}
        \label{fig:ferromagneticWireIntegratedMagnetophoresis}
\end{figure}

Jung and Han~\cite{Jung2008} presented a method for the lateral-driven continuous magnetophoretic separation of red and white blood cells from peripheral whole blood, based on their intrinsic magnetic properties. The separation is achieved using a high-gradient magnetic field, caused by a ferromagnetic wire array inlaid on a glass substrate. The wire array creates an even lateral magnetophoretic force on the whole area of the microchannel, which improved the separation efficiency and throughput. When the flow rate and external magnetic flux were $20$ $\mu$L/h ($\bar{u}\approx 21.3$ $\mu$m/s) and $0.3$ T, respectively, the microseparator continuously separated out $93.9\%$ of red blood cells and $89.2\%$ of white blood cells from the whole blood.

Earlier, Han \etal{}~\cite{Han2006,Han2006a} showed a HGMS microseparators that could separate $93.5\%$ of red blood cells and $97.4\%$ of white blood cells from blood at a flow rate of $5$ $\mu$L/h ($\bar{u}\approx 77.2$ $\mu$m/s). They again only used the native magnetic properties of red and white blood cells, without additional markers or labelling. The magnetophoretic microseparator incorporated a nickel wire running along the centre of a microfluidic channel, as shown in Figure~\ref{fig:HanFrazier_1}. The channel featured three outlets that were used to separate red and white blood cells from diluted whole blood in continuous flow. The nickel wire was magnetised by an external magnetic field ($0.2$ T), such that as the cells passed through the chamber the red cells (paramagnetic) were attracted towards the wire, forcing them towards the centre of the chamber and therefore out of the middle outlet. However, the white cells (diamagnetic) were repelled by the field towards the outer edges of the chamber, exiting the chip via the upper and lower outlets, thereby achieving separation (see Figure~\ref{fig:HanFrazier_2}). Qu \etal{}~\cite{Qu2008} demonstrated a similar device in which a nickel wire was situated inside the chamber of a microfluidic device and used to separate red and white blood cells by their respective attraction and repulsion.

\begin{figure}[htb]
        \centering
             \begin{subfigure}[b]{0.5\textwidth}
			\includegraphics[width=\textwidth]{img/chapters/chapter_1_introduction/HanFrazier_1.pdf}
             \caption{}    
			\label{fig:HanFrazier_1}
        \end{subfigure}
        \hfill
        \begin{subfigure}[b]{0.4\textwidth}
			\includegraphics[width=\textwidth]{img/chapters/chapter_1_introduction/HanFrazier_2.pdf}
             \caption{}    
			\label{fig:HanFrazier_2}
        \end{subfigure}        
        \caption[HGMS microseparator for separating red and white blood cells]{(a) Illustration of Han and Frazier's magnetophoretic microseparator. The design incorporates a $50$ $\mu$m wide nickel wire, which is magnetized by a $0.2$ T strong external magnetic field. (b) Red and white blood cells were separated based on their intrinsic magnetic properties. Red and white blood cells in a water or plasma behave like paramagnetic and diamagnetic particles, respectively. Figure reproduced from~\cite{Han2006}.}
        \label{fig:HanFrazier}
\end{figure}

Xia \etal{}~\cite{Xia2006} integrated a micro-fabricated comb shaped HGMS geometry on one side of a microfluidic channel. When magnetised by an external permanent magnet they found that the comb structure could locally concentrate the gradient of the applied magnetic field to pull magnetic particles from one laminar flow path to another, and thus selectively remove them from a continuously flowing biological fluid, as depicted in Figure~\ref{fig:Xia2006}. The comb structure provided a magnetic field of $0.018-0.025$ T and a field gradient of at least $50$ T/m ($\approx 55-250$ T/m) across the $200$ $\mu$m wide and $50$ $\mu$m deep microfluidic channel. This device was used to remove $1.6$ $\mu$m sized magnetic particles (Bangs Laboratories) as well as magnetically labelled Escherichia coli (\textit{E. coli}) bacteria. In their study, $130$ nm magnetic particles were used to label \textit{E. coli} bacteria. Quantification of the separation efficiency for the microfluidic separation device revealed that $92\%$ of the magnetic particles exited the collection outlet at a flow rate of $40$ $\mu$L/h ($\bar{u}\approx 1.1$ mm/s) when suspended in phosphate buffered saline (PBS), whereas when mixed in isotonic saline with red dye the separation efficiency dropped to $83\%$ at a flow rate of $25$ $\mu$L/h ($\bar{u}\approx 0.7$ mm/s). \textit{E. coli} cells suspended in PBS were separated with an efficiency of $89\%$ and when suspended in isotonic saline $53\%$ could be isolated for flow rates of $30$ $\mu$L/h and $25$ $\mu$L/h, respectively.

\begin{figure}[htb]
        \centering
        \includegraphics[width=0.6\textwidth]{img/chapters/chapter_1_introduction/Xia_1.pdf}
        \caption[Micro-fabricated comb shaped HGMS microfluidic device]{Schematic depiction of the combined micromagnetic separation device for separating \textit{E. coli} that contains a microfabricated layer of soft magnetic NiFe material adjacent to a microfluidic channel with two inlets and outlets. Inset shows how magnetic particles flowing in the upper source path are pulled across the laminar streamline boundary into the lower collection path when subjected to a magnetic field gradient produced by the microfabricated NiFe layer located along the lower side of the channel. Figure reprinted with permission of Springer~\cite{Xia2006}.}
        \label{fig:Xia2006}
\end{figure}

Afshar \etal{}~\cite{Afshar2011} used multiple sharp tips on either side of the fluid channel ($200$ $\mu$m wide, $100$ $\mu$m high and $10$ mm long) to control, immobilize and release particles in a controlled manner and to continuously separate magnetic particles by size (see Figure~\ref{fig:Afshar2011}). The magnetic tips that served as field concentrators were powered by an external electromagnet. A coil current of $120$ mA generated a magnetic field of about $5$ mT between the open ends of the yoke with an estimated field gradient of approximately $90$ mT/mm perpendicular to the flow direction. Two different types of superparamagnetic particles with a diameter of $1.05$ $\mu$m (MyOne, Dynabeads) and $2.8$ $\mu$m (M270, Dynabeads) were used to characterize the system. The high field strength around the tips allowed for efficient particle manipulation. In particular, mixtures of $1.05$ $\mu$m and $2.8$ $\mu$m sized particles were reported as separated with an efficiency of $70\%-80\%$ when run under continuous conditions with a flow rate of $22$ $\mu$L/h ($\bar{u}\approx 0.3$ mm/s).

\begin{figure}[htb]
    \centering
	\includegraphics[width=\textwidth]{img/chapters/chapter_1_introduction/Afshar2011.png}
	\caption[Microfluidic HGMS devices using an electromagnet]{(a) Magnetic actuation system comprising an electromagnet (coil), a magnetic yoke and soft-magnetic poles, integrated with the microfluidic chip. Figure reproduced from~\cite{Afshar2011}.}
	\label{fig:Afshar2011}
\end{figure}

HGMS devices, however, can suffer from poor particle recovery because the embedded magnetic elements might retain a residual magnetization which causes particle trapping. Instead of having embedded magnetic elements and structures integrated in the microfluidic device and then magnetized by an external magnetic field, it is possible to generate the required non-uniform magnetic field simply using external permanent magnets. Permanent magnets may not give the flexibility of varying the magnetic strength nor can they be completely switched off, like electromagnets, but they are inexpensive and able to produce a strong magnetic field without being bulky.

One of the earliest adopters to use permanent magnetic elements was Zborowski \etal{}~\cite{Sun1998,Zborowski1999}. The group of Zborowski \etal{} has been developing FFF devices that have typically employed quadrupole magnetic fields to provide the lateral forces necessary to radially move magnetic particles in an axial flow~\cite{Zborowski2011a,Carpino2007,Williams2009}. The magnetic quadrupole configuration exerted a radially outward force on the magnetically labelled cells within a cylindrical column (see Figure~\ref{fig:Zobrowski1999}). The quadrupole generated a maximum magnetic field of $0.76$ T at the channel wall, and a magnetic gradient of $0.17$ T/mm in the radial direction. The circular flow splitter (diameter $\approx 4$ mm) separated labelled CTCs to a purity of $99.6\%$ when started with an initial concentration of $20\%$ at a flow rate for the cell sample of $0.1-0.75$ mL/min ($\bar{u}\approx 0.1-1$ mm/s). The idea of using a quadrupole arrangement of magnets has been further developed by Kennedy \etal{}~\cite{Kennedy2007}, where a quadrupole magnetic flow sorting device is modelled, designed, manufactured and tested for the successful sorting and isolation of pancreatic islets. The separation efficiency had not been reported but can be assumed similar to the one described by Zborowski~\etal{} as channel geometry and magnetic flux density are similar.

\begin{figure}[htb]
        \centering
        \begin{subfigure}[b]{0.45\textwidth}
                \includegraphics[width=\textwidth]{img/chapters/chapter_1_introduction/Zobrowski.pdf}
                \caption{}    
                \label{fig:Zobrowski1999}
        \end{subfigure}
        \begin{subfigure}[b]{0.45\textwidth}
                \includegraphics[width=\textwidth]{img/chapters/chapter_1_introduction/Kennedy.pdf}
                \caption{}    
                \label{fig:Kennedy2007}
        \end{subfigure}
        \caption[Circular quadrupole magnetic flow sorter]{(a) Schematic representation of a circular quadrupole flow sorter. The arrows indicate direction of the fluid flow. (b) Simplified cross section of a circular quadrupole flow sorter. The mixed cell/particle suspension is introduced in inlet a' while buffer fluid is introduced in inlet b'. The radially directed magnetic force leads to the movement of the labelled cells in the radial direction towards exit b, while the unlabelled cells are unable to move in the radial direction and leave the channel through exit a. Figure reproduced from~\cite{Zborowski1999} and~\cite{Kennedy2007}}
        \label{fig:quadrupoleMagneticFlowSorter}
\end{figure}

Pamme \etal{}~\cite{Pamme2004,Pamme2006a} demonstrated the use of a single off-chip permanent magnet to continuously sort magnetic particles of different sizes ($2$ $\mu$m and $4.5$ $\mu$m sized particles from Micromod and Dynabeads, respectively) from a mixture of both magnetic and nonmagnetic particles ($6$ $\mu$m polystyrene beads, Polysciences Inc.) and also successfully separated magnetically labelled macrophages and tumour cells. The particles were sorted in a rectangular $6\times6$ mm separation chamber with a depth of $20$ $\mu$m (see Figure~\ref{fig:magneticSortingChamberPamme}) based on their different magnetic mobility. The chamber had $16$ outlets and depending on how far a particle was deflected by the cylindrical magnet would exit through a different outlet. Different sizes of particles were separated with an efficiency of $75\%$ at a flow rate of $2.5$ $\mu$L/min, which resulted in a flow velocity of $\bar{u}\approx0.3$ mm/s in the separation chamber.

\begin{figure}[htb]
        \centering
		\includegraphics[width=0.5\textwidth]{img/chapters/chapter_1_introduction/Pamme2004.png}
        \caption[Free-flow magnetophoresis in a microfluidic separation chamber to sort different sized magnetic particles]{Concept of continuous magnetophoresis to separate magnetic particles of different sizes. Magnetic particles are pumped through the flow chamber. The applied magnetic field deflects the particles perpendicular to the direction of flow. Depending on the particle size and magnetic mobility, they exit the chamber through a different outlet. Reprinted with permission from~\cite{Pamme2004}.}
        \label{fig:magneticSortingChamberPamme}
\end{figure}

\subsubsection{Diamagnetic repulsion}\label{subsection:diamagneticRepulsion}
A less commonly used method for particle manipulation exploits diamagnetic repulsion, because diamagnetic repulsive forces are much weaker compared to paramagnetic (superparamagnetic) attractive forces. However, by applying very high magnetic fields and field gradients such as those generated by superconducting magnets, diamagnetic repulsion forces have been used to manipulate large diamagnetic objects~\cite{Watarai2004}. The repulsive effect of diamagnetism, however, can be greatly enhanced by suspending the diamagnetic particles in a paramagnetic solution, which then allows the use of electromagnets and strong permanent magnets, rather than bulky and expensive superconducting magnets.

Kimura~\etal{}~\cite{Kimura2004,Kimura2005} demonstrated the micro-patterning of cells and particles suspended in aqueous paramagnetic solutions. The suspensions were placed atop a magnetic field modulator constructed from alternating layers of iron and aluminium. By applying an external magnetic field of approximately $\approx 1$ T, the field modulator generated a magnetic field, which alternated between $0.2$ T and $\approx 1$ T every $300$ $\mu$m. The alternating magnetic field caused the $20$ $\mu$m sized particles and cells to migrate, forming periodic lines above the iron layers where the field was weakest. 

Winkleman~\etal{}~\cite{Winkleman2007} described the fabrication of a fluidic device for separating diamagnetic particles of different sizes and densities. The basis for the separation is the balance of the magnetic and gravitational forces on diamagnetic materials suspended in a paramagnetic medium. Diamagnetic particles with a wide size range from $5$ $\mu$m to $5$ mm were suspended in a paramagnetic  solution and continuously pumped through a funnel shaped fluid device. The funnel was exposed to an inhomogeneous magnetic field, exerted by two permanent magnets ($0.4$ T at the magnet surface) with their like poles facing. The magnetically exposed particles in the chamber were levitating at different heights depending on their density and thus exited different outlets of the fluidic device.

A similar approach for guiding differently sized particles towards different outlets based on their varying repelling force was also used by Pamme~\etal{}~\cite{Tarn2009,Peyman2009,Vojtisek2012}. The group reused their initial microfluidic chamber design, which had previously been used to sort superparamagnetic particles, but rather than using superparamagnetic particles, they suspended diamagnetic particles of $5$ $\mu$m and $10$ $\mu$m in diameter in a paramagnetic solution. The particles are no longer attracted, but repelled by the magnetic field when continuously passing through the microfluidic chamber. Full resolution of the two particle population was achieved when the flow rate did not exceed $20$ $\mu$L/h ($\bar{u} \approx 46$ $\mu$m/s). The same group also demonstrated the application of simultaneously trapping magnetic and diamagnetic particles~\cite{Peyman2009,Tarn2013}. Herein, a mixture of $2.8$ $\mu$m sized superparamagnetic particles (M270, Dynabeads) and $10$ $\mu$m sized diamagnetic polystyrene particles (Megabeads) were prepared in a paramagnetic liquid and pumped through a silica capillary ($100$ $\mu$m inner diameter). A magnet assembly of two permanent magnets on either side of the capillary with their opposite poles facing, formed a magnetic \textit{barrier}. The magnetic field, trapped superparamagnetic particles in the area with the highest flux density, whereas diamagnetic particles were pushed towards the area with the lowest flux density, as schematically depicted in Figure~\ref{fig:diamagneticParticleTrapping}. This led to the formation of two distinct particle plugs. A $100\%$ separation between the two particle types could be achieved at a maximum flow rate of $10$ $\mu$L/h ($\bar{u} \approx 0.35$ mm/s). At higher flow rates, the drag force on the diamagnetic particles was too large to successfully trap them.

\begin{figure}[htb]
        \centering
		\includegraphics[width=0.6\textwidth]{img/chapters/chapter_1_introduction/magneticAndDiamagneticTrapping.pdf}
        \caption[Diamagnetic plug trapping technique]{Schematic of simultaneously trapping superparamagnetic and diamagnetic particles. Superparamagnetic particles (red) and diamagnetic particles (grey) in a paramagnetic medium are pumped through a capillary and trapped by a pair of magnets. The superparamagnetic particles are attracted to the region between the magnets, while the diamagnetic particles are repelled and unable to pass through this magnetic \textit{barrier}. Figure adapted from reference~\cite{Tarn2013}.}
        \label{fig:diamagneticParticleTrapping}
\end{figure}

\begin{landscape}
\begin{table}[htb]
\begin{center}
\caption[Summary of continuous magnetophoretic separation techniques]{Summary of different continuous magnetophoretic separation techniques. The table compares magnetic field properties, fluid dynamic settings and channel dimensions. The separation performance is given in the column: Separation. If no information could be found to the corresponding property, it is indicated with a hyphen.}
\vspace{1ex}
\label{tab:summaryContinuousFlowSeparationTechniques} 
\begin{tabular}{l cc cc c c c}
\hline
& \multicolumn{2}{c}{Magnet} & \multicolumn{3}{c}{Fluid Channel} &  \\ 
\cmidrule(lr){2-3} \cmidrule(lr){4-6}
Magnet design & $|\mathbf{B}|$ & $\nabla|\mathbf{B}|$ & Width & Height & Flow rate & Separation  & Reference \\
 & [T] & [T/m] & [$\mu$m] & [$\mu$m] & [$\mu$L/h] & efficiency & \\
\hline
\multirow{3}{*}{Electromagnet} & 0.016 & 100-1000 & 18 & 6 & 0.36 & $85\%$ & \cite{Pekas2005} \\
& 0.002-0.007 & 3-70 & 300 & 100 & 48 & $93\%$ & \citep{Derec2010} \\
& 0.0028 & 40 & 40 & 40 & 10 & $-$ & \citep{Siegel2006} \cite{Shevkoplyas2007} \\
\cline{2-8} 
\multirow{6}{*}{\shortstack[l]{Soft-magnetic\\ element}} & $0.08$ & $5000$ & $-$ & $15$ & $-$ & $40\%$ & \cite{Inglis2004}, \cite{Inglis2006a} \\
& $0.2$ & $40000$ & $1000$ & $50$ & $5$ & $90\%$ & \cite{Kim2013} \\
& $-$ & $-$ & $3000$ & $50$ & $8400$ & $93\%$ & \cite{Park2015} \\
& $0.3$ & $-$ & $360$ & $100$ & $20$ & $93.9\%$ & \cite{Jung2008} \\
& $0.2$ & $-$ & $360$ & $50$ & $5$ & $93\% - 97\%$ & \cite{Han2006}, \cite{Han2006a} \\
& $0.3$ & $-$ & $149-287$ & $73$ & $13.8$ & $93.7\%$ & \cite{Qu2008} \\
\cline{2-8}
\multirow{2}{*}{Comb magnet} & $0.018-0.025$ & $55-250$ & $200$ & $50$ & $25-40$ & $83\%-92\%$ & \cite{Xia2006} \\
& $0.018-0.025$ & $55-250$ & $200$ & $50$ & $25-30$ & $53\%-89\%$ & \cite{Xia2006} \\
\cline{2-8}
\multirow{2}{*}{Magnetic tip} & $0.013-0.016$ & $25$ & $200$ & $50$ & $25$ & $20\%$ & \cite{Xia2006} \\
& $0.005$ & $90$ & $200$ & $100$ & $22$ & $70-80\%$ & \cite{Afshar2011}\\
\cline{2-8}
\multirow{2}{*}{Quadrupole} & $0.76$ & $170$ & $9500$ & $9500$ & $6000-45000$ & $99.6$ & \cite{Zborowski1999}, \cite{Zborowski2011a} \\
& $0.7-1.4$ & $-$ & $15900$ & $63500$ & $-$ & $-$ & \cite{Kennedy2007} \\
\cline{2-8}
\multirow{1}{*}{Bar magnet} & $0.25$ & $100-200$ & $6000$ & $25$ & $200$ & $75\%$ & \cite{Pamme2004}, \cite{Pamme2006a} \\
\cline{2-8}
\multirow{4}{*}{\shortstack[l]{Superconducting\\ magnet}}  & $1$ & $2000-3000$ & $-$ & $-$ & $-$ & $-$ & \cite{Kimura2004}, \cite{Kimura2005} \\
& $0.4$ & $20$ & $16000$ & $6000-15000$ & $-$ & $-$ & \cite{Winkleman2007} \\
& $7-10$ & $34.7-49.5$ & $6000$ & $20$ & $-$ & $-$ & \cite{Tarn2009} \\
& $7-10$ & $34.7-49.5$ & $6000$ & $20$ & $20-100$ & $-$ & \cite{Peyman2009} \\
\hline
\end{tabular}
\end{center}
\end{table}
\end{landscape}

% useful links
% https://www.ncbi.nlm.nih.gov/pmc/articles/PMC4310825/
% https://www.ncbi.nlm.nih.gov/pmc/articles/PMC3578272/
% https://www.ncbi.nlm.nih.gov/pmc/articles/PMC4331226/
% http://pubs.rsc.org/en/content/articlepdf/2014/lc/c3lc90136j
% http://www.sciencedirect.com/science/article/pii/S0003267017302337
% http://aip.scitation.org/doi/pdf/10.1063/1.1823015
% http://pubs.rsc.org/en/content/articlehtml/2017/cs/c7cs00230k
% https://link.springer.com/article/10.1007/s10404-017-1933-4

\section{Numerical particle trajectory simulation}\label{sec:particleTrajectorySimulation}
In this section, the various models for particle trajectory simulation will be discussed. The accurate prediction of the dynamical behaviour of discrete particles released in a contiuous flow of fluid is key for a better understanding and optimisation of several branches of science and technology, e.g. microfluidics, chemical processing, combustion processes or dust precipitation~\cite{Brennen2005,Jakobsen2014,Zhou2015}.

From a theoretical point of view, magnetic particle separation (magnetophoresis) is essentially a two-phase flow problem, where dispersed flow consists of discrete particles (droplets, bubbles, solids, etc.) moving in a continuous phase (air, water, etc.). To model the motion of dispersed particles in a continuous phase, two approaches are generally considered, namely the Lagrangian and Eulerian method. 

The Lagrangian method considers particles as a discrete phase and tracks the pathway of each individual particle. The detailed particle motion behaviour, which facilitates a better understanding of the physical phenomena, can be revealed by solving the Newtonian motion equations in Lagrangian coordinates while solving continuity and momentum equations for the continuum phase~\cite{Rizk1993,Gidaspow1994}. By studying the statistics of particle trajectories, the Lagrangian method is able to predict the detailed particle distributions~\cite{Fluent2009,Fluent2009a}. This requires considerable computational effort, which is why efficient algorithms for Lagrangian particle tracking in fluid flow are an ongoing research topic. To date, a number of algorithms have been implemented in discrete particle simulations. The reader can refer to~\cite{Gouesbet1999,Lain2002,Stuart2011} for the historical development of the subject.

Conversely, the Eulerian approach treats the particle phase as a continuum and develops its conservation equation on a control volume basis~\cite{Csanady1963,Wells1983,Tu1995}. Individual particle trajectories no longer explicitly appear, but are handled at a conceptual level, where a continuous scalar field (mostly particle concentration) represents the particle positions~\cite{Fluent2009,Fluent2009a}. 

%The choice of modelling approach is in essence dependent on the objective and characteristics of the problem. The Eulerian approach is commonly adopted for the prediction of interpenetrating fluid problems, where both phases can be described as a continuum~\cite{Chiesa2005}. However, the Eulerian approach can also be used to model dispersed particle solutions by collectively modelling the spatially varying concentration in each computational cell. The concentration is governed by a partial differential equation that accounts for both force-induced drift and Brownian diffusivity of the particles. To be able to describe a dispersed phase as a continuum, the volume fraction should be high and hence this approach is suitable for dense flow problems~\cite{Stenmark2013}. The Eulerian model also readily allows modelling of particle-particle stresses in dense particle flows using spatial gradients of particle volume fractions~\cite{Gidaspow1994}. However, modelling a distribution of types and sizes of particles complicates the continuum formulation of the Eulerian approach because separate continuity and momentum equations must be solved for each size and type~\cite{Rizk1993,Gidaspow1994}. If the problem involves a dilute dispersion of particles (volume fractions $\leq5\%$) with small Stokes numbers, the particle-particle collisions can be neglected and the system can be assumed to be one-way coupled, where only the fluid has an effect on the particles but not \textit{vice versa}. Under these conditions, the Lagrangian approach has shown to be more accurate compared to the Eulerian simulation approach~\cite{Squires1991,Elghobashi1992,Loomans2002,Riddle2004}. Using a continuum model for the fluid phase and a Lagrangian model for the particle phase, allows economical solution for flows with a wide range of particle types, sizes, shapes, and velocities~\cite{Gidaspow1994}. 

The performance of magnetophoretic microsystems for bioapplications can be modelled in advance of fabrication to determine system parameters that optimize the separation or capture efficiency of a target biomaterial. To date, several research groups have studied and
modelled magnetic transport and separation at the microscale. Since magnetic separation deals with dilute magnetic particle dispersions, where the particles are only a few micrometres in diameter, the Lagrangian strategy is predominately used to model magnetophoresis~\cite{Peyman2008,Lehmann2006}. Eulerian particle methods have been explored as an alternative to Lagrangian particle tracking to model the varying particle concentration for sub-micron particles where the Brownian diffusion has a significant impact~\cite{Vie2013,Laurent2013,Masi2014}. 

Furlani \etal{}~\cite{Furlani2006,Furlani2007,Furlani2010} developed magnetic particle transport models based on the Lagrangian approach to study the particle transport of magnetic particles and the continuous separation of white and red blood cells in a magnetophoretic microsystem. The presented microsystem consists of a linear array of permalloy elements embedded beneath a microfluidic channel with a rectangular cross section and dimensions of $30$ mm, $1$ mm and $120$ $\mu$m for length, width and height, respectively. They assumed an applied external magnetic field of $0.5$ T to magnetize the embedded magnetic elements. Analytical expressions were derived for the field distribution within the microchannel, and the magnetic force on the particles. The model was used to predict the magnetic response of magnetic particles and the separation of red and white blood cells. A successful separation was predicted when the cells had at least $60$ seconds to interact with the magnetic structure.  

Sinha \etal{}~\cite{Sinha2007,Sinha2009} and Nandy \etal{}~\cite{Nandy2008} studied the magnetophoretic transport of magnetic particles in a straight microchannel when a magnetic point dipole is placed adjacent to the microfluidic channel. Explicit mathematical expressions for the fluid velocity and the magnetic force are derived. The model was used to characterize the capture efficiency of the magnetophoretic separator. The analysis showed that the capture efficiency of the magnetophoretic separator depends on two dimensionless numbers that compare the magnetic force and particle drag. The two numbers incorporate all physical design parameters and operating conditions, such as flow rate, particle size, magnetic strength and microchannel geometry. Sinha \etal{}~\cite{Sinha2009} also validated the model using $1$ $\mu$m sized particles (PFCM-4056, Kisker-Biotech). The non-uniform magnetic field was established by an electromagnet with a tapered tip, that approximated the point dipole, and an adjustable current coil. The tapered tip was placed in close proximity to the microchannel and could generate a maximum magnetic flux of $30$ mT closest to the tip, which reduced to $10$ mT at a distance of $1$ mm away from it. 

Similarly, Modak \etal{}~\cite{Modak2009}, simulated the trajectories of the magnetic particles in a T-shaped microchannel under the influence of a magnetic dipole. The capturing efficiency was found to be a function of different nondimensional parameters, which incorporate dipole strength, residence time, particle launching position, etc. 

It is important to note that the aforementioned studies only considered a two dimensional (2D) model and are all assuming one-way coupling between the particles and a prescribed fully developed flow field. Only a few authors have studied magnetic particle transport in microfluidic systems taking into account two-way particle-fluid coupling wherein momentum is transferred from the particles back to the fluid phase. Khashan and Furlani~\cite{Khashan2012,Khashan2013} have studied theoretical particle and fluid transport systems comparing the simplified one-way coupling with the two-way coupling model. The model predicts the bead separation dynamics and capture efficiency using a Lagrangian approach and computational fluid dynamics. Their analysis concluded that, while one-way particle-fluid coupling enables rapid calculation, two-way coupling should be modelled whenever more accurate results are needed or particle concentrations are high. Most of the two-way coupling studies, however, have been limited in that they only consider point dipole magnetic field sources~\cite{Modak2009,Modak2010,Furlani2012}.

More recently, multiphysics packages, e.g. ANSYS or COMSOL, have been used to simulate trajectories of magnetic particles in fluid dynamic systems. Finite element methods (FEM) and the increasing computational power make it possible to simultaneously simulate multiple physical effects, such as magnetic and fluid dynamic forces, in highly complex geometries. For example, T\'{o}th \etal{}~\cite{Toth2016} studied magnetic particle trapping on a Fe-Ni grid by simulating hydrodynamic and magnetophoretic processes in COMSOL Multiphysics. Radulovi\'{c} \etal{}~\cite{Radulovic2015} investigated separation methods to separate magnetic microspheres of different sizes with radii between $1$ $\mu$m and $10$ $\mu$m. Magnetic field and force on the particles, as well as the fluid dynamics, were simulated in a 2D model in COMSOL.

Haverkort \etal{}~\cite{Haverkort2009} performed computational simulations of blood flow and magnetic particle motion to optimise magnetic drug targeting in cardiovascular diseases. For this purpose, they reproduced the complex model of a coronary artery and a carotid artery in ANSYS to investigate the capture efficiency of magnetic particles within the artery when a strong external magnetic field is applied.

Computational simulations make it possible to study the feasibility or performance of magnetic separation devices before starting experiments or entering clinical trials. Simulations are useful to investigate the influence of different factors independently, and for optimisation. However, the multiphysics packages may enable accurate 3D modelling, but are often lacking control over key parameter values and the models often make incomplete assumptions about material properties that result in inaccurate predictions.

%Abhishek \etal{} have, through simulations, identified an optimal fluid channel geometry to maximise the capture and separation efficiency of three different magnetic particle types while minimising cross-contamination.


%%%%%%%%%%%%%%%%%%%%%%%%%%%%%%%%
%%%%%%%%%%%%%%%%%%%%%%%%%%%%%%%%
% https://link.springer.com/content/pdf/10.1007%2Fs10404-013-1280-z.pdf
%Therefore, analytical models of particles’ transport are often favored to enable accurate and fast parametric optimization of microfluidic system before device fabrication. A two-dimensional (2D) analytical model of magnetophoresis has been developed for such purpose (Furlani 2006, 2007; Furlani and Sahoo 2006). Similarly, a 2D ‘‘negative magnetophoresis’’ model was later developed to investigate the transport of non-magnetic particles in magnetic fluids and its validity has been confirmed by experimental results (Zhu et al. 2011b). Here, we present a three-dimensional (3D) analytical model of microfluidic particle transport in magnetic fluids. The 2D model developed before only considers particle transport in the plane that is perpendicular to the channel depth, and assume the particle is fixed at a specific position along the channel depth. This assumption is not valid in most of realistic experimental setups, where particles are free to flow along the channel depth, on which its velocity closely depends. Our new model on the other hand provides comprehensive information of particles’ trajectory in 3D and is much closer to real experimental conditions. It takes into account important optimization parameters including fluid properties, magnet dimensions and relative positions of the magnet to the channel, and provides crucial information such as magnetic fields distribution, forces, particle velocity and trajectory in 3D. We envision this model can be used to perform quick and accurate 3D parametric optimizations for magnetic fluid-based microfluidic devices. The 3D model developed here is not limited to permanent magnet-based devices; it can also be applied toward electromagnet based microfluidic systems provided that the expressions of magnetic field distribution are known.
%%%%%%%%%%%%%%%%%%%%%%%%%%%%%%%%
%%%%%%%%%%%%%%%%%%%%%%%%%%%%%%%%

%%%%%%%%%%%%%%%%%%%%%%%%%%%%%%%%%%%%%%%%%%
% CONTINUE HERE WITH PARTICLE TRAJECTORY SIMULATION
%%%%%%%%%%%%%%%%%%%%%%%%%%%%%%%%%%%%%%%%%%

% Examples include a wide spectra of phenomena occurring in environmental (pollution dispersion in atmosphere and oceans, sedimentation in rivers), engineering (clean rooms, food and pharmaceutical industry) to biomedical (deposition of hazardous particles in the human respiratory or cardiovascular system) applications. 

%The study of flows, in particular of multiphase flows, on micro- and nanoscales is anticipated in the near future to have great impact on the design process for a number of high-throughput devices including magnetic separation devices~\cite{Abgrall2008,Hong2009}. 

% from original theory section
%To investigate particle movement and separation mechanisms in microfluidic channels, the computational simulation approach offers cost and turnaround time benefits, since it allows to explore and possibly eliminate numerous preliminary designs. In result, the manufacturing and testing of a host of competing microfluidic systems or processes can be partially replaced with modelling. 

%The multiphysics and multiscale nature of the transport phenomena occurring in  micro- and nanoscales implies that highly sophisticated numerical approaches have to be adopted. Since there is no unified approach suitable for all micro- and nanofluidic modelling problems, a number of heterogeneous numerical strategies have to be employed. 

%While both the continuum and the molecular level modelling are understood quite well, a significant modelling challenge is associated with physical phenomena, such as magnetic forces (dynamic melting), which cannot be entirely addressed within th scope of either continuum or molecular approach. 

%%%%%%%%%%%%%%%%%%% 
% https://web.stanford.edu/group/ctr/ResBriefs/2014/05_vie.pdf 
%First, if one is aiming at the statistics of the disperse phase, i.e., the values of local Number Density Function (NDF), many realizations are needed to develop converged statistics. Additionally, when the average number of particles per control volume is large, Lagrangian methods can become very expensive due to the extensive computing clock time needed to track all particles, as well as to complexities associated with the computational load balancing on parallel machines (Garcia 2009)

%Eulerian particle methods have been explored as an alternative to Lagrangian particle tracking (Druzhinin & Elghobashi 1998; Ferry & Balachandar 2001, 2002; Kaufmann et al. 2008; Masi & Simonin 2014; Masi et al. 2014; de Chaisemartin 2009b; Laurent et al. 2012; Vi´e et al. 2015). The goal of such methods is to solve the statistics of the disperse phase directly. Inspired by approaches in kinetic theory of gases (Chapman & Cowling 1939), the NDF f(t, x, vp) is defined as the number of particles per unit volume, with certain velocity, vp, averaged over many realizations. This NDF satisfies a Population Balance Equation (PBE) (referred to as the Williams-Boltzmann equation in the context of spray Williams (1958))
%%%%%%%%%%%%%%%%%%%%

% needs to go in introduction
%Numerical simulations of problems in fluid dynamics are pooled under the heading \textit{Computational Fluid Dynamics} (CFD). CFD enables numerical solutions to the flow field in various channel geometries, using a method based on approximation for solving the well established Navier-Stokes equation. Different CFD software packages are commercially available, however, in this work ANSYS Fluent will solely be used to simulate the microfluidic behaviours.

%%%%%%%%
%The Lagrangian approach for the simulation of the disperse phase is based on Newton's equation of motion. The equation of motion for small rigid spheres was proposed by Maxey and Riley~\cite{Maxey1983}.

%\subsection{Lagrangian approach}\label{subsec:lagrangianApproach}

%\subsection{Eulerian approach}\label{subsec:eulerianApproach}


%\section{FEM}
%The finite element method (FEM) is a numerical technique for obtaining approximate solutions of differential equations. It has evolved into one of the most powerful and widely used techniques for finding numerical solutions of partial differential equations that occur in engineering and science. The reason for the popularity of the FEM is due to its flexibility and its vast range of applications~\cite{Zienkiewicz1971,Strang1988,Rao2005}. 

%The fundamental principle of the FEM is to break a continuous problem into a discrete physical representation consisting of a finite number of regions or finite elements. In each element the unknown function $\Phi$ is represented by an interpolation function $\tilde{\Phi}$ with unknown coefficients. Thus, the original differential equation problem with an infinite number of degrees of freedom is converted into a problem with a finite number of degrees of freedom, or in other words, the solutions of the entire system are approximated by a finite number of unknown coefficients. So the solution derived is actually a numerical solution which applies to discrete locations.

%In order to derive the system of algebraic equations various approximate methods have been developed, and among them the Ritz variational method and Galerkin approximation method have been used most widely. ANSYS Maxwell uses the Ritz variational method which is why only this approach will be discussed here. A brief explanation to the Galerkin approximation can be found in the Appendix~\ref{sec:galerkinMethod} for the sake of completeness or in the literature~\cite{Jin2014}. 

%With the development of high-speed computers as well as advances in numerical algorithms, numerical solution methods became a new tool to solve problems for a wide range of engineering disciplines, such as civil, mechanical, electrical and chemical engineering~\cite{Martin1973,Reddy1993}. In fact, numerical solutions methods are playing an ever-increasing role as a research and design tool, as well as helping to interpret and understand the results of theory and experiment, and vice versa.
%However, to work with numerical solutions, a solid background in both, the physics and numerical analysis, is needed. 

%Similar to a theoretical approach, numerical analysis firstly represents the problems by physical models. These physical models should then be interpreted as mathematical models by establishing the governing equations as well as the boundary and initial conditions. Different from theoretical analysis, to obtain an approximate solution numerically, numerical methods adopt a discretization method which substitutes the governing differential equations by a series of algebraic equations which can then be solved on a computer. In general, for mesh-based methods, the discretization process divides the whole domain into small sub-domains in space and time, so the solution derived above is actually a numerical solution which applies to discrete locations. Thus, the accuracy of the solution is dependent on the quality and fineness of discretization, amongst other things.

%The work throughout this project has made extensive use of finite element analysis (FEA) to model the magnetic field. FEA is especially useful for calculating the magnetic fields because analytical expressions for magnetic fields are available for only the most simple geometries~\cite{Furlani2006}. The software package ANSYS Maxwell was used because it offers the possibility to model 3D geometries and is highly recognized throughout the literature. It uses the finite element method to numerically solve the discretized model.
%%%%%%%%%%%%%%%%%%%%%%%%%%%

\section{Motivation}\label{sec:motivation}
This work was initiated after discussions with a research group and clinicians at the John Radcliffe Hospital in Oxford. It was proposed that a continuous flow magnetic separator could be used for the separation and removal of a specific vesicle from a mixed vesicle sample obtained from pre-eclampsia patients. Pre-eclampsia occurs in up to $3\%$ of pregnancies and is one of the leading causes of maternal mortality in both developed as well as developing countries. It leads to the death of approximately $50,000$ women annually~\cite{Redman2005}. It is a life threatening condition to both mother and child. It results in other symptoms such as high blood pressure, stroke, protein in urine, clotting disorder and a non-functional placenta which would affect the nutritional requirements of the baby~\cite{Redman2005,Calvert2012}.

During normal pregnancy, the placenta sheds cellular vesicles into the maternal blood as part of a normal process of turnover and repair, but these vesicles are significantly elevated in pre-eclampsia and it has been hypothesized that the increased levels of cellular micro- and nano-vesicles in human fluids are associated with a wide range of diseases such as pre-eclampsia, cardiovascular disease, cancer, sepsis, HIV and diabetes~\cite{Sargent2014}.

The key to managing pre-eclampsia is early detection and diagnosis. In addition to a diagnosis based on patient symptoms, it is also possible to monitor the aforementioned vesicles and biological cells within the body which serve as a biomarker. For example, the variations in the physical structure of the nano-vesicles ($30-100$ nm) and micro-vesicles ($0.1-1$ $\mu$m) found in blood plasma can serve as indicators of a disease~\cite{Sargent2014}.

The isolation, detection and characterization of these vesicles will provide more insight on these diseases and help in providing appropriate treatment methods. However, the placental derived vesicles in the maternal plasma samples are found to be less then $1\%$ of the total vesicle population. The pre-dominant vesicle component are lipoprotein vesicles, which represent $98\%$ of all vesicles and platelet vesicles. The situation is an even greater problem in other diseases where the disease associated particles are of unknown origin and could represent a very small proportion of the vesicle population. The rare occurrence of disease associated vesicles presents a challenge for current established detection methods such as light scattering techniques.

Furthermore, cellular vesicles and lipid vesicles are of the same dimensions. This makes it difficult to use physical based separation method for this application. Currently, ultracentrifugation is used, but the procedure is time consuming and relies on the use of complicated and expensive equipment often handled by specialized and trained personnel.

The separation of such cellular vesicles with the help of magnetic particles and an externally applied magnetic field seems to be a very promising technique to isolate magnetically labelled biomarkers for various reasons given above. The use of external permanent magnets not only allows for portable and automated devices but also makes device manufacturing easier. However, permanent magnets create a unidirectional force causing particles to eventually collide with one of the inner walls of the fluidic channel. Particles being driven towards a surface are likely to be trapped in a region where the fluid velocity is zero because of the no-slip condition at the channel walls, and in some cases experience a binding force if the wall surface is oppositely charged. Such conditions adversely affect the particle throughput and continuous separation; this undesired particle loss is rarely mentioned in the literature.

The separation of placentile derived microvesicles was the initiator of this work on magnetic separation, but not the sole focus. This study extends to the separation of other biological entities including cells which, being significantly larger than microvesicles, require additional factors to be taken into account.

\section{Aims and objectives}\label{sec:aimsAndObjectives}
One of the objectives of this project was to understand how magnetic particles move when placed in a non-uniform magnetic field. This will facilitate the design of more efficient biomedical magnetic separators which can separate particles from a continuous flow without a large number of particles getting \textit{lost} in the system. Such designs highly depend on the knowledge of the properties of the magnetic particles. Knowing the magnetic responsiveness of the particles, namely the magnetophoretic mobility, helps to predict their trajectory when suspended in a liquid medium and exposed to a magnetic field. The performance of continuous separation devices can be affected by particle variations and typically requires a different set of system parameters for each particle type in order to achieve the desired separation results. This makes the use of such continuous separation systems complex for clinicians. Therefore, a robust and easy to use separation device is needed that enables the usage of various particle types without compromising the performance of diagnostic assays for clinicians. Additionally, the device should be able to isolate a bio-entity in extremely low concentrations in larger sample volumes such as might be required when screening for disease in samples of urine.

Characterization of commercially available magnetic particles has been used to improve the accuracy of the particle trajectory model. The trajectories of magnetic microparticles in a microfluidic separation channel under continuous and pulse flow condition using two novel magnet configurations are analytically studied. The model incorporates the experimentally found magnetic bead properties in order to represent the magnetic particle response as realistically as possible. Furthermore, the model is extended to a three dimensional (3D) problem, which allows the simulation of more elaborate magnet designs and offers more flexibility for the trajectory simulations. 

Theoretical investigations paired with experimental validation is rare in the literature. Thus, the new magnetic configurations are simulated and experimentally tested to validate the model and to verify the simulated magnetic field pattern. Confidence in the exact field pattern allows better control of the beads in the region where they are separated by use of an external permanent magnet, and provides accurate field data for calculating the resultant magnetophoretic forces.

The results of this work will help others make more accurate particle separation predictions and the novel magnet designs can be applied to continuous magnetic particle separation in order to contribute to the further development and improvement of inexpensive $\mu$TAS or LOC.

It is acknowledged by the author that although previous research has shown promising results, the opportunity for progression is large. For instance, more robust devices need to be developed that do not rely on accurately set flow rates or well known particle mobilities. Ease of manufacturing of such separation devices is also a vital point for the proliferation of this very promising field. Especially, the field of analytical simulation is seen to have a huge potential because it can remove the need to laboriously fabricate and test different microfluidic channels and magnetic separation designs.

%In order to realistically capture the particle response to fluid flow structures, a full one-way coupling approach is implemented. The particles are moving due to the action of gravity, buoyancy, drag, pressure gradient and added mass forces. When the fluid flows through the area of a non-uniform magnetostatic field, the resulting mangetophoretic forces working on the particles is also considered. 
%
%
%We predict microparticle trajectories and aggregation using a Lagrangian particle tracking model 
%
%theoretical investigations on large-scale particle movement inside complicated 3D microchannels are very rare. %Chao snippet
%
%The increasing interest in single-cell analysis and the rapidly improving sensitivity of molecular biology methods in applications to single-cell genome, transcriptome and signalling analysis put a special emphasis on high quality, high volume, and relatively inexpensive cell separation methods such as those provided by cell magnetophoresis in microfluidic devices.

\section{Outline}
%%%%%%%%%%%%%%%%%
%Our laboratory is interested in developing on-chip technologies for magnetic separation of living cells from biological fluids (e.g., blood, cerebrospinal fluid) which could be potentially used to develop portable devices for in-field diagnosis or therapy of diseases caused by blood-born pathogens, such as sepsis. If effective, this same type of on-chip magnetic separation technologies may also be potentially useful for isolating rare cells, such as cancer cells, stem cells or fetal cells in the maternal circulation. For these goals, it was necessary to develop a new on-chip HGMC-microfluidic approach that offers improvements over the existing designs in terms of biocompatibility, separation efficiency, and rate of clearance, while minimizing the disturbance on normal blood cells and biomolecules.
%%%%%%%%%%%%%%%%%%

After an introduction and a literature review of magnetic particle separation and its key features, the theory of magnetophoresis in microchannels as well as the modelling approach is described in Chapter~\ref{ch:chapter2_theory}. Chapter~\ref{ch:experiments} outlines the materials used in the experiments and describes the experimental procedures. In Chapter~\ref{ch:magnetophoretic_mobility}, the magnetophoretic mobility and susceptibility of five commercially available magnetic bead types are studied. In Chapter~\ref{ch:hydrodynamicFlowInAMicrofluidicDevice}, the fluid dynamics within a rectangular duct and the hydrodynamic focusing is analytically as well as experimentally investigated. Chapter~\ref{ch:magneticSeparationConfiguration} presents two new magnetic configurations and the effect on the generated magnetic field by varying key design parameters is identified. Chapter~\ref{ch:magneticParticleSeparationSimulation} discusses the separation efficiency of the previously introduced magnetic configurations by comparing numerical particle trajectory simulations and experimental results. Chapter~\ref{ch:summaryAndOutlook} summarizes all key findings and gives an outlook for future research. 

\section{Contribution}\label{sec:contribution}
In this thesis, three key contributions in the field of magnetic particle separations are given. First, the susceptibility of magnetic micron sized beads has been characterized using two independent studies and their field dependence has been proven. Second, the results of the susceptibility study led to the development of a more accurate model for magnetic separation; combining FEM, particle tracking techniques and experimentally determined bead properties. The model serves as a tool for predicting magnetic particle trajectories which will help optimise future magnetic separator designs. Third, with the help of the model, two novel magnet configurations have been investigated in detail and experimentally verified. The magnet configurations not only attract superparamagnetic particles but also exhibit a repelling force. This phenomenon has not previously been reported in the literature as a means to guide magnetic particles along specific trajectories. The proposed magnetic separator using the novel magnet configurations was shown to be successful in separating magnetic particles with a separation efficiency of $35\%$ and $80\%$ in continuous and pulsed mode operation, respectively.


% Chapter 2 describes the properties of superparamagnetic particles, including their material composition and magnetic behaviour, along with a first discussion on magnetic and hydrodynamic forces acting on the particles. Chapter 3 deals with the design and fabrication of several magneto-fluidic microsystems that are used in our studies on multiple scales. To understand the influence of local magnetic and hydrodynamic particle interactions, we subsequently describe in Chapter 4 our experiments and analyses performed on the particle dynamics effects in open fluid volumes, such as velocity measurements, the influence of chain formation and related manipulation possibilities.
%
% Our further study on the dynamics of particles confined in microchannels is described in Chapter 5. We focus on new interaction phenomena that are analyzed with the help of established analytical and numerical models, and discuss a more application oriented system with soft-magnetic flux-guides for local particle manipulation. Chapter 6 discusses our numerical studies in a more system level approach, where the fluid driving efficiency of the observed particle configurations is evaluated on the basis of fluid actuation and near surface mixing. All conclusions are listed in Chapter 7, which is finalized by an outlook containing new research questions for further studies.