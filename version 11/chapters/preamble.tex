%---------------------------------------------------------------------------
% Preface

%\chapter*{Preface}

%Blah blah \dots

%%%%%%%%
%To enable the isolation of magnetic particles from water samples, a method to continuously separate and concentrate the particles from complex samples is required. A novel microfluidic separation device is described that incorporates hydrodynamic focusing and magnetophoresis to achieve these requirements. 
%%%%%%

 \cleardoublepage

%---------------------------------------------------------------------------
% Abstract
\chapter*{Abstract}
 \addcontentsline{toc}{chapter}{Abstract}
%This work is aimed to develop a 3D model to predict the trajectories of individual magnetic beads in a magnetic field. The model is used to understand the transport of the beads in a microchannel during magnetic separation. Simulations and experiments are combined to perform a comprehensive study of the effects of multiple parameters on the efficiency of magnetic particle separation.  
 
Magnetic particle separation in microfluidic systems is a method for enriching or isolating biological entities such as cells or biomarkers of diseases using magnetic beads that covalently bind to a target; it is compatible with lab-on-a-chip technology and has potential to be integrated with other functional blocks. This thesis describes the operation of a magnetic particle separator using a novel configuration of permanent magnets to provide higher separation efficiency with minimum trapping of particles. The novel magnet configurations consist of two (Double Magnet) or four (Quadrupole) rectangular permanent magnets arranged around a microfluidic channel. The magnet configurations guide magnetic particles along adjustable trajectories by simply varying the separation distance between the magnets or their aspect ratio. Influence of the salient geometrical parameters on the behaviour of the trajectory is studied and experimental verifications are presented using particle suspensions of  $1$ $\mu$m and $2.8$ $\mu$m sized magnetic beads. In addition, it has been discovered that the magnetic beads gather along a line that can be uniquely positioned within the fluidic cell leading to improved performance. 

The performance of the magnetic separator is studied for five magnetic particle types by simulating $400$ individual particle trajectories following an Eulerian-Lagrangian model. The model incorporates experimentally found particle properties such as size and susceptibility. Magnetic separation was performed in continuous flow and pulse mode operation. In continuous separation a maximum separation efficiency of $35\%$ and $55\%$ have been experimentally found for the Double Magnet and Quadrupole configuration, respectively, and in pulse mode operation a separation efficiency of more than $80\%$ could be achieved in experiments and simulations.

Further, in order to improve the accuracy of the magnetophoretic force calculations, the magnetic responsiveness of the particle types across a magnetic field range of $38-70$ mT was studied using a particle tracking system and SQUID relaxometry. The susceptibility of all particle types showed a magnetic field dependence and a promising correlation between the two measurement techniques could be found.

%The results have enabled a better understanding of how the configuration might be used to improve separation efficiency in a continuous flow microfluidic device. 


 \cleardoublepage
 
%---------------------------------------------------------------------------
% Acknowledgement

\chapter*{Acknowledgements}

I would like to bestow significant credit for this work to Prof. Steve Sheard, who has accepted me as a research student. I could not have imagined having a better advisor and mentor for my doctoral work with such patience, motivation, enthusiasm, and immense knowledge. Without his guidance, insights, and creativity, many of the advances reported here would not have been realized. Over the course of my PhD I have learned so much and grown in a number of ways both professionally and personally thanks to a considerable part of Steve.

Very special thanks goes to Naomi Wise, who has supported me over endless hours in the laboratory and who made these sometimes frustrating and seemingly hopeless moments a lot more bearable.

I also want to thank Olayinka Oduwole and Duo Zhang for their inspiring and stimulating discussions. 

I am also  grateful for the financial support I got from the Janggen-Poehn Stiftung from the city of St. Gallen in Switzerland, which supported me financially for my time at Oxford University. 

Last but not least, I wish to express sincerely gratitude to my beloved family for their understanding and endless love and to my dear friends for their help and warm encouragement. 

 \cleardoublepage
 
%---------------------------------------------------------------------------
% Table of contents

 \setcounter{tocdepth}{2}
 \tableofcontents

 \cleardoublepage

%---------------------------------------------------------------------------
% List of figures

 \setcounter{tocdepth}{2}
 \listoffigures

 \cleardoublepage
 
%---------------------------------------------------------------------------
% List of tables

 \setcounter{tocdepth}{2}
 \listoftables

 \cleardoublepage
 
%---------------------------------------------------------------------------
% Nomenclature

\chapter*{Nomenclature}\label{chap:Nomenclature}
 \addcontentsline{toc}{chapter}{Nomenclature}

\section*{Symbols}
\begin{tabbing}
 \hspace*{1.6cm} \= \hspace*{8cm} \= \kill

 $A$ \> Cross sectional area \> [m$^{2}$] \\[0.5ex]
 $\mathbf{B}$ \> Magnetic flux density \> [T] \\[0.5ex]
 $C$ \> Constant \> [$-$] \\[0.5ex]
 $D_{H}$ \> Hydraulic diameter of channel \> [m] \\[0.5ex]
 $D_{p}$ \> Hydraulic diameter of particle  \> [m] \\[0.5ex]
 $E$ \> Energy functional \> [$-$] \\[0.5ex] 
 $\mathbf{F}$ \> Force \> [N] \\[0.5ex]
 $\mathbf{H}$ \>  Magnetic field intensity \> [A/m] \\[0.5ex]
 $H_{0}$ \>  Null hypothesis \> [$-$] \\[0.5ex] 
 $H_{a}$ \>  Alternative hypothesis \> [$-$] \\[0.5ex] 
 $H_{C}$ \>  Coercivity \> [A/m] \\[0.5ex] 
 $J$ \> Current density \> [A/m${^2}$] \\[0.5ex]
 $L_{H}$ \> Hydrodynamic entrance length \> [m] \\[0.5ex]
 $\mathbf{M}$ \> Magnetization \> [A/m] \\[0.5ex]
 $M_{S}$ \> Saturation magnetization \> [A/m] \\[0.5ex]
 $M_{R}$ \> Remanence \> [A/m] \\[0.5ex]
 $N$ \> Number of particles \> [$-$] \\[0.5ex]
 $P$ \> Wetted perimeter \> [m] \\[0.5ex] 
 $\dot{Q}$ \> Volume flow \> [m$^{3}$/s] \\[0.5ex]
 $R_{H}$ \> Hydraulic resistance \> [kg$\cdot$ s/m$^{2}$] \\[0.5ex]
 Re \> Reynolds number \> [$-$] \\[0.5ex] 
 $S$ \> Magnetophoretic driving force \> [T$\cdot$A/m$^{2}$] \\[0.5ex]
 Stk \> Stokes number \> [$-$] \\[0.5ex]
 $T$ \> Temperature \> [K] \\[0.5ex] 
 $V$ \> Volume \> [m$^{3}$] \\[0.5ex]
 $X_{i}$ \> Independent variable \> [$-$] \\[0.5ex] 
 $\hat{X}_{p}$ \> Normalized particle travelling distance \> [$-$] \\[0.5ex] 

 $a$ \> Vertical magnet separation \> [m] \\[0.5ex]
 $b$ \> Horizontal magnet separation \> [m] \\[0.5ex]
 $c$ \> Particle release distance in $z$ \> [m] \\[0.5ex]
 $d$ \> Diameter \> [m] \\[0.5ex]
 $d_{m}$ \> Distance to magnet \> [m] \\[0.5ex]
 $e$ \> Element \> [$-$] \\[0.5ex]
 $g_{w}$ \> Gap between particle and wall \> [m] \\[0.5ex]
 $h$ \> Height \> [m] \\[0.5ex]
 $k_{B}$ \> Boltzmann constant \> [m$^{2}\cdot$kg/s$^{2}\cdot$K] \\[0.5ex] 
 $l$ \> Length \> [m] \\[0.5ex] 
 $l_{w}$ \> Distance between particle midpoint and wall \> [m] \\[0.5ex] 
 $m$ \> Magnetic moment \> [A$\cdot$m$^{2}$] \\[0.5ex]
 $n$ \> Number of particles in particle chain \> [$-$] \\[0.5ex]
 $p$ \> Pressure \> [N/m$^{2}$] \\[0.5ex]
 $r_{p}$ \> Particle radius \> [m] \\[0.5ex]
 $s$ \> Sample standard deviation \> [$\ast$] \\[0.5ex]
 $t$ \> Time \> [s] \\[0.5ex]
 $\delta t$ \> Time step \> [s] \\[0.5ex]
 $\Delta t$ \> Time interval \> [s] \\[0.5ex] 
 $\mathbf{u}$ \> Velocity vector \> [m/s] \\[0.5ex]
 $\bar{u}$ \> Mean velocity \> [m/s] \\[0.5ex]
 $w$ \> Width \> [m] \\[0.5ex]
 $x$ \> Sample point \> [$\ast$] \\[0.5ex]

 $\alpha$ \> Significance acceptance level \> [$-$] \\[0.5ex] 
 $\beta$ \> Particle velocity wall correction factor \> [$-$] \\[0.5ex] 
 $\delta$ \> Fluid channel displacement \> [m] \\[0.5ex]
 $\gamma$ \> Velocity ratio \> [$-$] \\[0.5ex]
 $\eta$ \> Dynamic viscosity \> [kg/m$\cdot$s] \\[0.5ex] 
 $\epsilon$ \> Discretization error \> [V$\cdot$s/m] \\[0.5ex] 
 $\varepsilon$ \> Residual \> [A/m${^2}$] \\[0.5ex]
 $\theta$ \> Finite element aspect ratio \> [$-$] \\[0.5ex]
 $\bar{\theta}$ \> Average finite element aspect ratio \> [$-$] \\[0.5ex]  
 $\vartheta$ \> Finite element aspect ratio benchmark \> [$-$] \\[0.5ex]
 $\kappa$ \> Dimensionless scaling factor \> [$-$] \\[0.5ex]
 $\mu$ \> Mean of population \> [$\ast$] \\[0.5ex] 
 $\mu_{0}$ \> Magnetic permeability of free space \> [N/A${^2}$] \\[0.5ex] 
 $\mu_{r}$ \> Relative permeability \> [$-$] \\[0.5ex] 
 $\nu$ \> Magnetophoretic mobility \> [m$^{3}$/T$\cdot$A$\cdot$s] \\[0.5ex]
 $\pi$ \> Constant $\pi=3.14159\cdots$ \> [$-$] \\[0.5ex]
 $\rho$ \> Density \> [kg/m$^{3}$] \\[0.5ex]
 $\varrho_{in}$ \> Inner diameter of finite element \> [m] \\[0.5ex] 
 $\varrho_{out}$ \> Outer diameter of finite element \> [m] \\[0.5ex] 
 $\sigma$ \> Standard deviation \> [$-$] \\[0.5ex] 
 $\tau$ \> Relaxation time \> [s] \\[0.5ex]
 $\tau_{0}$ \> Characteristic time scale \> [s] \\[0.5ex]
 $\upsilon$ \> Volume ratio \> [$-$] \\[0.5ex] 
 $\varphi$ \> Wall distance to particle radius ratio \> [$-$] \\[0.5ex] 
 $\chi$ \> Magnetic susceptibility \> [$-$] \\[0.5ex]
 $\omega$ \> Grid element size \> [m] \\[0.5ex]
 $\xi$ \> Percentage energy error in FEM system \> [$-$] \\[0.5ex]
 $\zeta$ \> Separation efficiency \> [$-$] \\[0.5ex]

 $\Delta$ \> Difference operator, Laplace operator \> [$-$] \\[0.5ex]
 $\nabla$ \> Nabla operator \> [$-$] \\[0.5ex]
 $\Lambda$ \> Magnetic anisotropy energy density \> [J/m$^{3}$] \\[0.5ex] 
 $\Theta$ \> Dimensionless magnetophoretic velocity \> [$-$] \\[0.5ex] 
 $\Phi$ \> Magnetic vector potential \> [V$\cdot$s/m] \\[0.5ex]
 $\tilde{\Phi}$ \> Magnetic vector potential approximation \> [V$\cdot$s/m] \\[0.5ex]
 $\Omega$ \> Simulation domain \> [$-$] \\[0.5ex]
 $\partial\Omega$ \> Simulation domain boundary \> [$-$] \\[0.5ex]
 $\Gamma$ \> Domain boundary \> [$-$] \\[0.5ex]
 $\Pi$ \> Plane where beads are released from \> [$-$] \\[0.5ex]
\end{tabbing}

\section*{Indices}
\begin{tabbing}
 \hspace*{1.6cm}  \= \kill
 $a$ \> Absolute \\[0.5ex] 
 $b$ \> Elliptical particle band \\[0.5ex]
 $c$ \> Centre \\[0.5ex] 
 $d$ \> Drag \\[0.5ex] 
 $e$ \> Element \\[0.5ex] 
 $ext$ \> External \\[0.5ex] 
 $f$ \> Fluid \\[0.5ex]
 $i,j$ \> Iteration parameter \\[0.5ex]
 $k$ \> Exit channel \\[0.5ex]
 $m$ \> Magnet \\[0.5ex]
 $M$ \> Manufacturer \\[0.5ex]
% $n$ \> Number of iterations \\[0.5ex]
 $p$ \> Particle \\[0.5ex] 
 $r$ \> Relative \\[0.5ex] 
 $R$ \> Remanence \\[0.5ex] 
 $s$ \> Sheath \\[0.5ex]  
 $S$ \> Saturation \\[0.5ex]  
 $w$ \> Wall \\[0.5ex]  
 $x$ \> $x$ direction \\[0.5ex]  
 $y$ \> $y$ direction \\[0.5ex]  
 $z$ \> $z$ direction \\[0.5ex]  
 $in$ \> Incoming \\[0.5ex]
 $out$ \> Outgoing \\[0.5ex]    
 $min$ \> Minimum \\[0.5ex]
 $\parallel$ \> Parallel \\[0.5ex]  
 $\perp$ \> Perpendicular \\[0.5ex]  
 $0$ \> Fluidic channel 
\end{tabbing}

\section*{Acronyms and Abbreviations}
\begin{tabbing}
 \hspace*{1.6cm}  \= \kill
 2D \>  Two dimensional \\[0.5ex]
 3D \>  Three dimensional \\[0.5ex] 
 AC \>  Alternating current \\[0.5ex]
 ANOVA \> Analysis of variance \\[0.5ex]
 CCD \> Charge-coupled device \\[0.5ex]
 CD \> Compact disk \\[0.5ex] 
 CFC \> Circulating fetal cells \\[0.5ex]
 CFD \> Computational fluid dynamics \\[0.5ex]
 CFL \> Courant-Friedrichs-Lewy \\[0.5ex]
 CTC \> Circulating tumour cells \\[0.5ex]
 CTV \> Cell tracking velocimetry \\[0.5ex]
 DEP \> Dielectrophoresis \\[0.5ex]
 pDEP \> Positive dielectrophoresis \\[0.5ex]
 nDEP \> Negative dielectrophoresis \\[0.5ex]
 DC \> Direct current \\[0.5ex]
 DI \> Deionised water \\[0.5ex]
 DLS \> Dynamic light scattering \\[0.5ex]
 DLD \> Deterministic lateral displacement \\[0.5ex]
 DNA \> Deoxyribonucleic acid \\[0.5ex]
 \textit{E. coli} \> Escherichia coli \\[0.5ex] 
 ELISA \> Enzyme linked immunosorbent assays  \\[0.5ex]
 FFF \> Field-flow fractionation  \\[0.5ex]
 FEM \> Finite element method \\[0.5ex]
 Fe$_{3}$O$_{4}$ \> Magnetite \\[0.5ex]
% GCV \> Generalized cross-validation \\[0.5ex]
 $\gamma-$Fe$_{3}$O$_{4}$ \> Maghemite \\[0.5ex]
 HGMS \> High gradient magnetic separation \\[0.5ex]
 LFT \> Lateral flow test \\[0.5ex]
 LOC \> Lab-on-a-chip \\[0.5ex]
 LOD \> Lab-on-a-disk \\[0.5ex]
 MACS \> Magnetic activated cell sorting \\[0.5ex]
 MEMS \> Micro-electro-mechanical systems \\[0.5ex]
 MRI \> Magnetic resonance imaging \\[0.5ex]
 NdFeB \> Neodymium \\[0.5ex]
 PBS \> Phosphate buffered saline \\[0.5ex]
 PDE \> Partial differential equation \\[0.5ex]
 PDMS \> Polydimethylsiloxane \\[0.5ex]
 PCR \> Polymerase chain reaction \\[0.5ex]
 qPCR \> Quantitative polymerase chain reaction \\[0.5ex]
 POC \> Point-of-care \\[0.5ex]
 PNU \> Pusan National University \\[0.5ex] 
 SEM \> Scanning electron microscope \\[0.5ex]
 SQUID \> Superconducting quantum interference device \\[0.5ex]
 $\mu$-TAS \> Micro Total Analysis Systems
\end{tabbing}

\cleardoublepage

%---------------------------------------------------------------------------
% Publications

\chapter*{Publications}\label{chap:publications}
 \addcontentsline{toc}{chapter}{Publications}
 
\begin{itemize}
  \item[] Grob, D. T., Wise, N., Oduwole, O., \& Sheard, S. (2017). Magnetic susceptibility characterisation of superparamagnetic microspheres. \textit{Journal of Magnetism and Magnetic Materials}, 452: 134-140.

  \item[] Oduwole, O., Grob, D. T., \& Sheard, S. (2016). Comparison between simulation and experimentally observed interactions between two magnetic beads in a fluidic system. \textit{Journal of Magnetism and Magnetic Materials}, 407: 8-12.
  
  \item[] Wise, N., Grob, D. T., Morten, K., Thompson, I., \& Sheard, S. (2015). Magnetophoretic velocities of superparamagnetic particles, agglomerates and complexes. \textit{Journal of Magnetism and Magnetic Materials}, 384:328-334.

\end{itemize}


\cleardoublepage

%---------------------------------------------------------------------------
