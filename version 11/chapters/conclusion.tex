\chapter{Conclusion}\label{ch:conclusion}

This project has demonstrated the successful isolation of magnetic particles in continuous flow and pulse mode using two novel dipole configuration of rectangular permanent magnets. In particular, the focus was on controlling the region in which the particles were isolated. The first step was to conduct numerical studies to verify that there is a region where isolation was feasible, and also to predict how the geometric parameters (width and height of the magnets, and the separation between them) affected the \textit{separating power} and the location in which the particles were isolated. The studies concluded that flat magnets with low aspect ratios were preferable for the task because of a stronger magnetophoretic driving force and better compactness.

It was found that the magnetic particles 

formed an elliptical band along the contour of the equilibrium position and that both the position and width of the band could be adjusted by altering the separation between the magnets. 


The experimental results are in good agreement with the results of the finite element analysis, suggesting that there is a linear relationship between the equilibrium position and the magnet separation. Magnet tapering was also explored as a solution to the potential problem of beads becoming tapped along the equilibrium contour at the tail-end of the magnets during continuous flow conditions. The reduction in the concentration of the beads as the taper increases is indicative of a lower force, although the extent in which the force is reduced should be quantified in the future. The formation of a relatively thick equilibrium band was not predicted from theory. To gain a better understanding of why this was the case, experiments were carried out on beads that had previously been aggregated into large clusters using a separate external permanent magnet. A comparison between the cases where the beads had been pre-aggregated and those where they had not showed that the larger bead clusters formed a narrower band because of the larger force acting on them. More importantly, the images for the large bead clusters clearly show that there are gaps between the clusters where there are no beads; this could also be happening at a smaller scale in the cases where the beads were not pre-aggregated. This suggests that the formation of aggregates limits the thinness of the equilibrium band because interactions between the beads are no longer negligible. In addition to achieving bead isolation, the second aim of the project was to characterise the magnetic beads by measuring the magnetophoretic mobility. The dipole configuration was again used to generate the magnetic field. A low concentration of beads were added to an aqueous solution of polyethylene glycol $200$ to reduce the velocity and make the tracking of individual beads easier. The magnetophoretic mobility was found to have a distribution with a distinct peak just below $0.04$ mm$^3$/T-A-s.

%%%%%%%%%%%%%%%%%%
The objective of the work described in this thesis was to develop a biosensor to detect pathogenic bacteria in water. The key features of the biosensor were: to enumerate and assess the viability of captured bacteria from continuous flow, and to reduce the detection time of current microbiological methods. 

Available microbiological techniques were investigated to ascertain whether they could provide a basis for viable, rapid detection of bacteria. An MFD was designed, incorporating magnetophoresis to separate and concentrate bacteria from water. MPs were used as solid-capture bodies to isolate E. coli from water. The research conducted for this thesis is summarised and discussed in this Section.

Microfluidics enable measurements from small volumes of complex fluids with speed and sensitivity. MFDs are a powerful and versatile tool for biosensors [20]. In this study an MFD was designed to incorporate magnetophoresis to separate and concentrate bound bacteria from a sample flow. A novel quadrupole magnet configuration was designed that enabled the focussing of the MPs away from the sample flow. A separation of $80\%$ of the MPs from the sample flow was established which is a marginal improvement on the $75\%$ trapping efficiency of the MFD reported by Ramadan et al [351].
The described MFD had limitations; the magnet design encouraged surface collisions and loss of MPs. Further developments are required to optimise the configuration; these are described later in this Chapter. MFDs reported in the literature, incorporating magnetophoresis to separate micro-particles from continuous flow, have not addressed the vertical forces experienced by the MPs [e.g. 312], and the problematic removal of these MPs. This study highlights that there are additional areas within the topic of magnetophoresis that need to be addressed to better design a continuous magnetophoretic separation devices. 

MPs have been used in numerous biotechnology applications over the past decades which include areas of drug delivery [328], culture assays [352] and biomarker concentration [353]. To understand better the characteristics of MPs and their behaviour within a magnetic field, a simple method was developed to determine the magnetic mobility of MPs anda predictive model was developed to calculate the magnetic drift velocities of the MPs. It was found that some brands of MPs have better uniformity in size and magnetic content than others. When used in continuous separations, to allow for predictability, MPs should be chosen on their uniformity. Predictability of the trajectories of MPs is not as important a characteristic when using MPs in batch applications. The self-assembly of MPs into chains is a phenomenon that must be carefully considered when using MPs in a continuous flow separation device. The magnetically induced drift velocity of chains of different numbers of MPs was determined by a simple model.




