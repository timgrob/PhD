\chapter{Microfluidic separation device}\label{ch:fluidMechanics}
%\chapter{Microfluidics for magnetic particle separation}\label{ch:fluidMechanics}

%\textit{Microfluidic devices for manipulating fluids are widespread and finding uses in many scientific and industrial contexts nowadays. Applications are reaching from chemical, biological or even optical and information industry. Their design often requires unusual geometries and the interplay of multiple physical effects. Even as the basic science and technological demonstrations develop, other problems must be address to complete the cycle of development. The solutions to these problems will require imagination and ingenuity.}

\section{Introduction}\label{sec:introductionFluidMechanics}
Separation of micron-sized particles has been employed in diagnostics, chemical and biological analyses, for chemical processing and environmental assessment. The miniaturized particle separation devices offer many advantages over conventional separation techniques (e.g. density gradient centrifugation), such as small sample volumes, portability, low cost and potential for integration on a LOC or $\mu$TAS.

By employing the unique characteristics of micro-scale flow phenomena, various techniques have been established for fast and accurate separation and sorting of microparticles in a continuous manner. Magnetic separation is one of these techniques. Generally speaking, magnetic particles are transported through the microfluidic devices via the motion of fluids and are then deflected by the influence of an external magnetic field. Due to the non-linearity of the magnetic force, the particles' starting position is important to avoid surface collision. Therefore, developing the means to manipulate fluid flows in such microfluidic magnetic separation devices is a highly important issue in the design of $\mu$TAS. 

The flow focusing technique provides a particularly effective means of controlling the passage of chemical reagents or bio-samples through the microchannels of microfluidic devices and has been successfully demonstrated in a wide variety of applications including flow cytometers for cell/particle counting~\cite{Shapiro2005} and sorting or micro- and nanoparticle production~\cite{Xu2004,Kim2007,Whitesides2006}. 

In the following chapter, the basic mathematical models for the fluid velocity profile as well as the hydrodynamic focusing will be given. These models are validated by simulations as well as experiments. A new microfluidic magnetic separation device was then designed, based on a pre-existing geometry, which allows the precise positioning of a sample flow using hydrodynamic focusing. 

\section{Fluid motion in microfluidic systems}
In general, the flow field inside a fluid channel can be described by the incompressible Naiver-Stokes equation~\cite{Happel2012}:

\begin{equation}
	\rho_{f} \left( \frac{\partial\mathbf{u}}{\partial t} + (\mathbf{u}\cdot\nabla)\mathbf{u} \right) = -\nabla p + \eta_{f} \nabla^{2}\mathbf{u}
	\label{eqn:naverStokesEquation}
\end{equation}

where $\mathbf{u}$, $\rho_{f}$, $\eta_{f}$ and $\nabla p$ are the velocity vector, the fluid density, fluid dynamic viscosity and pressure gradient, respectively. The different terms in the Navier-Stokes equation (Equation~\ref{eqn:naverStokesEquation}) correspond to the inertial forces, the convection force, pressure forces and diffusion forces applied to the fluid.

On the micro-scale, laminar flow is typically achieved~\cite{Whitesides2006}. Whether a flow regime is laminar or turbulent can be determined by calculating the dimensionless Reynolds number, which represents the ratio of inertial force to viscous force:

\begin{equation}
	\text{Re} = \frac{\rho_{f}u_{f}D_{H}}{\eta_{f}}
 	\label{eqn:reynoldsNumber}
\end{equation}

Here, $\rho_{f}$ is the fluid density, $\eta_{f}$ is the fluid dynamic viscosity, $u_{f}$ is the fluid velocity and $D_{H}$ is the hydraulic diameter (characteristic length) of the fluid channel. For a rectangular duct, the hydraulic diameter, $D_{H}$ is given by the ratio between the cross section and the wetted area: 

\begin{equation}
 	D_{H} = \frac{4A}{W}
\end{equation}

with the cross sectional area $A$ and the wetted perimeter $W$ of the cross section. Going from a turbulent ($\text{Re} \gg 1$) to a laminar regime ($\text{Re} \ll 1$), or vice versa, is a very smooth transformation process without a clear transition point. In the literature, a transition threshold of $\text{Re} \approx 2300$ is normally given as a rule of thumb~\cite{Happel2012}.

In most circumstances in microfluidics, the Reynolds number is at least one order of magnitude smaller than unity, ruling out turbulence flows in the channel. At low Reynolds numbers the fluid forces are dominated by viscous drag rather than inertial forces, and the corresponding flow is characterized by smooth, constant fluid motion. For these conditions the entrance length is defined as the distance between the inlets and the point that the flow becomes fully developed. This distance is directly proportional to the Reynolds number and can be estimated by~\cite{Kays2012}

\begin{equation}
	L_{H} = 0.06 D_{H} \text{Re}
\end{equation}

For such small $\text{Re}$ number flows and the assumption that the flow is fully developed, the inertial force and the convective term become negligible. Without the nonlinear convection and time dependence the Navier-Stokes equation (Equation~\ref{eqn:naverStokesEquation}) can be simplified to the so called Stokes equation~\cite{Happel2012}: 

\begin{equation}
	\eta_{f} \nabla^{2}\mathbf{u} = \nabla p 
	\label{eqn:stokesEquation}
\end{equation}

In the absence of moving beads, the hydrodynamics of a microchannel in the limit of laminar flow is standard textbook material. Equation~\ref{eqn:stokesEquation} is the staple of introductory hydrodynamics and can be analytically solved for a number of simple geometries. The solution to Equation~\ref{eqn:stokesEquation} gives the so called Poiseuille flow for a channel of rectangular cross section and can be mathematically expressed by the infinite Fourier series summation~\cite{White2006}

\begin{equation}
	u_{f}(x,y) = \frac{4h_{0}^{2}}{\eta_{f}\pi^3}\left(-\frac{dp}{dz}\right)\sum_{k=0}^{\infty}(-1)^{k}\times\left\{ 1 - \frac{\cosh\Big[\frac{(2k+1)\pi x}{h_{0}}\Big]}{\cosh\Big[\frac{(2k+1)\pi w_{0}}{2h_{0}}\Big]} \right\} \frac{\cos\Big[\frac{(2k+1)\pi y}{h_{0}}\Big]}{(2k+1)^{3}}
	\label{eqn:velocityProfileRectangularDuct}
\end{equation}

where $h_{0}$ and $w_{0}$ are the height and the width of the fluid channel and $dp/dz$ describes the pressure drop along $z$. The pressure driven velocity profile across a cross section of a rectangular microchannel is shown in Figure~\ref{fig:velocityProfile3D}.

\begin{figure}[htb]
	\centering
   \includegraphics[width=0.75\textwidth]{img/chapters/fluidMechanics/velocityProfile3D.png}
	\caption[3D velocity profile in a rectangular duct]{3D velocity profile in a rectangular duct of a viscous fluid.}
\label{fig:velocityProfile3D}%
\end{figure}

\section{Particle laden flow in microfluidic systems}\label{sec:particleLadenFlowInMicrofluidicSystems}
Particle laden flows refer to a multiphase flow in which one of the phases is continuously connected and the other phase is made up of small, immiscible, and typically dilute particles. The above discussion on fluid motion in microfluidic systems does not take a discrete phase into account. When particles are added to the flow, they might alter the fluid properties. 

Depending on the volume fraction of the dispersed phase, the particle laden flow needs to be modelled differently. The particle volume fraction, $\upsilon_{p}$, is described by the ratio~\cite{Elghobashi1994}:

\begin{equation}
	\upsilon_{p} = \frac{V_{tot}}{V_{cell}}
\end{equation}

where $V_{tot}$ is the total volume of the solid phase in the control cell and $V_{cell}$ is the volume of the control cell. If the concentration of particles is high ($10^{-3} < \upsilon_{p}$), the particle-particle interaction and its effect on the fluid (four-way coupling) must be modelled. For intermediate concentrations ($10^{-6} < \upsilon_{p} < 10^{-3}$), particle interaction may be neglected (two-way coupling). For low concentrations ($\upsilon_{p} < 10^{-6}$), the fluid flow is not considerably influenced by the particle flow (one-way coupling)~\cite{Hryb2009}.

In most microfluidic cases, the particles occur in low concentrations and are very small, hence the dynamics are governed primarily by the continuous phase. The behaviour of the particles in a fluid flow can be characterised by the Stokes number, Stk, which is defined as the ratio of the characteristic time of the particles to a characteristic time of the flow:

\begin{equation}
 \text{Stk} = \frac{\tau_{p} u_{f}}{D_{p}}
\end{equation}

where $u_{f}$ is the velocity of the fluid, $D_{p}$ is the hydraulic diameter of the particle and $\tau_{p}$ is the relaxation time of the particle, which is given by:

\begin{equation}
	\tau_{p} = \frac{\rho_{p}d_{p}^{2}}{18\eta_{f}}
\end{equation} 

where $\rho_{p}$ is the particle density, $d_{p}$ is the particle diameter and $\eta_{f}$ is the dynamic viscosity of the fluid. 

The Stokes number is a measure of flow tracer fidelity, which means it gives information about how well particles follow the fluid flow. For large Stokes numbers (Stk $\gg 1$), particles will detach from the flow and the particle motion is weakly affected by the flow, whereas for small Stokes numbers (Stk $\ll 1$) the particles follow the streamlines closely, which means the particles are in quasi equilibrium with the surrounding flow.

In the present work, a dilute concentration ($\upsilon_{p} < 10^{-6}$) and a small Stokes number (Stk $\ll 1$) is assumed. Thus, the flow does not depend on the particle dynamics and therefore can be solved in an uncoupled way. This allows calculating the particle laden flow according to Equation~\ref{eqn:stokesEquation} and independently from other particle movements.

\section{Hydrodynamic focusing}\label{sec:hydrodynamicFocusing}
Hydrodynamic focussing is a widely used technique in the microfluidic world. It involves the reshaping and squeezing of a central stream by varying the pressure of the two sheath flows, as schematically shown in Figure~\ref{fig:hydrodynamicFocusingSchematic}~\cite{Domagalski2007,Golden2012,Dziubinski2015}. This way, one can achieve precise control of the focused sample stream width and position, which is of crucial importance to magnetic particle separation because of the non-uniform nature of the magnetic force. 

\begin{figure}[htb]
\centering
   \includegraphics[width=0.75\textwidth]{img/chapters/fluidMechanics/hydrodynamicFocusing.png}
	\caption[Hydrodynamic focusing schematic in a three inlet fluidic device]{A schematic of the three inlet channel of the microfluidic device displaying hydrodynamic focusing of a sample flow $\dot{Q}_{c}$ by varying the two sheath flows $\dot{Q}_{s1}$ and $\dot{Q}_{s2}$.}%
\label{fig:hydrodynamicFocusingSchematic}%
\end{figure} 
%\begin{tikzpicture}[scale=1,domain=0:4]
%	\draw[black!,thick] (0,1+2*1.4142) -- (1.4142,1+1.4142) -- (4,1+1.4142);
%	\draw[black!,thick] (-2,1) -- (0,1) -- (-1.4142,1+1.4142);
%	\draw[black!,thick] (-2,-1) -- (0,-1) -- (-1.4142,-1-1.4142);
%	\draw[black!,thick] (0,-1-2*1.4142) -- (1.4142,-1-1.4142) -- (4,-1-1.4142);

%	\draw[color=red] plot (\x, {1/(1+\x)});
%	\draw[color=red] plot (\x, {-1/(1+\x)});      
%\end{tikzpicture}

Due to the planar nature of most microfluidic devices only a two dimensional hydrodynamic focusing will be considered here.

To mathematically model the width of the sample stream the model for hydrodynamic focusing in a rectangular channel by Lee et al. will be followed here~\cite{Lee2006}. It uses the mass conservation principle for incompressible fluids, which states that the volume flow passing through the central inlet channel must equal the volume of fluid passing through the focused stream, and the sum of the volume flows passing through the three individual inlet channels must be the same as the total fluid passing though the main channel:

\begin{eqnarray}
	\dot{Q}_{c} &=& w_{c}h_{0}\bar{u}_{c} \label{eqn:sampleFlowMassConvervation} \\
	\dot{Q}_{c} + \dot{Q}_{s1}+ \dot{Q}_{s2} &=& w_{0}h_{0}\bar{u}_{0} \label{eqn:totalFlowMassConvervation} \\
\end{eqnarray} 

where  $\dot{Q}_{c}$,  $\dot{Q}_{s1}$ and  $\dot{Q}_{s2}$ are the volume flow of the centre sample stream and the two sheath flows, respectively. The parameter $w_{c}$ describes the width of the focused sample stream and the two velocities $\bar{u}_{c}$ and $\bar{u}_{0}$ are the mean velocity of the hydrodynamically focused flow and the mean velocity in the main channel. 

By rearranging the two equations (Equation~\ref{eqn:sampleFlowMassConvervation} and Equation~\ref{eqn:totalFlowMassConvervation}) the relationship between the width of the hydrodynamically focused stream and the channel width can be expressed as:

\begin{equation}
	\frac{w_{c}}{w_{0}} = \frac{\dot{Q}_{c}}{\gamma(\dot{Q}_{c}+\dot{Q}_{s1}+\dot{Q}_{s2})}
	\label{eqn:hydrodynamicFocusingWidthRatio}
\end{equation}

where the unknown velocity ratio $\gamma=\bar{u}_{c}/\bar{u}_{0}$ needs to be found. The two mean velocities, $\bar{u}_{c}$ and $\bar{u}_{0}$, can be found by integrating Equation~\ref{eqn:velocityProfileRectangularDuct} along $x$ and $y$, and the velocity ratio $\gamma$ can then be expressed as:

\begin{eqnarray}
	\gamma = \frac{\bar{u}_{c}}{\bar{u}_{0}} = \frac{1-\left( \frac{192 h_{0}}{\pi^{5}w_{c}}\right)\sum_{k=0}^{\infty}\frac{1}{(2k+1)^{5}}\frac{\sinh[(2k+1)\pi w_{c}/2h_{0}]}{\cosh[(2k+1)\pi w_{0}/2h_{0}]}}{1-\left(\frac{192 h_{0}}{\pi^{5}w_{0}}\right)\sum_{k=0}^{\infty}\frac{\tanh[(2k+1)\pi w_{0}/2h_{0}]}{(2k+1)^{5}}}
	\label{eqn:hydrodynamicFocusingVelocityRatio}
\end{eqnarray}

Based on Equation~\ref{eqn:hydrodynamicFocusingVelocityRatio} one can see that the hydrodynamic focusing effect does not solely depend on the relative flow rates of the sheath and sample flows but is also dependent on the aspect ratio, $h_{0}/w_{0}$,  of the microfluidic channel and the width of the hydrodynamically focused stream, $w_{c}$. Thus, one ends up with an implicit formula for $w_{c}$; but for the case where the aspect ratio is close to zero ($h_{0}/w_{0} << 1$), a velocity ratio of $\gamma \approx 1$ can be assumed~\cite{Stiles2005}.

\section{Microfluidic separation device}\label{sec:microfluidicSeparationDevice}
To enable particle separation and simultaneously have the ability to hydrodynamically focus the injected particles, a custom designed microfluidic channel was manufactured by iBidi GmbH in polydimethylsiloxane (PDMS). The design was achieved by altering an existing iBidi geometry. The new design has three inlets and three outlets and the following dimensions; length $24$ mm, width $3$ mm and depth $0.4$ mm. The three inlets and three outlets allow for hydrodynamic focusing at the inlet, and collection of the three separate fluid streams at the output. A schematic of the device is shown in Figure~\ref{fig:newMicrofluidChannelDesign}. 

\begin{figure}[htb]
        \centering
        \begin{subfigure}[b]{0.48\textwidth}
                \includegraphics[width=\textwidth]{img/chapters/fluidMechanics/customMicrofluidChannelDesign2D.png}
                \caption{}  
        \end{subfigure}
        \begin{subfigure}[b]{0.48\textwidth}
                \includegraphics[width=\textwidth]{img/chapters/fluidMechanics/customMicrofluidChannelDesign3D.png}
                \caption{}                
        \end{subfigure}
        \caption[Schematic of the customized iBidi $\mu$-slide]{A schematic of the customized iBidi $\mu$-slide, with a length of $24$ mm, a width of $3$ mm and a depths of $0.4$ mm.}
        \label{fig:newMicrofluidChannelDesign}
\end{figure}

%\begin{figure}[htb]
%   \centering
%   \includegraphics[width=0.55\textwidth]{img/chapters/fluidMechanics/customMicrofluidChannelDesign2.png}
%   \caption[Schematic of the customized iBidi $\mu$-slide]{A schematic of the customized iBidi $\mu$-slide, with a length of $24$ mm, a width of $3$ mm and a depths of $0.4$ mm.}
%   \label{fig:newMicrofluidChannelDesign}
%\end{figure}

\subsection{Hydrodynamic focusing in a microfluidic separation device}\label{sec:hydrodynamicFocusingInAMicrofluidicSeparationDevice}
The hydrodynamic focusing effect of the newly designed microfludic separation device has been numerically and experimentally tested and verified.  

Numerical simulations of problems in fluid dynamics are pooled under the heading \textit{Computational Fluid Dynamics} (CFD). CFD enables numerical solutions to the flow field in various channel geometries, using a method based on approximation for solving the well established Navier-Stokes equation. Different CFD software packages are commercially available, however, in this work ANSYS Fluent will solely be used to simulate the microfluidic behaviours. 

Due to the small size of the magnetic particles and their highly diluted concentration, the flow in the microfluidic channel can be solved in an uncoupled manner (see Section~\ref{sec:particleLadenFlowInMicrofluidicSystems}) and only the fluid phase is simulated here by using an Eulerian simulation approach.

In order to simulate the fluid dynamics in the microfluidic channel, a three dimensional symmetric geometry, representing the three inlets of the microfluidic channel was solved. The model was meshed by a fine hexahedral mesh with 464558 elements and 91424 nodes, as shown in Figure~\ref{fig:hydrodynamicFocusingSimulationMesh}. Increasing the number of points beyond this was found to have no noticeable impact on the model (grid independence).

\begin{figure}[htb]
\centering
   \includegraphics[width=0.75\textwidth]{img/chapters/fluidMechanics/hydrodynamicFocusingSimulationMesh.png}
\caption[3D mesh model of microfluidic channel]{3D mesh model used for the finite element method simulations of the hydrodynamic focusing effect.}%
\label{fig:hydrodynamicFocusingSimulationMesh}%
\end{figure} 

The different hydrodynamic flow conditions were modelled with varying aspect ratios of the input flow rates where the input flows were assumed to be water. A \textit{no-slip} boundary conditions was assumed for all the simulations. The hydrodynamic focusing effect, with the same aspect ratios as in the numerical simulation, were also experimentally verified. In order to make the sample stream visible, the water was dyed with black ink. The solution of the simulations and the experiments are shown in Figure~\ref{fig:hydrodynamicFocusingSimulationAndExperiments}. It can be seen that the simulations accurately represent the experimentally found results for different aspect ratios. 


\begin{figure}[htb]
        \centering
        \begin{subfigure}[b]{0.4\textwidth}
				\includegraphics[width=\textwidth]{img/chapters/fluidMechanics/HDF_Simulation_01_01_01_mod.png}
				\caption{1:1:1}
        \end{subfigure}
        %%%%%%%%%%%%%%%%
        \begin{subfigure}[b]{0.4\textwidth}
				\includegraphics[width=\textwidth]{img/chapters/fluidMechanics/100_100_100.png}
				\caption{1:1:1}
        \end{subfigure}
        %%%%%%%%%%%%%%%%
        \begin{subfigure}[b]{0.4\textwidth}
				\includegraphics[width=\textwidth]{img/chapters/fluidMechanics/HDF_Simulation_02_01_02_mod.png}
				\caption{2:1:2}
        \end{subfigure}
        %%%%%%%%%%%%%%%%
        \begin{subfigure}[b]{0.4\textwidth}
				\includegraphics[width=\textwidth]{img/chapters/fluidMechanics/200_100_200.png}
				\caption{2:1:2}
        \end{subfigure}
        %%%%%%%%%%%%%%%%
        \begin{subfigure}[b]{0.4\textwidth}
				\includegraphics[width=\textwidth]{img/chapters/fluidMechanics/HDF_Simulation_03_01_03_mod.png}
				\caption{3:1:3}
        \end{subfigure}
        %%%%%%%%%%%%%%%%
        \begin{subfigure}[b]{0.4\textwidth}
				\includegraphics[width=\textwidth]{img/chapters/fluidMechanics/300_100_300.png}
				\caption{3:1:3}
        \end{subfigure}
        %%%%%%%%%%%%%%%%
        \begin{subfigure}[b]{0.4\textwidth}
				\includegraphics[width=\textwidth]{img/chapters/fluidMechanics/HDF_Simulation_04_01_04_mod.png}
				\caption{4:1:4}
				\caption{Simulation}
        \end{subfigure}                        
        %%%%%%%%%%%%%%%%
        \begin{subfigure}[b]{0.4\textwidth}
				\includegraphics[width=\textwidth]{img/chapters/fluidMechanics/400_100_400.png}
				\caption{4:1:4}
                \caption{Experiment}
        \end{subfigure}         
        \caption[Hydrodynamic focusing comparison between simulations and experiments]{Comparison of the hydrodynamic focusing between the simulation and the experiment for different input flow aspect ratios. The aspect ratio is between 1 and 4 whereas the centre sample flow is kept at 100 $\mu$l/min.}
        \label{fig:hydrodynamicFocusingSimulationAndExperiments}
\end{figure}

The normalized width of the hydrodynamically focused sample stream was also compared to the mathematically developed model (Equation~\ref{eqn:hydrodynamicFocusingWidthRatio}). Figure~\ref{fig:hydrodynamicFocusingNormalizedWidth} shows the width ratio between the sample stream width and the width of the fluid channel for various aspect ratios and inlet flows. The experimental results, as well as the simulation, compare well with the theoretical values, which are calculated using Equation~\ref{eqn:hydrodynamicFocusingWidthRatio}. For micron sized particles these results (continuous phase) can also be taken for particle laden flow, because the Stokes number will be small.

\begin{figure}[htb]
\centering
\includegraphics[width=0.75\textwidth]{img/chapters/fluidMechanics/hydrodynamicFocusingIBidiDevice_errBar.eps}
\caption[Hydrodynamic focusing in iBidi $\mu$-slide: normalized width ratio for different aspect ratios]{The normalized width ratio of the sample flow ($w_{c}/w_{0}$) plotted against multiple aspect ratio for different sample flow rates. The flow rates given in the legend correspond to the centre sample flow.}
\label{fig:hydrodynamicFocusingNormalizedWidth}%
\end{figure} 

\section{Conclusion}\label{sec:conclusionFluidMechanics}
The working principle of hydrodynamic focusing is the pressure difference of the two sheath flows compared to the centre sample flow. This way, the width of the sample and its position can be hydrodynamically controlled such that the sample flow exactly meets the requirement for the separation device. In this section, the hydrodynamic focusing phenomenon is first investigated by employing potential flow theory. A model to predict the width of the focused sample stream is proposed. Then the flow field inside a rectangular duct is simulated numerically. Finally, a micromachined flow chamber is designed and fabricated on plastic susbstrates, which meets the requirements for the hydrodynamic focusing as well as for the particle separation task. Hydrodynamic focusing is verified with the use of microscopic visualization of water sheath flow and dye containing sample flow. The width of the focused stream is measured for different aspect ratios. Experimental data matched the proposed model and were also in-line with the numerical results. 

The good agreement between the theoretical model and the experiments also indicate that all assumptions are valid and the results can also be taken as an approximation for the particle laden flow. Thus, suspended magnetic particles can be accurately positioned within the microfluidic separation device, which will be relevant for magnetic particle separation. 