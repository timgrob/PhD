\chapter{Summary and Outlook}\label{ch:summaryAndOutlook}
\section{Summary of achievement}\label{s:SummaryAndContributions}
The aim of this project was to develop a model to accurately simulate the trajectories of single superparamagnetic microbeads in an external magnetic field to have a platform which can test and evaluate designs of magnetic separation devices.\ The model incorporated experimentally determined bead properties in order to best represent their behaviour in an applied magnetic field. The magnetic field was modelled in ANSYS and the magnetophoretic motion of the beads was simulated using a 3D Lagrangian tracking scheme.\ The trajectory simulations were used to analyse the separation efficiency of two novel magnet configurations under continuous flow and pulse flow mode.\ The simulations were also tested experimentally, which yielded benchmarks for validating the magnetophoretic particle transport model on a practical microfluidic platform.\\
In Chapter~\ref{ch:introduction}, an overview of the research area was given and the key concepts of microfluidic devices were defined.\ Further, various particle manipulation methods and their reported separation performance were summarized and critically compared.\ A similar summary and comparison was also given in the context of numerical modelling approaches to simulate magnetic particle trajectories.\\
In Chapter~\ref{ch:chapter2_theory}, the physical concepts of fluid flow, magnetism, magnetic forces and magnetic deflection were outlined.\ In addition, statistical data analysis techniques that are used later in the thesis were explained.\\
Chapter~\ref{ch:experiments} outlined the utilised materials and showed all chip designs used in the experiments.\ The chosen materials and devices were adequately justified.\ Additionally, all procedures necessary to repeat the practical and numerical experiments were described in this chapter.\\
In Chapter~\ref{ch:magnetophoretic_mobility}, the magnetic responses of five commercially available beads were measured at different magnetic field magnitudes.\ Further, the chapter included SEM images of the beads to confirm their size distribution.\ The study concluded that even if Chemicell's SiMAG beads showed the highest magnetophoretic mobility, Dynabeads M270 are preferable for magnetic separation processes where an accurate control is needed, which is of particular importance in continuous separation to minimise particle loss.\\
Chapter~\ref{ch:hydrodynamicFlowInAMicrofluidicDevice} successfully demonstrated the hydrodynamic focusing effect within a microchannel.\ The hydrodynamic focusing effect was verified by simulations and experiments for different flow rates and flow rate ratios while remaining laminar flow.\ The sample stream could be accurately positioned and its width could be reduced to $4\%$ of the channel width.\\
Chapter~\ref{ch:magneticSeparationConfiguration} presented two magnet configurations for magnetic separation, i.e. Double Magnet configuration and Quadrupole configuration.\ It was discovered that both configurations exhibit regions where magnetic particles experience no lateral magnetic force but formed a focussed band.\ The aim of the magnet configuration study was on controlling the region in which the beads are isolated and what effect different magnet configuration parameters have on the focussing band.\ The study has successfully demonstrated isolation of hydrodynamically pre-focused magnetic beads into controlled regions with various particle types.\ The formation of the focussing band was also studied for different particle types, particle concentrations and time intervals.\\
In Chapter~\ref{ch:magneticParticleSeparationSimulation}, the separation efficiency of both magnet configurations was simulated using a 3D Lagrangian tracking scheme and experimentally verified.\ The model simulations showed that the Double Magnet configuration, under continuous flow conditions, could achieve a relative separation efficiency of $100\%$ at a flow rate of $10$ $\mu$L/min or smaller.\ If the flow rate was increased, the relative separation efficiency was found to decrease.\ There also was a negative correlation between relative separation efficiency and particle throughput, i.e. improving throughput was detrimental to relative separation efficiency.\ Regarding throughput, simulations predicted that $85\%$ of all the introduced particles can be recovered at a flow rate of $\dot{Q}=375$ $\mu$L/min.\ The Double Magnetic configuration was also experimentally evaluated in this study.\ The experimentally determined separation efficiencies approximately matched the simulation results.\ Different scenarios were considered, which indicated that multiple effects and their combination play a role during magnetic separation.\\
The particle separation efficiency and throughput can be improved using a Quadrupole magnet configuration operating in pulse mode.\ An absolute separation efficiency of $80\%$ could be achieved in the experiments and was confirmed by simulations.\ The pulse mode also had the advantage of minimising the number of beads lost in the system.\ This is especially important for biological separation where the targeted entities appear in small quantities such as required in the diagnosis of pre-eclampsia which is dependent on the small number of placental derived vesicles in the maternal blood plasma referred to in the thesis.\\
Existing clinical purification systems, e.g. the KingFisher Flex system by ThermoFisher Scientific or the MagNA Pure System by Roche, use tube based batch processes.\ These systems offer a fully automated solution to purify a range of samples (e.g. whole blood or body fluids).\ The protocol performs multiple washing steps and particle isolation in a 96-well microtiter plate in $40-60$ min.\ The separation configurations presented in this thesis only perform the particle isolation step, but with the two proposed operation modes, continuous or pulsed, the sample throughput is at least one order of magnitude larger compared to the commercially available batch separation systems mentioned above.\ The tube based systems are limited by the 96-well microtiter plates which have a working volume of $200-300$ $\mu$L.\ This makes the presented approach particularly useful to processing large sample volumes containing low concentrations of biomarkers, e.g. urine samples.\ To date, pre-eclampsia is detected by traces of proteins found in urine.\ Therefore, the proposed magnetic separation device could already have an application for detecting pre-eclampsia when combined with the antibody binding molecule for placentile derived vesicles developed at the JR Hospital in Oxford~\cite{Dragovic2013}.\\
The achieved separation efficiency, in particular in pulsed mode, is comparable with what has been previously reported in the literature (see Table~\ref{tab:summaryContinuousFlowSeparationTechniques}).\ Research groups that have reported a separation efficiency of $90\%$ or higher have mostly soft magnetic elements or use an electromagnet integrated into the fluid channel.\ This makes the fabrication process of the fluid channel more complex and thus potentially prone to errors and irregularities.\ With the separation geometry presented in this thesis, the magnet configuration can be kept separate from the fluid channel while keeping the ability to guide and focus magnetic beads along a predefined trajectory.\ The width of the focussing band where magnetic beads migrate to is independent of the beads used and thus, the magnetic separation process is less prone to variations in bead properties.\\
Having a robust and easy to use diagnostic assay that can process large sample volumes has been one of the key objectives of this thesis.\ The proposed magnetic separation system is relevant to processing sample volumes up to $\approx 100$ mL/h.\ Further, the two magnet configurations with their focussing band achieved a robust magnetic particle separation in a test environment.\ However, the loss of beads within the fluid channel remained a problem and needs to be addressed in the future to enable a more practical separation device.\\
In order to have an easy to use diagnostic assay that can potentially be used by clinicians also requires an easy setup and analysis process.\ The reduced fabrication costs ultimately allow for a disposable separation device, which will reduce the risk of sample contamination between separation processes and make the handling of the device easier.\ Thus, the device developed in this thesis would best be suited for the use by a laboratory technician since setting up the device still requires training.\ In order to also have the device used by clinicians, the device would greatly benefit from integrated downstream sample analysis and tests.\ So far, solutions for routine assays have automated the sample preparation, but no system exists to the author's knowledge that also performs sample analysis.\ Therefore, having the analysis integrated on the same device would be a further step towards a POC device that could eventually be used by patients.\\
It is further acknowledged by the author that the proposed device has only been tested with magnetic particle suspensions and buffer solutions and not on real clinical samples.\ Clinical samples are far more complex mixtures but the basic principles of operation will remain the same.\\
While the modelling method developed here was developed for magnetophoretic separations, on the wider scale the method could also be applied to similar on-chip continuous flow devices that involve the lateral deflection of particles flowing through liquid via the use of an external forces.\ A variation on microparticle device application could be the detection of specific proteins on the surface of cells by using antibody binding of magnetic nanoparticles (rather than micron-sized beads) to these surface proteins.\ The exact number of magnetic nanoparticles bound to the cell would not be of relevance due to the gathering band.\\
The magnetic separation configuration could also be used to promote a chemical reaction on different reagents attached to the surface of mobile magnetic particles by moving the magnetic particles laterally across different solution streams.\ The lateral movement of the particles can be controlled by controlling the gathering band, which can be adjusted by machining the magnets.\\
Additionally, the method reported in this thesis for determination of magnetic susceptibility could be used by manufacturers of magnetic particles for quality control of their magnetic beads.\\

%%%%%%%%%%%%%%%%%%%%%%%%%%%%%%%%%%%%%%%%%%%%%%%%%%%%%%%%%%%%%%%%%%%%%%%%%%%%%%%%
% Clinical samples are far more complex mixtures which can introduce additional challenges; e.g. variations in viscosity when dealing with whole blood samples, which can also cause clogging of the fluid channel or the collected output sample might get contaminated by other particles in the mixture that also interact with the magnetic particles.\ But, even if the presented platform had been used in a highly controlled testing environment, it serves well as a proof-of-concept.\\

% inspiration
% with a number of parameters studied that showed the focussing was improved with increased residence time between two magnets, larger particles, and higher magnetic susceptibility of the paramagnetic solution
%
%Previously, the platform had been applied to proof-of-principle bioassays394,424 and DNA hybridisation,425,426 to great effect. Here, studies were undertaken to determine its applicability to a number of different areas, including its use for (i) chemical reactions such as the fluorescamine reaction and peptide synthesis, (ii) the deposition of polyelectrolyte layers onto magnetic templates, and (iii) clinically relevant sandwich immunoassays such as the detection of the C-reactive protein biomarker.
%
% Despite encountering difficulties, some of the main goals of this work were achieved to a degree, including the ability to perform a chemical reaction (amide bond formation) on the surface of mobile magnetic particles, and the deposition of a polyelectrolyte layer onto magnetic templates in continuous flow
%
% Tarn thesis
%The results demonstrate the importance of temperature control during particle separations, as fluctuations in the system temperature could lead to poor reproducibility.
%
%%%%%%%%%%%%%%%%%%%%%%%%%%%%%%%%%%%%%%%%%%%%%%%%%%%%%%%%%%%%%%%%
% First version of summary
%%%%%%%%%%%%%%%%%%%%%%%%%%%%%%%%%%%%%%%%%%%%%%%%%%%%%%%%%%%%%%%%
%
%The aim of this project was to develop a model to accurately simulate the trajectories of single superparamagnetic microbeads in an external magnetic field to have a platform which can test and evaluate designs of magnetic separation devices.\ The model incorporated experimentally determined bead properties in order to best represent their behaviour in an applied magnetic field.\ The magnetic field was modelled in ANSYS and the magnetophoretic motion of the beads was simulated using a 3D Lagrangian tracking scheme.\ The trajectory simulations were used to analyse the separation efficiency of two novel magnet configurations under continuous flow and pulse flow mode.\ The simulations were also tested experimentally, which yielded benchmarks for validating the magnetophoretic particle transport model on a practical microfluidic platform.\ The focus of the magnet configuration study was on controlling the region in which the beads are isolated and what effect different magnet configuration parameters have on the separating efficiency.\ This thesis has successfully demonstrated isolation of hydrodynamically pre-focused magnetic beads into controlled regions.\ 
%
% An accurate magnetic separation model and a high degree of control over the magnetic particles requires a good understanding of the properties of the magnetic beads.\ The magnetic response of five commercially available beads have been measured at different magnetic field magnitudes.\ The study concluded that even if Chemicell's SiMAG beads showed the highest magnetophoretic mobility, Dynabeads M270 are preferable for magnetic separation processes where an accurate control is needed, which is of particular importance in continuous separation to minimize particle loss.\ This conclusion will guide researchers working on separation devices as the accurate characterisation of magnetic beads is rare in the literature.
%
%The model simulations have shown that the Double Magnet configuration, used under continuous flow conditions, could achieve a relative separation efficiency of $100\%$ at a flow rate of $10$ $\mu$l/min or smaller.\ If the flow rate is increased, the relative separation efficiency was found to decrease.\ There is a negative correlation between relative separation efficiency and particle throughput, i.e.\ improving throughput was detrimental to relative separation efficiency.\ Regarding throughput, simulations predict that $85\%$ of all the introduced particles can be recovered at the highest flow rate explored ($\dot{Q}=375$ $\mu$l/min).\ 
%
%The Double Magnetic configuration was experimentally evaluated under the same conditions.\ The experimentally determined separation efficiencies approximately matched the simulation results.\ Different scenarios were considered, which indicated that multiple effects and their combination play a role during magnetic separation.\ Further analysis needs to take the interplay of these effects and other factors (e.g.\ particle-particle interaction) into account.
%
%The particle separation efficiency and throughput could be improved using a Quadrupole magnet configuration operated in pulse mode.\ An absolute separation efficiency of at least $80\%$ could be achieved in the experiments and this was confirmed by simulations.\ The pulse mode also has the advantage of minimising the number of beads lost in the system.\ This is especially important for biological separation where the targeted entities appear in small quantities such as required in the diagnosis of pre-eclampsia which is dependent on the small number of placental derived vesicles in the maternal blood plasma referred to in the thesis.\ 
%
%With the two proposed operation modes, continuous and pulse, the sample throughput is much larger compared to other batch separation techniques.\ This makes the presented approach particularly useful to process large sample volumes, e.g.\ urine samples.\ To date, pre-eclampsia is detected by traces of proteins found in urine.\ Therefore, the proposed magnetic separation device could already have a real life application for detecting pre-eclampsia when combined with the anti-body binding molecule for placentile derived vesicles developed at the JR Hospital in Oxford~\cite{}.
%
%%%%%%%%%%%%%%%%%%%%%%%%%%%%%%%%%%%%%%%%%%%%%%%%%%%%%%%%%%%%%%%%
%%%%%%%%%%%%%%%%%%%%%%%%%%%%%%%%%%%%%%%%%%%%%%%%%%%%%%%%%%%%%%%%
%
% The purified sample could then be used for further development in a chain of downstream analys
%
%The driving force for miniaturization is an increasing demand for low-cost instruments capable of rapidly analysing compounds in very small sample volumes with a high level of automation.\ The magnet configurations investigated have the unique characteristics to be able to align magnetic particles along a customized path depending on the magnets' shape.\ This allows for a simple geometry of the microfluidic device with no expensive integrated magnetic components because the particles will be guided by an external magnetic field.\ Thus, the microfluidic device, due to its small size and simple geometry, can be cheaply manufactured; whereas the magnetic setup, which is required to generate a large enough magnetic field, can be kept externally and separately.\ As the cost per microfluidic device is lowered, it becomes economically feasible to use such devices as disposable items.
%
%The increasing interest in magnetic microparticles and their application in bioanalysis fuels the development of Lab-on-a-Chip (LOC) or Micro Total Analysis System ($\mu$TAS) employing them as active components.\ However, to have accurate prediction on magnetic particle separation, their magnetic properties need to be known.\ Therefore, the magnetic response of commercially available superparamagnetic particles was measured.\ More particularly, the magnetophoretic mobility and the corresponding susceptibility of these particles was determined.\ The particles showed a nonlinear behaviour between the applied magnetic field ($30-65$ mT) and their magnetization.\ This nonlinearity might need to be taken into account in HGMS devices, because these separation devices often produce magnetic fields sufficiently strong ($\geq 0.5$ T) to saturate the magnetic particles.\ 
%
%The purified sample could then be used for further development in a chain of downstream analys 
%
%We determined the magnetic field in the device by numerical simulations, which were consistent with experimetnal measures of the magnetic force exerted on calibrated magnetic beads % Derec
%
%Here we extended the concept of particle mobiliy to magnetic separations and demonstrated its usefulness in designing a novel type of magnetic separator based on a continuous flow through process.\ % Zborowski
%
%This thesis work looked into the theoretical and numerical analysis of magnetic particle separation.
%
%An analytical model has been developed for the Lagrangian tracking of magnetophoretic motion of magnetic microspheres in a fully developed flow inside a microchannel under the influence of a magnetic field.\  
%
%Magnetic separation permits the isolation of target cells from heterogeneous, crude samples [231].\ Microfluidics has increased in popularity within scientific research over the past decade.\  The versatility they offer in design allows for an array of applications.\ Laminar flow that is achievable at the micro-scale enables highly predictive characteristics and for continuous separation to be achieved.\ The ability to incorporate automation and analysis on-chip is an attractive characteristic.\ Microfluidic separations are still troubled by trapped air-bubbles and blockages.\ Such phenomena are mitigatable, however are still an occurrence that can disrupt separation.\ The low flow limits observed in this experiment will restrict the amount of sample that can be processed.\ Running multiple MFDs, alongside each other will increase the concentration and separation of larger volumes, and could also facilitate multi-analyte detection, which is preferable when monitoring water for contaminants.% Naomi
%
%Devices for biosensing often contain high-surface-area elements for efficient reaction kinetics.\ Within microfluidic applications it is a challenge to find an integrated fluid actuation technique that facilitates pumping with reasonable flow rates, while having a robust control on local fluid replenishment and mixing near no-slip surfaces.\ 
%
%It was found that an easy to manufacture system could be fabricated to accurately, hydrodynamically focus a core stream in three-dimensions.
%
%More particularly the emphasis, leading to this choice of system, was on reducing the effect of unbalanced forces arising in more common planar geometries.\ The first step was to conduct simulations and numerical studies to verify that beads could be decentralised uniformly from a central focus point, whilst in a continuous flow.\ It was then necessary to find flow regimes and geometrical parameters that could best optimise the separation efficiency.
%
%This study concluded that cylindrical magnets with higher separation ratios were necessary to obtain a more uniform magnetic field around the circumference of channels.\ This was seen as an important factor because it led to a more uniform radial transport of beads in all directions around the axis of the channel.\ This in turn allowed the magnetically induced transport of beads to be predicted more accurately.
%
%Numerical analysis showed how hydrodynamic focussing could be used to control two critical cases of bead mobilities which would result in successful separation.\ Calculations helped identify the flow regimes leading the widest range between these two critical mobility cases.
%
%Experiments also showed the effects that higher concentration of beads focussed to relatively small radii had on their agglomeration.\ The direct effect of which was to increase the overall mobility of beads.\ It was found that a precise control over the viscosity of the buffer and the suspension solution and of the throughput of the flows could shift the range of bead mobilities.\ Centring this range around the mean value of mobility acquired by agglomeration of beads could guarantee $100\%$ separation.
%
%This project has demonstrated the successful isolation of magnetic beads in hydrostatic conditions using a novel dipole configuration of rectangular permanent magnets.\ In particular, the focus was on controlling the region in which the beads were isolated.\ The first step was to conduct numerical studies to verify that there is a region where isolation was feasible, and also to predict how the geometric parameters (width and height of the magnets, and the separation between them) affected the separating power and the location in which the beads were isolated.\ The studies concluded that flat magnets with low aspect ratios were preferable for the task because of a stronger magnetophoretic force and better compactness.\ % ciaran
%
%This project successfully demonstrated the isolation of magnetic beads in continuous flow conditions using a novel quadrupole configuration of magnets.\ It was found that an easy to manufacture system could be fabricated to accurately, hydrodynamically focus a core stream in three-dimensions.\ % Harold
%
%Our measurements in open fluid volumes revealed a large spread in single particle velocities.\ Recently published research on this matter supports the indications we found for the polydisperse nature of superparamagnetic particles, and shows that particles can differ a lot in susceptibility [79] and even show a remanent magnetic moment [40].\ The particle dynamics and interaction effects we have analyzed will be affected by a spread in particle properties.
%
%This work has highlighted the various characteristics of the developed MPC.\ The MPC has a variety of interesting properties such as superparamagnetic characteristics, bio-compatibility, and high mechanical stability.\ It can be remotely heated by applied magnetic fields.\ A key advantage of the composite is the homogeneous dispersion of the nanoparticles with low particle agglomerations, which enables the fabrication of microstructures with small feature sizes.\ In the following possible future applications with the composite are presented % marcel suter
%
%%%%%%%%
% 3D hydrodynaic focusing --> to enable continuous separation
% the model needs to be tested on variation of particle agglomerations and size distribution --> also check on beads economy
% look at multiple cells/ larger cells/ multiple particles
% experimentally test puls cycle
% collection pots integrated --> to have further analysis steps easily carried out --> Syphoning effect eliminated
% have the particles aligned along one region and then have them coated with specific target cells
%%%%%%%%
%
\section{Future Work}\label{sec:futureWork}
The devices proposed here could be a valuable alternative for magnetic separation by guiding particles along a customized path, with high potential for further development in a cascade of downstream analysis of the sorted particles.\ In order to move toward a commercial application, using the proposed magnet configurations, certain components of the existing design need further investigation.\ 
%The key components of magnetophoretic systems are the magnetic particles, the magnetic field and the microfluidic channel.\ Therefore, these are the three things which need further investigation.
%
\subsection{Simulation of suspension with higher order agglomerates}
It has been shown in this work that different effects, e.g. particle agglomerations, can reduce the separation efficiency.\ In future analysis, the model should be extended to enable particle suspension simulations which include a distribution of higher order agglomerates.\ This would allow determination of the impact on separation efficiency of agglomerates in a particle suspensions.\
%
\subsection{Magnetic microparticles synthesis}
Magnetic microparticles, which are the central element of the magnetic separation, should be investigated more thoroughly.\ Only when their physical structure is fully understood, one can make accurate predictions about their magnetic behaviour.\ Especially, the iron content and its vari ation among different particles lead to variability in the performance of magnetic particle separation.\ Scanning electron microscope or transmission electron microscope can give further insight into the particle structure.\ However, in order to optimise the properties of magnetic particles for continuous flow separation it is better to synthesise the particles \textit{in house} or obtain them through a collaboration with an external group, as reported in reference~\cite{Zborowski2002}.\ There is a significant interest in the synthesis of superparamagnetic microparticles for a range of applications~\cite{Ma2003,Liu2003,Liu2004}.\
%The market size of magnetic beads has grown from $225$ million USD in $2013$ to $236$ million USD in $2016$ and is estimated to surpass $242$ million USD by $2021$ according to a magnetic bead market report~\cite{•}.
%
\subsection{Optimisation of magnet geometry and configuration}
This work has shown that it is possible to guide magnetic particles along a well-defined trajectory by using a Double Magnet or Quadrupole configuration.\ However, vertical forces negatively affected the particle separation efficiency.\ It has been shown that by physically shaping the magnets, (e.g. by tapering), vertical forces can be reduced and the gathering points can be adjusted for specific microfluidic channel geometries.\ This work has demonstrated the effect of tapered magnets and parabolically shaped magnet edges, but further optimisation is possible.\ Future work should analyse different tapering angles and parabolic magnet shapes, as well as their combination, to generate a magnetic field which bests suits the purpose of continuously separating magnetic beads, while maximising bead throughput.\\
With the Double Magnet or Quadrupole configuration in place, magnetic beads align in specific areas, and can be guided along defined trajectories.\ This opens the potential to manipulate by magnetophoresis the beads along complex trajectories to per perform specific tasks, e.g. transferring across a multilaminar flow to initiate a chemical reaction or binding biomarkers to the magnetic beads.\

\subsection{Localised particle injection}
The particles' trajectory is influenced by their initial release positions.\ The variation in position results in different magnetic drift forces and interaction time with the magnet and is mainly responsible for particles getting trapped in the system.\ A more accurate localized particle injection, using 3D flow focussing prior to particles entering the channel, could improve reproducibility and increase their separation efficiency, especially in continuous particle separation where the interaction time with the magnet is key.\ Beads could be accurately injected into the inlet channel through a nozzle.\ Figure~\ref{fig:nozzleInjection} shows a \textit{sweet spot} where beads should be continuously injected in order to avoid particle-wall interaction.\ However, in order to achieve a high separation efficiency, position, velocity and magnetophoretic mobility of the particles need to be optimised.\
\begin{figure}[htb]
\centering
\includegraphics[width=0.6\textwidth]{img/chapters/chapter_8_future_work/nozzleInjectionPosition.png}
\caption[Bead injection through a nozzle]{Preferred initial release position of magnetic beads to achieve a high continuous separation efficiency in a double magnet configuration. The different colours correspond to the following: \textit{green}$-$beads get successfully separated and \textit{blue}$-$beads are trapped in the system. The open and filled circles represent starting and end position of the beads, respectively. This figure shows the continuous separation at $\dot{Q}=10$ $\mu$L/min.}
\label{fig:nozzleInjection}
\end{figure}
For a commercial application, the ease of use is important.\ The prototype presented in this thesis took half an hour or more to setup and the bead collection process required significant practice.\ With further development, it would be entirely possible to fabricate a disposable separation device requiring minimal setup, and in which the buffer and bead suspensions can be injected under automation, being drawn and delivered to solution reservoirs.\ This would remove some of the labour intensive steps in diagnostic protocols such as the ELISA analysis, and promote the integration of other microfluidic based systems.\ This could lead to more complex diagnostic biochemistry being performed on very low concentration of target biomarkers, and offer cheap, fast and highly specific sensors.\
%
%%
% Future work would involve using a flow focussing effect prior to particles entering the chamber, such that they would all follow reproducible trajectories.\ A further improvement could also be the use of narrower outlets to collect fractions that are only slightly separated.\  It would also be interesting to actually tune the temperature of the system in order to direct particles to specific outlets, then change it slightly so that they exit via a different outlet, which would allow different downstream procedures to be performed simultaneously on the same particle types
%%
% Once the causes and solutions to these problems can be determined, the multilaminar flow device will show far greater diversity in what it can be used to achieve, and this remains a priority for future investigations otherwise the system will find its applicability somewhat limited.
%%
% Some future applications of the multilaminar flow device may also require longer residence times for particles, or more reagent streams if there are a large number of steps in a reaction/assay.\  In both of these cases the number of inlets can simply be increased to allow a greater number of reagent streams.\  Future devices would also benefit from a particle focussing inlet channel, as the current devices suffer from having a great deal of particle spread as they traverse the channel due to the variation in starting position when they enter the chamber.\  By focussing the particles when they enter the chamber they would each have the same starting position, hence they would experience the same magnetic forces and follow the same trajectory through the chamber, thereby increasing reproducibility in the results